\section{Các hàm và tính toán}
Trước khi thảo luận về khả năng của máy tính, ta môt tả mối liên hệ giữa giải quyết bài
toán và tính toán hàm.

Một \textit{hàm} theo nghĩa toán học là một tương ứng giữa một tập các giá trị đầu vào và
một tập các giá trị đầu ra sao cho với mỗi đầu vào ta chỉ có tương ứng một đầu ra duy
nhất. Một ví dụ là hàm là chuyển đơn vị yard thành mét. Ở đây, mỗi giá trị tính theo yard
được đặt tương ứng với một giá trị tính theo mét và hai giá trị này phải phản ánh cùng một
độ dài. Một ví dụ khác là hàm sắp xếp, nó gắn mỗi danh sách các giá trị số đầu vào với một
danh sách các giá trị số đầu ra chính là dãy đầu vào nhưng đã được sắp xếp theo chiều tăng
dần. Một ví dụ khác nữa là hàm cộng, nó gắn một cặp giá trị số đầu vào với một giá trị đầu
ra là tổng của cặp số đó.

Quá trình xác định các giá trị đầu ra cụ thể tương ứng với giá trị đầu vào được gọi là
\textit{việc tính toán các hàm}. Khả năng tính toán các hàm là quan trong bởi vì chính
cách tính toán các hàm mà ta có thể giải quyết các bài toán. Ví dụ, để giải bài toán cộng,
ta phải tính toán hàm cộng; để sắp xếp một danh sách, ta phải tính toán hàm sắp xếp. Theo
cách này, một nhiệm vụ cơ bản của khoa học máy tính là tìm cách tính toán các hàm liên
quan đến bài toán mà ta muốn giải quyết.

Ta xem xét, ví dụ, một hệ thống tính hàm trong đó các đầu vào và đầu ra của hàm có thể
được xác định trước và ghi trong một bảng. Mỗi khi cần tính đầu vào của hàm, ta chỉ cần
tìm vị trí đầu vào ở trong bảng và ở vị trí tương ứng ta tìm thấy đầu ra. Bởi vậy việc
tính toán các hàm được rút gọn thành quá trình tìm kiếm trong bảng. Các hệ thống kiểu này
tiện lợi nhưng bị giới hạn do có nhiều hàm không thể được biểu diễn đầy đủ theo dạng
bảng. Ví dụ là bảng ở Hình~\ref{fig:fig111}, bảng này cố gắng biểu diễn hàm chuyển đơn vị
đo từ yard sang mét. Rõ ràng bảng này không đầy đủ bởi vì danh sách các khả năng cho cặp
đầu vào/đầu ra là không bị giới hạn.

\begin{figure}
\label{fig:fig111}
  \begin{center}
    \begin{tabular}{cc}
      \hline
      \textbf{Yard}      & \textbf{Mét}      \\
      \textbf{(đầu vào)} & \textbf{(đầu ra)} \\ \hline
      $1$                & $0.9144$          \\
      $2$                & $1.8288$          \\
      $3$                & $2.7432$          \\
      $4$                & $3.6576$          \\
      $5$                & $4.5720$          \\
      $\vdots$           & $\vdots$          \\
      \hline
    \end{tabular}
  \end{center}
\caption{Biểu diễn không đầy đủ của hàm chuyển đơn vị đo yard thành mét }
\end{figure}


Một cách tiếp cận khác hiệu quả hơn để tính toán hàm là dùng công thức đại số thay vì cố
gắng liệt kê mọi tổ hợp đầu vào/đầu ra trong một bảng. Ví dụ, ta có thể dùng công thức đại
số:
\[
V = P (1 + r)^n
\]
để mô tả cách tính giá trị của một khoản đầu tư $P$ sau $n$ năm biết tỉ lệ lãi kép là $r$.

Nhưng khả năng biểu diễn của các công thức đại số cũng có giới hạn. Có những hàm mà quan
hệ giữa đầu vào và đầu ra quá phức tạp để mô tả theo cách này. Ví dụ các hàm lượng giác
như $\sin$ hoặc $\cos$. Nếu cần tính toán $\sin$ của góc $38$ độ, bạn phải vẽ tam giác
thích hợp, đo các cạnh của nó, và tính toán tỉ lệ mong đợi--đây là một quá trình không thể
biểu diễn dùng các phép toán đại số của giá trị $38$. Tất nhiên, máy tính cầm tay của bạn
có thể cố gắng tính được $\sin$ của góc $38$ độ. Thực ra, nó cố gắng áp dụng các kỹ thuật
toán học tinh tế để đạt được một giá trị xấp xỉ khá tốt với $\sin$ của $38$ độ, và ghi ra
kết quả cho bạn.

Ta đã thấy, với hàm phức tạp, ta phải áp dụng các kỹ thuật phức tạp để tính toán. Một câu
hỏi đặt ra là phải chăng với mọi hàm phức tạp tuỳ ý, ta luôn tìm thấy một hệ thống để tính
toán nó. Câu trả lời là không. Một kết quả đặc sắc trong Toán học khẳng định rằng có những
hàm không thể được định nghĩa dựa trên một quá trình từng bước xác định đầu ra dựa trên
các giá trị đầu vào. Các hàm này được gọi là hàm không tính được; còn các hàm mà các đầu
ra có thể được xác định từ đầu vào bằng phương pháp thuật toán được gọi là \textit{tính
  được}.

Việc phân biệt giữa các hàm tính được và không tính được là rất quan trọng trong khoa học
máy tính. Bởi vì các máy chỉ có thể thực hiện các nhiệm vụ được mô tả bởi thuật toán, nên
việc nghiên cứu các hàm tính được chính là nghiên cứu khả năng của máy. Nếu ta xác định
được các chức năng cần thiết để một máy có thể tính được mọi hàm tính được thì việc xây
dựng các máy có đủ các chức năng này đảm bảo rằng chúng có mọi khả năng mà ta có thể mong
muốn. Ngược lại, nếu ta khám phá ra rằng lời giải của một bài toán cho trước đòi hỏi phải
tính một hàm không tính được, vậy ta có thể kết luận rằng lời giải của bài toán đó vượt ra
ngoài khả năng của máy.

\subsection*{Câu hỏi \& Bài tập}
\begin{enumerate}
\item Hãy chỉ ra một vài hàm có thể biểu diễn đầy đủ dưới dạng bảng.

\item Hãy chỉ ra một vài hàm mà đầu ra của nó có thể được mô tả bởi các biểu thức đại số
  theo đầu vào.

\item Hãy chỉ ra một hàm không thể mô tả dùng công thức đại số. Dù thế, có phải hàm này là
  tính được?

\item Các nhà toán học cổ Hy Lạp đã biết cách dùng thước thẳng và com-pa để vẽ hình. Họ đã
  phát triển các kỹ thuật tìm trung điểm của đoạn thẳng, dựng các góc vuông, và vẽ các tam
  giác đều. Tuy vậy, có những ``tính toán'' gì mà ``hệ thống tính toán'' của họ không thực
  hiện được?
\end{enumerate}


%%% Local Variables: 
%%% mode: latex
%%% TeX-master: "../tindaicuong"
%%% End: 
