\section{Bài tập cuối chương}
  

\begin{multicols}{2}
  \begin{enumerate}

  \item Chỉ ra xem làm thế nào cấu trúc có dạng:
    \begin{flushleft}
      \quad\texttt{while X equals 0 do;} \\
      \quad\quad \texttt{.} \\
      \quad\quad \texttt{.} \\
      \quad\quad \texttt{.} \\
      \quad\texttt{end;}
    \end{flushleft}
    có thể được mô phỏng với Bare~Bones.

  \item Viết một chương trình Bare~Bones đặt giá trị $1$ vào biến \texttt{Z} nếu biến
    \texttt{X} nhỏ hơn hoặc bằng biến \texttt{Y}, ngược lại đặt giá trị $0$ vào biến
    \texttt{Z}.

  \item Viết một chương trình Bare~Bones đặt \texttt{X} lần giá trị $2$ (hay $2^X$) vào
    biến~\texttt{Z}.

  \item Trong mỗi trường hợp sau đây hãy viết một đoạn chương trình Bare~Bones thực hiện
    các việc được chỉ ra dưới đây:
    \begin{enumerate}
    \item Gán \texttt{Z} bằng $0$ nếu \texttt{X} là số chẵn; ngược lại gán \texttt{Z} bằng
      $1$.

    \item Tính tổng các số nguyên từ $0$ đến \texttt{X}.
    \end{enumerate}


  \item Viết một đoạn chương trình Bare~Bones chia giá trị \texttt{X} cho giá trị
    \texttt{Y}. Ta quy ước bỏ qua phần dư; có nghĩa rằng, $1$ chia $2$ bằng $0$ và $5$
    chia~$3$ bằng $1$.

  \item Mô tả hàm tính bởi chương trình Bare~Bones dưới đây, với giả sử đầu vào biểu diễn
    bởi \texttt{X} và \texttt{Y}; còn đầu ra biểu diễn bởi bởi \texttt{Z}:
    \begin{flushleft}
      \quad\texttt{copy X to Z;} \\
      \quad\texttt{copy Y to Aux;} \\
      \quad\texttt{while Aux not 0 do;} \\
      \quad\quad \texttt{decr Z} \\
      \quad\quad \texttt{decr Aux} \\
      \quad\texttt{end;}
    \end{flushleft}

  \item Mô tả hàm tính bởi chương trình Bare~Bones dưới đây, với giả sử đầu vào biểu diễn
    bởi \texttt{X} và \texttt{Y}; còn đầu ra biểu diễn bởi bởi \texttt{Z}:
    \begin{flushleft}
      \quad\texttt{clear Z;} \\
      \quad\texttt{copy X to Aux1;} \\
      \quad\texttt{copy Y to Aux2;} \\
      \quad\texttt{while Aux1 not 0 do;} \\
      \quad \quad\texttt{while Aux2 not 0 do;} \\
      \quad \quad\quad \texttt{decr Z} \\
      \quad \quad\quad \texttt{decr Aux2} \\
      \quad \quad\texttt{end;} \\
      \quad \quad \texttt{decr Aux1;} \\
      \quad\texttt{end;} \\
    \end{flushleft}

  \item Viết một chương trình Bare~Bones tính tuyển loại của hai biến \texttt{X} và
    \texttt{Y}, đặt kết quả vào biến \texttt{Z}. Giả sử rằng~\texttt{X} và~\texttt{Y} khởi
    tạo với giá trị chỉ bằng~$0$ hoặc~$1$.

  \item Chỉ ra rằng nếu ta cho phép các lệnh trong Bare~Bones được gán nhãn bởi một giá
    trị nguyên và thay thế cấu trúc lặp \texttt{while} với lệnh nhảy điều kiện biểu diễn
    dưới dạng
    \begin{flushleft}
      \texttt{if \textit{tên} not 0 goto \textit{nhãn};}
    \end{flushleft}
    với \texttt{\it tên} có thể là mọi biến và \texttt{\it nhãn} là một giá trị nguyên
    được sử dụng để gán cho một lệnh nào đó, thì ngôn ngữ mới vẫn là ngôn ngữ lập
    trình phổ dụng.

  \item Trong chương này ta đã thấy cách cài đặt lệnh
    \begin{flushleft}
      \texttt{copy \textit{tên1} to \textit{tên2};}
    \end{flushleft}
    trong Bare~Bones. Chỉ ra xem làm thế nào lệnh này có thể được cài đặt nếu cấu trúc
    \texttt{while} được thay thế bởi cấu trúc lặp sau
    \begin{flushleft}
      \texttt{repeat \dots until (\textit{tên} equals 0)}
    \end{flushleft}

  \item Chỉ ra rằng ngôn ngữ Bare~Bones có thể vẫn là ngôn ngữ lập trình phổ dụng nếu lệnh
    \texttt{while} được thay thế bởi cấu trúc lặp sau 
    \begin{flushleft}
      \texttt{repeat \dots until (\textit{tên} equals 0)}
    \end{flushleft}

  \item Thiết kế một máy Turing chỉ dùng duy nhất một ô trên băng nhưng nó không bao giờ
    đạt được trạng thái dừng.

  \item Thiết kế một máy Turing đặt các số~$0$ vào mọi ô bên trái của ô nhớ hiện thời cho
    tới khi nó gặp một ô nhớ chứa dấu sao.

  \item Giả sử xâu gồm các số $0$ và $1$ trên băng của một máy Turing được giới hạn bởi
    các dấu sao ở hai đầu. Thiết kế một máy Turing quay xâu bít này sang phải một ô, giả
    sử rằng máy bắt đầu với ô nhớ hiện hành chứa dấu sao bên phải nhất của xâu.

  \item Thiết kế một máy Turing đảo xâu bít~$0$ và $1$ nằm giữa ô nhớ hiện hành (có chứa
    dấu sao) và ô nhớ chứa dấu sao đầu tiên bên trái.

  \item Tóm tắt luận đề Church-Turing.

  \item Chương trình Bare~Bones sau đây có tự kết thúc? Giải thích câu trả lời của bạn
    \begin{flushleft}
      \texttt{copy X to Y;} \\
      \texttt{incr Y;} \\
      \texttt{incr Y;} \\
      \texttt{while X not 0 do;} \\
      \quad \texttt{decr X;} \\
      \quad \texttt{decr X;} \\
      \quad \texttt{decr Y;} \\
      \quad \texttt{decr Y;} \\
      \texttt{end;}\\
      \texttt{decr Y;} \\
      \texttt{while Y not 0 do;} \\
      \quad \texttt{incr X;} \\
      \quad \texttt{decr Y;} \\
      \texttt{end;}\\
      \texttt{while X not 0 do;}\\
      \texttt{end;}\\
    \end{flushleft}


  \item Chương trình Bare~Bones sau đây có tự kết thúc? Giải thích câu trả lời của bạn.
    \begin{flushleft}
      \texttt{while X not 0 do;}\\
      \texttt{end;}\\
    \end{flushleft}

  \item Chương trình Bare~Bones sau đây có tự kết thúc? Giải thích câu trả lời của bạn.
    \begin{flushleft}
      \texttt{while X not 0 do;}\\
      \quad \texttt{decr X;}\\
      \texttt{end;}\\
    \end{flushleft}

  \item Phân tích tính hợp lệ của cặp lệnh sau đây:
    \begin{flushleft}
      \texttt{Lệnh sau là đúng.}\\
      \texttt{Lênh trước là sai.}\\
    \end{flushleft}

  \item Phân tích tính hợp lệ của khẳng định ``Người đầu bếp trên một con tàu nấu ăn cho
    mọi người và chỉ những những người không nấu được cho bản thân anh ta.'' (Ai nấu ăn
    cho người đầu bếp?)

  \item Giả sử rằng bạn ở trong một thành phố mà mỗi người hoặc là kẻ nói thật, hoặc là kẻ
    nói dối. (Một người nói thật luôn nói sự thật, một người nói dối luôn nói dối.) Bạn có
    thể dùng câu hỏi gì để hỏi một người và biết anh ta là nói thật hay nói dối?

  \item Tóm tắt ý nghĩa của các máy Turing trong lý thuyết khoa học máy tính.

  \item Tóm tắt ý nghĩa của bài toán dừng trong lý thuyết khoa học máy tính.

  \item Giả sử rằng bạn cần phải tìm ra xem có ai trong nhóm của đã sinh vào một ngày đặc
    biệt nào đó. Một cách tiếp cận là có thể hỏi các thành viên vào một lúc nào đó. Nếu
    bạn theo cách tiếp cận này, sự xuất hiện của sự kiện nào có thể chỉ ra cho bạn rằng có
    một thành viên như vậy? Sự kiện nào có thể chỉ cho bạn rằng không có thành viên nào
    như vậy? Bây giờ giả sử rằng bạn phải tìm ra xem có ít nhất một số nguyên dương có
    tính chất đặc biệt nào đó và bạn áp dụng cùng cách tiếp cận kiểm tra một cách có hệ
    thống các số nguyên cùng một lúc. Nếu, trên thực tế, có nhiều số nguyên có tính chất
    này, làm thế nào bạn có thể tìm ra? Nếu, tuy vậy, không có số nguyên nào có tính chất
    này làm thế nào bạn có thể tìm ra? Có phải nhiệm vụ kiểm tra xem một giả thuyết là
    đúng có nhất thiết đối xứng với nhiệm vụ kiểm tra xem nó có sai hay không?


  \item Có phải bài toán tìm kiếm một giá trị đặc biệt trong danh sách là bài toán đa
    thức? Chứng minh câu trả lời của bạn.

  \item Thiết kế một thuật toán quyết định xem một số nguyên có là nguyên tố hay không. Có
    phải lời giải của bạn là hiệu quả? Lời giải của bạn là trong thời gian đa thức hay
    không?

  \item Có phải một lời giải đa thức cho một bài toán luôn tốt hơn so với một lời giải mũ?
    Giải thích câu trả lời của bạn.

  \item Có phải một bài toán có một lời giải đa thức có nghĩa rằng nó luôn được giải trong
    thời gian có thể chấp nhận trong thực tế? Giải thích câu trả lời của bạn.


  \item Lập trình viên Charlie được giao bài toán phân nhóm (của một số chẵn người) vào
    thành hai nhóm con rời nhau và có số người bằng nhau sao cho sự khác biệt về tuổi của
    mỗi nhóm con là lớn nhất có thể. Anh ta đề nghị một lời giải là liệt kê mọi khả năng
    chia thành hai nhóm, tính toán sự khác biệt về tuổi của mỗi cặp, và lựa chọn ra cặp
    nhóm có khác biệt nhiều nhất. Lập trình viên Mary có một giải pháp khác, chị đề nghị
    rằng đầu tiên sắp xếp nhóm ban đầu theo tuổi và sau đó chia thành hai nhóm con. Mỗi
    nhóm con gồm một nửa từ nửa những người trẻ hơn và nửa còn lại từ những người già
    hơn. Hãy chỉ ra độ phức tạp của mỗi lời giải? Có phải bài toán này có độ phức tạp là
    đa thức, $\mathbf{NP}$, hay không đa thức?

  \item Tại sao cách tiếp cận sinh mọi khả năng có thể sắp xếp danh sách sau đó chọn một
    cách sắp xếp mong muốn không phải là cách hợp lý để sắp xếp một danh sách?

  \item Giả sử trò chơi xổ số dựa trên việc chọn đúng bốn số nguyên, mỗi số trong khoảng
    từ $1$ đến $50$. Hơn nữa giả sử rằng giải thưởng là quá lớn đến mức mà người mua sẽ có
    lợi nếu anh ta mua một vé số riêng cho mỗi tổ hợp số. Nếu bạn mất một giây để mua một
    vé, bạn sẽ mất bao lâu để mua một vé cho mỗi tổ hợp số? Thời gian này sẽ thay đổi thế
    nào nếu công ty xổ số yêu cầu chọn năm số thay vì bốn? bài toán này phải làm gì với
    kiến thức trong chương này?

  \item Thuật toán dưới đây có đơn định không?
    \begin{description}
    \item [] \textsl{\textbf{procedure} mystery (Number)} 

    \item []\textsl{\textbf{if} (Number > 5)} 

    \item[] \textsl{\textbf{then} (trả lời ``yes'')} 

    \item [] \textsl{\textbf{else} (lấy một giá trị nhỏ hơn $5$ và đưa số này ra làm câu
        trả lời)}
    \end{description}

  \item Thuật toán dưới đây có đơn định không? Hãy giải thích câu trả lời của bạn.
    \begin{description}
    \item []\textsl{Lái xe đi thẳng.}

    \item [] \textsl{Đến chỗ đường giao thứ ba, hỏi người đứng ở góc đường xem nên rẽ phải
        hay rẽ trái.}

    \item[] \textsl{Rẽ theo hướng người đó chỉ.}

    \item []\textsl{Lái xe qua thêm hai khối nhà và dừng ở đó.}
    \end{description}

  \item Xác định các điểm không đơn định trong thuật toán sau:
    \begin{description}
    \item []\textsl{Chọn ba số giữa $10$ và $100$.}

    \item [] \textsl{\textbf{if} (tổng của số được chọn lớn hơn~$150$)}

    \item[] \textsl{\textbf{then} (trả lời ``yes'')}

    \item []\textsl{\textbf{else} (Chọn một trong ba số đã chọn và đưa ra số ngày như câu
        trả lời)}
    \end{description}

  \item Thuật toán sau đây có độ phức tạp đa thức hay không đa thức? Giải thích câu trả
    lời của bạn.
    \begin{description}
    \item []\textsl{\textbf{procedure} mystery (ListOfNumbers)}

    \item [] \textsl{Chọn một tập các số trong ListOfNumbers.}

    \item[] \textsl{\textbf{if} (các số trong tập này cộng thêm với $125$)}

    \item []\textsl{\textbf{then} (Câu trả lời ``yes'')}

    \item []\textsl{\textbf{else} (Không đưa ra câu trả lời)}

    \end{description}

  \item Các bài toán dưới đây có thuộc lớp~$\mathbf{P}$?
    \begin{enumerate}
    \item Một bài toán với độ phức tạp~$n^2$

    \item Một bài toán với độ phức tạp~$3n$

    \item Một bài toán với độ phức tạp~$n^2 + 2n$

    \item Một bài toán với độ phức tạp~$n!$
    \end{enumerate}

  \item Tóm tắt sự phân biệt giữa khẳng định một bài toán là bài toán đa thức và khẳng
    định một bài toán là bài toán đa thức không đơn định.

  \item Đưa ra một ví dụ về bài toán thuộc cả lớp $\mathbf{P}$ và $\mathbf{NP}$.

  \item Giả sử rằng bạn được đưa ra hai thuật toán để giải cùng một bài toán. Một thuật
    toán có độ phức tạp thời gian~$n^4$ và một thuật toán có độ phức tạp~$4n$. Với kích
    thước đầu vào như thế nào thì thuật toán đầu hiệu quả hơn thuật toán sau?

  \item Giả sử rằng bạn phải đối mặt với bài toán người bán hàng du lịch trong trường hợp
    số thành phố là $15$ trong đó giữa mọi cặp thành phố đều nối trực tiếp với nhau bởi
    một con đường duy nhất. Có bao nhiêu đường đi khác nhau có thể có đi qua các thành phố
    này? Ta mất bao lâu để tính độ dài của mọi đường đi này biết rằng việc tính độ dài của
    một đường đi mất một micro-giây?

  \item Hãy đưa ra một ví dụ cho mỗi bài toán thuộc một phạm vi biểu diễn trong
    Hình~\ref{fig:fig1112}.

  \item Thiết kế một thuật toán tìm nghiệm nguyên của phương trình có dạng $x^2 + y^2 =
    n$, với $n$ là một số nguyên được đưa vào. Xác định độ phức tạp thời gian thuật toán
    của bạn.

  \item \label{ex:1145} Một bài toán khác cũng thuộc lớp NP đầy đủ là \textbf{bài toán
      sắp ba lô} (Knapsack problem). Bài toán này yêu cầu tìm các số trong danh sách sao
    cho tổng của các số này bằng một giá trị đặc biệt. Ví dụ, ba số $257, 388$ và~$782$
    trong danh sách
    \[
    642, 257, 771, 388, 391, 782, 304
    \]
    có tổng là $1427$. Hãy tìm các số trong danh sách sao cho tổng của chúng là
    $1723$. Bạn áp dụng thuật toán nào? và độ phức tạp của nó là gì?

  \item Hãy xác định điểm tương đồng giữa bài toán người bán hàng du lịch và bài toán sắp
    ba lô (xem bài tập~\ref{ex:1145}).
  \end{enumerate}

  
\end{multicols}


%%% Local Variables: 
%%% mode: latex
%%% TeX-master: "../tindaicuong"
%%% End: 
