\section{Các ngôn ngữ lập trình phổ dụng}
Trong Chương~\ref{ch6} ta đã nghiên cứu nhiều đặc điểm của ngôn ngữ lập trình bậc
cao. Trong mục này ta sẽ áp dụng kiến thức về tính toán để xác định những đặc điểm nào là
thực sự cần thiết. Ta sẽ thấy rằng hầu hết những đặc điểm của ngôn ngữ bậc cao ngày nay
thực ra chỉ giúp tiện lợi khi sử dụng hơn là đóng vai trò làm tăng khả năng cơ bản của
ngôn ngữ.

Cách tiếp cận của ta là mô tả một ngôn ngữ ra lệnh cho phép ta biểu diễn các chương trình
tính mọi hàm tính được bởi máy Turing (hay tương đương là mọi hàm tính được). Bởi vậy, nếu
một lập trình viên trong tương lai thấy một bài toán không thể giải được dùng ngôn ngữ
này, lý do sẽ không phải do hạn chế của ngôn ngữ mà là do không có thuật toán để giải
quyết bài toán đó. Một ngôn ngữ lập trình với tính chất này được gọi là \textbf{ngôn ngữ
  lập trình phổ dụng}.

Bạn sẽ ngạc nhiên khi thấy rằng một ngôn ngữ lập trình phổ dụng không nhất thiết phải phức
tạp. Thay vào đó, ngôn ngữ mà ta trình bày khá đơn giản. Ta sẽ gọi nó là Bare~Bones bởi vì
nó chỉ gồm tập tối thiểu các yêu cầu của một ngôn ngữ lập trình phổ dụng.

\subsection*{Ngôn ngữ Bare Bones}

Ta bắt đầu bằng việc xem xét các khai báo trong các ngôn ngữ lập trình. Các khai báo này
cho phép người lập trình suy nghĩ về các cấu trúc dữ liệu và kiểu dữ liệu (như mảng các
giá trị số hay xâu ký tự thuộc bảng chữ nào đó) dù máy chỉ đơn thuần thao tác với các bít
chứ không quan tâm đến các xâu biểu diễn gì. Trước khi đưa vào máy để thực hiện, một lệnh
ở mức cao liên quan đến các kiểu dữ liệu và cấu trúc phải được dịch thành các lệnh ở mức
máy để thao tác với các xâu bít để mô phỏng các hoạt động được yêu cầu.

Để thuận tiện, ta có thể diễn giải các xâu bít này như những biểu diễn nhị phân của các
giá trị nguyên không âm. Bởi vậy, không mất tổng quát mọi tính toán thực hiện bởi một máy
có thể được xem như các tính toán liên quan đến các số nguyên không âm. Và các ngôn ngữ
lập trình có thể được đơn giản hoá bằng cách yêu cầu người lập trình biểu diễn thuật toán
theo cách này (dù rằng điều này có thể gây khó khăn hơn trong lập trình).

Vì mục đích của ta là phát triển ngôn ngữ Bare~Bones đơn giản nhất có thể nên ta sẽ chọn
cách diễn giải này. Có nghĩa rằng, mọi biến trong Bare~Bones sẽ được xem xét như một dãy
bít, và được diễn giải như một số nguyên không âm theo ký hiệu nhị phân. Ví dụ, biến có
giá trị là xâu bít $10$ sẽ được gọi là chứa giá trị hai, và biến được gán bởi xâu bít
$101$ được gọi là chứa giá trị năm.

Theo cách này, mọi biến trong một chương trình Bare~Bones có cùng kiểu, nên ta không cần
khai báo xem tên biến và kiểu của nó là gì. Khi dùng Bare~Bones, lập trình viên có thể
dùng luôn biến mới khi cần, và hiểu rằng nó luôn tham chiếu tới một xâu bít diễn giải như
một giá trị nguyên không âm.

Tất nhiên, bộ dịch cho ngôn ngữ Bare Bone phải có khả năng phân biệt các tên biến với các
từ khác. Điều này được chỉ ra bởi cú pháp của Bare~Bones sao cho vai trò của mọi từ được
xác định bởi cú pháp duy nhất. Với mục đích này, ta quy định rằng tên biến phải bắt đầu
bởi một chữ trong bảng chữ tiếng Anh, và theo sau bởi một tổ hợp các chữ hoặc số (từ $0$
đến $9$). Vậy các xâu \texttt{XYZ}, \texttt{B747}, \texttt{abcdefghi}, và \texttt{X5Y} có
thể được sử dụng như tên biến, trong khi \texttt{2G5}, \texttt{\%o}, và \texttt{x.y} không
phải.

Ta cùng xem xét các lệnh trong Bare~Bones. Ngôn ngữ này có ba lệnh gán và một lệnh điều
khiển biểu diễn việc lặp. Ngôn ngữ này không có định dạng (free-format), bởi vậy mỗi lệnh
phải kết thúc bởi một dấu chấm phảy, điều này để dễ dàng dịch các lệnh riêng nhưng đặt
trên cùng một dòng. Tuy vậy, để dễ đọc ta thường đặt chỉ một lệnh trên một dòng.

Cả ba lệnh gán của ngôn ngữ Bare~Bones đều yêu cầu các nội dung của biến gán cũng chính là
biến yêu cầu được thay đổi trong lệnh. Lệnh đầu tiên cho phép ta gán giá trị không cho một
biến. Cú pháp của nó là
\begin{flushleft}
 \qquad \qquad \qquad \texttt{clear \textit{tên};}
\end{flushleft}
với \texttt{tên} có thể là mọi tên biến có thể.

Hai lệnh gán còn lại về cơ bản là ngược nhau
\begin{flushleft}
  \qquad\qquad\qquad\texttt{incr \textit{tên};}
\end{flushleft}
và
\begin{flushleft}
 \qquad\qquad\qquad \texttt{decr \textit{tên};}
\end{flushleft}
Một lần nữa, \texttt{\it tên} biểu diễn mọi tên biến. Lệnh trước làm tăng giá trị của biến
lên một và gán lại vào chính nó. Bởi vậy nếu biến \texttt{Y} đã có giá trị năm trước đó
thì lệnh
\begin{flushleft}
 \qquad\qquad\qquad \texttt{incr Y;}
\end{flushleft}
sau khi được thực hiện sẽ làm giá trị của \texttt{Y} tăng lên thành sáu.

Ngược lại, lệnh \texttt{decr} được dùng để làm giảm giá trị của biến đi một và gán lại cho
chính biến đó. Có một ngoại lệ là khi biến đã được gán bằng không thì lệnh không làm thay
đổi giá trị nữa. Bởi vậy, nếu giá trị của \texttt{Y} là năm thì sau khi lệnh
\begin{flushleft}
\qquad\qquad\qquad  \texttt{decr Y;}
\end{flushleft}
được thực hiện thì giá trị bốn sẽ được gán cho \texttt{Y}. Tuy nhiên, nếu giá trị của
\texttt{Y} đã bằng không thì giá trị này vẫn là không sau khi lệnh được thực hiện.

Bare~Bones chỉ cung cấp một cấu trúc điều khiển biểu diễn bởi cặp \texttt{while-end}. Dãy
lệnh sau
\begin{flushleft}
  \qquad \qquad\qquad \texttt{while \textit{tên} not 0 do;} \\
  \qquad\qquad\qquad\quad \texttt{.} \\
  \qquad\qquad\qquad\quad \texttt{.} \\
  \qquad\qquad\qquad\quad \texttt{.} \\
  \qquad\qquad\qquad\texttt{end;}
\end{flushleft}
(ở đó \texttt{tên} biểu diễn mọi tên biến có thể) sẽ thực hiện lặp lại một lệnh hoặc một
dãy lệnh nằm giữa cặp \texttt{while} và \texttt{end} cho đến khi giá trị của biến
\texttt{\it tên} bằng không. Chính xác hơn, nếu bắt gặp một cấu trúc \texttt{while-end}
khi đang thực hiện chương trình thì đầu tiên giá trị của biến được so sánh với không. Nếu
nó bằng không, cấu trúc này bị bỏ qua, và tiếp tục thực hiện tiếp lệnh ngay sau lệnh
\texttt{end}. Tuy vậy, nếu giá trị của biến khác không, dãy lệnh bên trong cấu trúc
\texttt{while-end} được thực hiện và điều khiển được trả lại vị trí của lệnh
\texttt{while}, và lúc này việc so sánh lại được thực hiện một lần nữa. Chú ý rằng khó
khăn của việc điều khiển lặp được đặt cho người lập trình, anh ta phải xác định rõ ràng
giá trị của biến nào bị thay đổi trong thân vòng lặp để tránh vòng lặp vô hạn. Ví dụ, dãy
lệnh
\begin{flushleft}
  \qquad \qquad\qquad \texttt{incr X;}\\
  \qquad \qquad\qquad \texttt{while \textit{X} not 0 do;} \\
  \qquad \qquad\qquad \quad \texttt{incr Z;} \\
  \qquad\qquad\qquad\texttt{end;}
\end{flushleft}
cho kết quả là một quá trình vô hạn vì khi lệnh \texttt{while} thực hiện, giá trị của
\texttt{X} không bao giờ bằng không, trong khi đó dãy
\begin{flushleft}
  \qquad \qquad\qquad \texttt{incr X;} \\
  \qquad \qquad\qquad \texttt{while \textit{X} not 0 do;} \\
  \qquad \qquad\qquad \quad \texttt{incr Z;} \\
  \qquad \qquad\qquad \quad \texttt{incr X;} \\
  \qquad\qquad\qquad\texttt{end;}
\end{flushleft}
cuối cùng cũng sẽ dừng và đặt giá trị ban đầu của \texttt{X} cho \texttt{Z}.

Để ý rằng các lệnh \texttt{while} và \texttt{end} phải xuất hiện theo cặp và lệnh
\texttt{while} phải xuất hiện trước. Tuy vậy, một lệnh \texttt{while-end} có thể xuất hiện
bên trong các lệnh lặp bằng cách đặt nó trong một cặp \texttt{while-end} khác. Trong các
trường hợp như thế này cặp \texttt{while} và \texttt{end} được hoàn thành bằng cách xem
xét toàn bộ chương trình để gắn mỗi lệnh \texttt{end} với lệnh \texttt{while} gần nó nhất
mà chưa được ghép cặp. Mặc dù không cần thiết về mặt cú pháp nhưng chúng ta thường sắp xếp
thụt vào để dễ đọc đối với các cấu trúc kiểu này.


Một ví dụ để kết thúc mục này là dãy lệnh trong Hình~\ref{fig:fig114}. Dãy lệnh này thực
hiện tính tích của các giá trị trong \texttt{X} và \texttt{Y} sau đó gán vào \texttt{Z},
mặt khác nó cũng ảnh hưởng đến giá trị ban đầu của \texttt{X}. (Cấu trúc
\texttt{while-end} được điều khiển bởi biến \texttt{W} nhằm lưu trữ giá trị gốc cho biến
\texttt{Y}.)

\begin{figure}[tbh]
  \begin{flushleft}
    \qquad \qquad\qquad  \texttt{clear Z;}\\
    \qquad \qquad\qquad \texttt{while X not 0 do;} \\
    \qquad \qquad\qquad \qquad \texttt{clear W;} \\
    \qquad \qquad\qquad \qquad \texttt{while Y not 0 do;} \\
    \qquad \qquad\qquad \qquad \qquad \texttt{incr Z;} \\
    \qquad \qquad\qquad \qquad \qquad \texttt{incr W;} \\
    \qquad \qquad\qquad \qquad \qquad \texttt{decr Y;} \\
    \qquad \qquad\qquad \qquad \texttt{end;} \\
    \qquad \qquad\qquad \qquad \texttt{while W not 0 do;} \\
    \qquad \qquad\qquad \qquad \qquad \texttt{incr Y;}\\
    \qquad \qquad\qquad \qquad \qquad \texttt{decr W;}\\
    \qquad \qquad\qquad \qquad \texttt{end;} \\
    \qquad \qquad\qquad \qquad \texttt{decr X;} \\
    \qquad\qquad\qquad\texttt{end;}
\end{flushleft}
  \caption{Một chương trình Bare Bone để tính \texttt{X*Y}}
  \label{fig:fig114}
\end{figure}

\subsection*{Lập trình Bare Bone}
Ta nhắc lại rằng mục đích của việc trình bày Bare~Bones là để thảo luận về khả năng về mặt
lý thuyết và thực tế của máy. Bare~Bones có thể rất khó khăn trong ứng dụng thực tế. Tuy
nhiên, như ta sẽ thấy ngay trong mục sau, nó lại cho ta mọi khả năng của một ngôn ngữ lập
trình phổ dụng. Bây giờ, ta sẽ xem xét xem làm thế nào mà Bare~Bones có thể được dùng để
biểu diễn một vài thao tác cơ bản.

Đầu tiên ta để ý rằng bằng cách tổ hợp các lệnh gán, ta có thể gán cho một biến một giá
trị tuỳ ý ( mọi số nguyên không âm). Ví dụ, dãy lệnh sau đây gán giá trị ba cho
biến~\texttt{X} bằng cách đầu tiên gán nó với giá trị không và sau đó tăng giá trị này lên
ba lần:
\begin{flushleft}
\qquad\qquad\qquad  \texttt{clear X;} \\
\qquad\qquad\qquad  \texttt{incr X;} \\
\qquad\qquad\qquad  \texttt{incr X;} \\
\qquad\qquad\qquad  \texttt{incr X;}
\end{flushleft}

Một thao tác phổ biến khác là sao chép dữ liệu từ một vị trí tới vị trí khác. Nói theo
cách của ngôn ngữ Bare~Bones, ta cần gán giá trị một biến bằng một biến khác. Điều này có
thể thực hiện bằng cách xoá biến cần gán và sau đó lặp lại việc tăng nó lên một số lần
thích hợp. Thực ra ta thấy rằng dãy lệnh
\begin{flushleft}
\qquad\qquad\qquad  \texttt{clear Z;} \\
\qquad\qquad\qquad  \texttt{while  X not 0 do;} \\
\qquad\qquad\qquad  \qquad \texttt{incr Z;} \\
\qquad\qquad\qquad  \qquad \texttt{decr X;} \\
\qquad\qquad\qquad  \texttt{incr X;}
\end{flushleft}
chuyển giá trị được gắn với \texttt{X} cho \texttt{Z}. Tuy vậy, dãy lệnh này lại phá huỷ
giá trị của \texttt{X}. Để tránh điều này, ta dùng một biến tạm để chuyển dữ liệu ban đầu
vào biến này. Sau đó ta dùng biến tạm này làm dữ liệu nguồn để lưu trữ lại giá trị của biến
gốc, đồng thời gán giá trị có biến cần gán bằng biến gốc. Theo cách này, việc chuyển dữ
liệu từ biến \texttt{Today} thành \texttt{Yesterday} có thể được thực hiện dùng dãy lệnh
như trong Hình~\ref{fig:fig115}.


Ta chọn cú pháp
\begin{flushleft}
\qquad\qquad\qquad  \texttt{copy \textit{tên1} to \textit{tên2};}
\end{flushleft}
(với \texttt{tên1} và \texttt{tên2} biển diễn các tên biến) như kí hiệu viết tắt cho một
cấu trúc lệnh có dạng như Hình~\ref{fig:fig115}. Bởi vậy, mặc dù Bare~Bones bản thân nó
không có lệnh \texttt{copy}, ta thường viết chương trình có lệnh này, và ta ngầm hiểu rằng
ta đã chuyển một cách không hình thức thành các chương trình Bare~Bones thực sự bằng cách
thay thế lệnh \texttt{copy} bằng các cấu trúc \texttt{while-end} thích hợp (với các biến
tạm có tên không xung đột với tên đã được sử dụng ở nơi khác trong chương trình.)

\begin{figure}[tbh]
  \begin{flushleft}
    \qquad \qquad\qquad  \texttt{clear Aux;}\\
    \qquad \qquad\qquad  \texttt{clear Tomorrow;}\\
    \qquad \qquad\qquad \texttt{while Today not 0 do;} \\
    \qquad \qquad\qquad \qquad \texttt{incr  Aux;} \\
    \qquad \qquad\qquad \qquad \texttt{decr Today;} \\
    \qquad \qquad\qquad \texttt{end;} \\

    \qquad\qquad \qquad \texttt{while Aux not 0 do;} \\
    \qquad\qquad \qquad \qquad \texttt{incr Today;} \\
    \qquad\qquad \qquad \qquad \texttt{incr Tomorrow;} \\
    \qquad\qquad \qquad \qquad \texttt{decr Aux;} \\
    \qquad\qquad \qquad \texttt{end;}
  \end{flushleft}
  \caption{Cài đặt hàm `` sao chép \texttt{Today} thành
    \texttt{Tomorrow}''}
  \label{fig:fig115}  
\end{figure}  

\subsection*{Tính phổ dụng của Bare Bones}
Bây giờ ta sẽ áp dụng luận đề Church-Turing để khẳng định rằng Bare~Bones là một ngôn ngữ
lập trình phổ dụng. Đầu tiên ta nhận thấy rằng mọi chương trình được viết trong Bare~Bones
có thể được nhìn như một việc tính toán hàm. Các hàm với đầu vào là các giá trị được gán
cho các biến trước khi thực hiện chương trình, và đầu ra của hàm là giá trị của các biến
khi chương trình kết thúc. Để tính hàm, ta đơn thuần chỉ chạy chương trình, bắt đầu
với việc khởi tạo giá trị cho các biến, và quan sát giá trị của các biến khi chương trình
kết thúc.

Dưới các điều kiện này, chương trình
\begin{flushleft}
\qquad\qquad\qquad  \texttt{incr X;}
\end{flushleft}
tính toán cùng một hàm (hàm tính số liền sau) được tính bởi máy Turing trong ví dụ trong
Mục~\ref{sec:112}. Thật vậy, nó tăng giá trị gắn với biến \texttt{X} lên một. Tương tự,
nếu chúng ta diễn giải các biến \texttt{X} và \texttt{Y} như các đầu vào và biến
\texttt{Z} như đầu ra thì chương trình
\begin{flushleft}
  \qquad \qquad\qquad  \texttt{copy Y to Z;}\\
  \qquad \qquad\qquad \texttt{while X not 0 do;} \\
  \qquad \qquad\qquad \qquad \texttt{incr  Z;} \\
  \qquad \qquad\qquad \qquad \texttt{decr X;} \\
  \qquad \qquad\qquad \texttt{end;} \\
\end{flushleft}
trực tiếp tính toán hàm cộng.

Các nhà nghiên cứu đã chứng minh rằng ngôn ngữ lập trình Bare~Bones có thể được sử dụng để
biểu diễn các thuật toán để tính mọi hàm tính được bởi máy Turing. Kết hợp với luận đề
Church-Turing nó chỉ ra rằng mọi hàm đều có thể được tính toán bởi một chương trình viết
bằng Bare~Bones. Bởi vậy Bare~Bones là một ngôn ngữ lập trình phổ dụng theo nghĩa rằng,
nếu có một thuật toán để giải quyết bài toán, thì bài toán đó có thể được giải bởi một
chương trình Bare~Bones nào đó. Nói tóm lại, Bare~Bones về mặt lý thuyết có thể dùng như
một ngôn ngữ lập trình đa năng.

Ta nói \textit{về mặt lý thuyết} vì một ngôn ngữ kiểu như thế này chắc chắn không tiện lợi
như các ngôn ngữ bậc cao đã được giới thiệu trong Chương~\ref{}. Tuy vậy, mỗi ngôn ngữ về
cơ bản đều chứa các đặc điểm của Bare~Bones như phần cơ bản nhất. Và phần cơ bản này đảm
bảo tính phổ dụng cho các ngôn ngữ bậc cao này; còn các đặc điểm thêm vào của ngôn ngữ chỉ
giúp làm cho nó thêm tiện lợi.

Mặc dù không thực tế trong môi trường lập trình ứng dụng, các ngôn ngữ kiểu Bare~Bones lại
hay được dùng trong nghiên cứu lý thuyết khoa học máy tính. Ví dụ, trong phụ lục~\ref{} ta
dùng Bare~Bones như một công cụ để giải quyết câu hỏi về tính tương đương của cấu trúc lặp
và cấu trúc đệ quy nảy sinh trong Chương~\ref{}. Nhờ công cụ này, nghi ngờ của ta về tính
tương đương của hai cấu trúc này đã được chứng minh.

\subsection*{Câu hỏi \& Bài tập}
\begin{enumerate}
\item Chỉ ra rằng lệnh \texttt{invert X;} (để chuyển các giá trị của \texttt{X} thành
  không nếu nó được khởi tạo khác không và thành một nếu nếu giá trị khởi tạo bằng không)
  có thể được mô phỏng bởi một đoạn chương trình Bare~Bones.

\item Chỉ ra rằng ngôn ngữ đơn giản Bare~Bones cũng chưa phải là tối thiểu bằng cách chứng
  minh rằng lệnh \texttt{clear} có thể được thay thế bằng tổ hợp của các lệnh khác trong
  ngôn ngữ.

\item Chỉ ra rằng cấu trúc \texttt{if-then-else} có thể mô phỏng dùng Bare~Bones. Có nghĩa
  rằng, bạn hãy viết một dãy lệnh trong Bare~Bones mô phỏng lệnh
  \begin{flushleft}
    \qquad \qquad  \texttt{if X not 0 then S1 else S2;}
  \end{flushleft}
  với \texttt{S1} và \texttt{S2} là một dãy lệnh bất kỳ.

\item Chỉ ra rằng mỗi lệnh Bare~Bones có thể biểu diễn theo ngôn ngữ máy ở Phụ
  lục~\ref{phuluc1} (Bởi vậy Bare~Bones có thể được sử dụng như một ngôn ngữ lập trình cho
  một máy kiểu này.)

\item Làm thế nào để có thể tính toán các số nguyên âm với Bare~Bones?

\item Mô tả hàm được tính bởi chương trình Bare~Bones sau đây, giả sử rằng đầu vào của hàm
  được biểu diễn bởi \texttt{X} và đầu ra bởi \texttt{Z}:
  \begin{flushleft}
    \qquad \qquad  \texttt{clear Z;}\\
    \qquad \qquad \texttt{while X not 0 do;} \\
    \qquad \qquad \qquad \texttt{incr  Z;} \\
    \qquad \qquad \qquad \texttt{incr  Z;} \\
    \qquad \qquad \qquad \texttt{decr  X;} \\
    \qquad \qquad \texttt{end;} 
  \end{flushleft}
\end{enumerate}

 
%%% Local Variables: 
%%% mode: latex
%%% TeX-master: "../tindaicuong"
%%% End: 
