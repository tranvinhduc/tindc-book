
\chapter{Giới thiệu chung} 
\minitoc
\vspace{0.5cm}
\noindent
% Trong chương này, ta xem xét phạm vi của khoa học máy tính, lịch sử phát triển và các kiến
% thức nền tảng chuẩn bị cho các chương tiếp theo.
Khoa học máy tính là một chuyên ngành nhằm xây dựng cơ sở khoa học cho những lĩnh vực như
thiết kế máy tính, lập trình, xử lý thông tin, thuật toán, giải quyết vấn đề dựa
trên thuật toán, và cả quá trình xây dựng thuật toán. Nó cung cấp nền tảng cho các ứng
dụng tin học hiện tại cũng như tương lai.

%Cuốn sách này nhằm giới thiệu một cách tổng quan về khoa học máy tính. Tuy nhiên, ta cũng muốn đánh
%giá đầy đủ phạm vi và tính động của ngành học này; vì vậy, ngoài các chủ đề nghiên cứu
%chính, ta còn xem xét đến lịch sử phát triển, hiện trạng, và những hướng nghiên cứu trong tương lai. Mục tiêu của cuốn sách là giúp người đọc hiểu về các chức
%năng của khoa học máy tính--những kiến thức này không những có ích cho những người làm
%nghiên cứu mà còn cho cả những người làm trong lĩnh vực ứng dụng công nghệ.
 
\input{ch0/role.tex}
\input{ch0/origins.tex}
\section{Khoa học thuật toán}
Thuở ban đầu, do khả năng lưu trữ thông tin bị giới hạn và các thủ tục lập trình dài dòng,
rắc rối nên các thuật toán được sử dụng trong các máy tính toán còn rất đơn giản. Theo
thời gian, các rào cản này dần bị phá bỏ; các máy đã được ứng dụng cho các nhiệm vụ lớn
hơn và phức tạp hơn. Việc biểu diễn các nhiệm vụ phức tạp này dưới dạng thuật toán đã bắt
đầu thách thức khả năng của con người. Chính vì vậy, ngày càng có nhiều nghiên cứu hướng
về thuật toán và lập trình.

Trong bối cảnh này, các kết quả lý thuyết của các nhà toán học mới bắt
đầu mới thể hiện tầm quan trọng của nó. Nhờ định lý về tính không toàn
vẹn của G\"odel, các nhà toán học đã bắt đầu nghiên cứu các vấn đề
liên quan đến thuật toán nảy sinh từ công nghệ cao. Từ đây, nổi lên
một nghành khoa học mới, gọi là \textit{khoa học máy tính}.

Ngày nay, khoa học máy tính cũng được coi như là như là khoa học thuật
toán. Phạm vi của ngành khoa học này rất rộng, liên quan đến cả toán
học, kỹ thuật, tâm lý học, sinh học, quản trị kinh doanh, và ngôn ngữ
học. Trong các chương tiếp theo, ta sẽ thảo luận nhiều vấn đề
khác nhau của ngành khoa học này. Với mỗi vấn đề, mục tiêu của chúng
ta bao gồm: giới thiệu các ý tưởng trung tâm, các vấn đề đang được
nghiên cứu, và một vài phương pháp đang được sử dụng để phát triển tri
thức trong lĩnh vực này.

Để có được một bức tranh tổng thể về khoa học máy tính, ta xem
xét những câu hỏi sau đây:

\begin{figure}[tb] 
\centering
    \scalebox{0.4}{\includegraphics{ch0/fig05.pdf}}
\caption{Vai trò trung tâm của thuật toán trong khoa học máy tính}
  \label{fig:fig0.5}
\end{figure}

\begin{itemize}
\item Những bài toán nào có thể được giải bằng thuật toán?

\item Làm sao để có thể phát hiện ra thuật toán một cách dễ dàng hơn?

\item Làm sao để biểu diễn và truyền tải thuật toán một cách tốt
  hơn?

\item Làm sao để các kiến thức về thuật toán giúp ta tạo ra
  các máy tốt hơn?

\item Làm sao để có thể phân tích và so sánh các đặc trưng 
  của các thuật toán khác nhau?
\end{itemize}

Chú ý rằng các câu hỏi này đều hướng đến một chủ đề chung là nghiên
cứu thuật toán~(Hình~\ref{fig:fig0.5})




%%% Local Variables: 
%%% mode: latex
%%% TeX-master: "../tindaicuong"
%%% End: 

\section{Khái niệm trừu tượng}
 
Bởi vì khái niệm trừu tượng được sử dụng rất nhiều trong khoa học máy tính và thiết kế hệ
thống máy tính nên ta cần phải hiểu khái niệm đó ngay từ trong chương mở đầu này. Thuật
ngữ \textbf{trừu tượng}~(abstraction) ở đây nhằm chỉ sự phân biệt giữa tính chất bên ngoài
của một đối tượng và chi tiết hình thành bên trong của đối tượng đó. Nhờ trừu tượng mà ta
có thể bỏ qua chi tiết bên trong của các thiết bị phức tạp như máy tính, ô-tô, hoặc lò
vi-ba sóng; thay vào đó chỉ cần nắm bắt được các giao diện chức năng để sử dụng.  Hơn nữa,
nhờ công cụ trừu tượng nên người ta mới xây dựng được những hệ thống phức tạp.  Máy tính,
ôtô, và các lò vi-ba sóng được tạo nên từ các thành phần, mỗi thành phần lại được xây dựng
từ các thành phần nhỏ hơn. Mỗi thành phần biểu diễn một mức trừu tượng. Ở mỗi mức trừu
tượng, ta sử dụng lại các thành phần mức thấp hơn mà không phải quan tâm chi tiết xem
chúng hình thành thế nào.

Nhờ có trừu tượng, ta có thể xây dựng, phân tích, và quản lý các hệ thống máy tính lớn,
phức tạp. Các hệ thống này rất rắc rối nếu nhìn các đối tượng của chúng một cách chi
tiết. Theo từng mức trừu tượng, ta chỉ nhìn hệ thống dưới dạng các thành phần, gọi là các
\textbf{công cụ trừu tượng} (abstract tools), mà bỏ qua chi tiết về cấu thành bên trong
của chúng. Điều này cho phép ta tập trung xem xét xem các thành phần (ở cùng mức) tương
tác với nhau như thế nào và làm thế nào để tập hợp chúng lại thành một thành phần ở mức
cao hơn. Bởi vậy, ta có thể hiểu từng phần của hệ thống.

Ta nhấn mạnh rằng việc trừu tượng không chỉ giới hạn trong phạm vi khoa học và công
nghệ. Nó là kỹ thuật đơn giản hoá quan trọng tạo ra lối sống của xã hội. Rất ít người
trong chúng ta hiểu sự tiện lợi trong cuộc hàng ngày được thực hiện như thế nào-- ta ăn và
mặc những thứ mà bản thân ta không thể làm ra được, ta sử dụng các thiết bị điện mà không
cần phải hiểu về công nghệ chế tạo ra nó. Ta sử dụng các dịch vụ của những người khác mà
không cần biết một cách chi tiết về nghề của họ. Từng bước, người ta cố gắng chuyên biệt
cách thực hiện một số thứ trong xã hội, còn những người khác chỉ cần cố gắng tìm hiểu để
sử dụng các kết quả đó như là công cụ trừu tượng. Theo cách này, kho các công cụ trừu
tượng của xã hội được mở rộng, và khả năng của xã hội không ngừng tiến xa hơn.

Ta còn gặp lại chủ đề trừu tượng nhiều lần trong nghiên cứu. Ta sẽ thấy rằng các thiết bị
máy tính được xây dựng ở mức của các công cụ trừu tượng.  Sự phát triển của các hệ thống
phần mềm lớn là theo cách mô đun hoá trong đó mỗi mô đun được sử dụng như là một công cụ
trừu tượng trong mô đun lớn hơn. Hơn nữa, trừu tượng đóng vai trò quan trọng trong nhiệm
vụ phát triển ngành khoa học máy tính, cho phép các các nhà nghiên cứu hướng sự chú ý tới
các lĩnh vực đặc biệt bên trong một lĩnh vực phức tạp. Trên thực tế, cách tổ chức của cuốn
sách này phản ánh đặc trưng của khoa học. Mỗi chương trọng tâm vào một chủ đề cụ thể,
chương này độc lập với chương khác, nhưng khi ghép lại cùng nhau nó trở thành một tài liệu
tổng quan dễ hiểu về các lĩnh vực rộng được nghiên cứu.

%%% Local Variables: 
%%% mode: latex
%%% TeX-master: "../tindaicuong"
%%% End: 

  
%%% Local Variables: 
%%% mode: latex
%%% TeX-master: "../tindaicuong"
%%% End: 
