\section{Khái niệm trừu tượng}
 
Bởi vì khái niệm trừu tượng được sử dụng rất nhiều trong khoa học máy tính và thiết kế hệ
thống máy tính nên ta cần phải hiểu khái niệm đó ngay từ trong chương mở đầu này. Thuật
ngữ \textbf{trừu tượng}~(abstraction) ở đây nhằm chỉ sự phân biệt giữa tính chất bên ngoài
của một đối tượng và chi tiết hình thành bên trong của đối tượng đó. Nhờ trừu tượng mà ta
có thể bỏ qua chi tiết bên trong của các thiết bị phức tạp như máy tính, ô-tô, hoặc lò
vi-ba sóng; thay vào đó chỉ cần nắm bắt được các giao diện chức năng để sử dụng.  Hơn nữa,
nhờ công cụ trừu tượng nên người ta mới xây dựng được những hệ thống phức tạp.  Máy tính,
ôtô, và các lò vi-ba sóng được tạo nên từ các thành phần, mỗi thành phần lại được xây dựng
từ các thành phần nhỏ hơn. Mỗi thành phần biểu diễn một mức trừu tượng. Ở mỗi mức trừu
tượng, ta sử dụng lại các thành phần mức thấp hơn mà không phải quan tâm chi tiết xem
chúng hình thành thế nào.

Nhờ có trừu tượng, ta có thể xây dựng, phân tích, và quản lý các hệ thống máy tính lớn,
phức tạp. Các hệ thống này rất rắc rối nếu nhìn các đối tượng của chúng một cách chi
tiết. Theo từng mức trừu tượng, ta chỉ nhìn hệ thống dưới dạng các thành phần, gọi là các
\textbf{công cụ trừu tượng} (abstract tools), mà bỏ qua chi tiết về cấu thành bên trong
của chúng. Điều này cho phép ta tập trung xem xét xem các thành phần (ở cùng mức) tương
tác với nhau như thế nào và làm thế nào để tập hợp chúng lại thành một thành phần ở mức
cao hơn. Bởi vậy, ta có thể hiểu từng phần của hệ thống.

Ta nhấn mạnh rằng việc trừu tượng không chỉ giới hạn trong phạm vi khoa học và công
nghệ. Nó là kỹ thuật đơn giản hoá quan trọng tạo ra lối sống của xã hội. Rất ít người
trong chúng ta hiểu sự tiện lợi trong cuộc hàng ngày được thực hiện như thế nào-- ta ăn và
mặc những thứ mà bản thân ta không thể làm ra được, ta sử dụng các thiết bị điện mà không
cần phải hiểu về công nghệ chế tạo ra nó. Ta sử dụng các dịch vụ của những người khác mà
không cần biết một cách chi tiết về nghề của họ. Từng bước, người ta cố gắng chuyên biệt
cách thực hiện một số thứ trong xã hội, còn những người khác chỉ cần cố gắng tìm hiểu để
sử dụng các kết quả đó như là công cụ trừu tượng. Theo cách này, kho các công cụ trừu
tượng của xã hội được mở rộng, và khả năng của xã hội không ngừng tiến xa hơn.

Ta còn gặp lại chủ đề trừu tượng nhiều lần trong nghiên cứu. Ta sẽ thấy rằng các thiết bị
máy tính được xây dựng ở mức của các công cụ trừu tượng.  Sự phát triển của các hệ thống
phần mềm lớn là theo cách mô đun hoá trong đó mỗi mô đun được sử dụng như là một công cụ
trừu tượng trong mô đun lớn hơn. Hơn nữa, trừu tượng đóng vai trò quan trọng trong nhiệm
vụ phát triển ngành khoa học máy tính, cho phép các các nhà nghiên cứu hướng sự chú ý tới
các lĩnh vực đặc biệt bên trong một lĩnh vực phức tạp. Trên thực tế, cách tổ chức của cuốn
sách này phản ánh đặc trưng của khoa học. Mỗi chương trọng tâm vào một chủ đề cụ thể,
chương này độc lập với chương khác, nhưng khi ghép lại cùng nhau nó trở thành một tài liệu
tổng quan dễ hiểu về các lĩnh vực rộng được nghiên cứu.

%%% Local Variables: 
%%% mode: latex
%%% TeX-master: "../tindaicuong"
%%% End: 
