\chapter{Mạng và mạng Internet}

\minitoc
\vspace{0.5cm}
\noindent
Trong chương này, ta sẽ thảo luận xung quanh khái niệm mạng; bao gồm việc nghiên cứu xem các máy tính kết nối với nhau, chia sẻ thông tin và
tài nguyên như thế nào. Ta cũng xem xét các  cấu trúc và điều khiển của hệ thống mạng, các ứng dụng mạng và những vấn đề liên quan đến an ninh
mạng. Chủ đề nổi bật trong chương này là mạng Internet, hệ thống kết nối các hệ thống mạng trên phạm vi toàn cầu.

%Nhu cầu chia sẻ thông tin và tài nguyên giữa các máy tính khác nhau dẫn tới các hệ thống
%máy tính được kết nối, gọi là các \textbf{mạng máy tính}, trong đó các máy tính được kết
%nối để dữ liệu có thể được truyền từ máy này sang máy khác. Trong các mạng máy tính này,
%người sử dụng có thể trao đổi các thông điệp và tài nguyên chia sẻ nằm rải rác trong toàn bộ hệ thống--như máy in, các gói
%phần mềm, và các phương tiện lưu trữ dữ liệu. Những
%phần mềm cơ bản đòi hỏi hỗ trợ như các ứng dụng từ những gói công cụ đơn giản cho đến một
%hệ thống mở rộng của phần mềm mạng mà cung cấp một cơ sở hạ tầng rộng lớn cho hệ thống
%mạng phức tạp. Theo một nghĩa nào đó, phần mềm mạng đang tiến hóa thành một hệ điều hành mạng lớn. %Trong
%chương này ta sẽ nghiên cứu lĩnh vực mở rộng này của khoa học máy tính.
%-----------------

 
   
\section{Cơ bản về mạng}
\label{sec:4.1}
Ta bắt đầu với các khái niệm cơ bản về mạng.

\subsection*{Phân loại hệ thống mạng}

Một mạng máy tính thường được phân loại dựa trên đặc tính về khoảng cách địa lý như
\textbf{mạng cục bộ (LAN)}, \textbf{mạng đô thị (MAN)} hay \textbf{mạng diện rộng
  (WAN)}. Mạng LAN thường bao gồm một tập hợp các máy tính trong một tòa nhà đơn lẻ hay
một liên hợp các tòa nhà. Ví dụ, các máy tính trong một khuôn viên trường đại học hay
trong một nhà máy chế tạo máy móc có thể được kết nối với nhau thông qua mạng LAN.  Mạng
MAN là mạng có phạm vi trung bình, ví dụ như mạng mở rộng trong phạm vi một địa phương cục
bộ.  Mạng WAN nối kết các thiết bị mạng có tầm khoảng cách lớn hơn, ví dụ như khoảng cách
giữa các thành phố hay giữa các vị trí trên thế giới.

Một cách thức phân loại mạng khác là chia thành mạng thành hai loại: \textbf{mạng mở} và
\textbf{mạng đóng}. Mạng mở là mạng mà sự vận hành bên trong của nó được dựa trên những
thiết kế trong một phạm vi công cộng. Mạng đóng là mạng được thiết kế của nó dựa trên
chính sự đổi mới của nó và được điều khiển bởi một cá nhân hay một tổ chức nào đó.

Một ví dụ về mạng mở là mạng Internet (mạng phạm vi toàn cầu). Trong mạng Internet, việc truyền thông tin được điều khiển bởi một tập các quy tắc chuẩn mở được thừa nhận rộng rãi như bộ giao thức TCP/IP. Mọi người đều có thể sử dụng những quy tắc chuẩn này mà
không phải trả phí cũng như không phải ký kết một thỏa thuận nào về quyền sử dụng. Ngược lại, các công ty như Novell, thường phát triển  hệ thống và duy trì quyền sở hữu của họ. Điều này cho phép công ty có thể thu được lợi nhuận từ việc bán
hay cho thuê các sản phẩm này. Những mạng dựa trên những hệ thống như vậy là ví dụ
về các mạng đóng.

\begin{figure}[tbh] 
\centering
    \scalebox{0.45}{\includegraphics{ch5/fig41.pdf}}
\caption{Các hình trạng mạng}
  \label{fig:fig4.1}
\end{figure}

Còn một cách phân loại mạng khác đó là dựa trên hình trạng của mạng. Hình trạng mạng
là mô hình xác định cách để các thiết bị trong mạng kết nối với nhau. Hình~\ref{fig:fig4.1} giới
thiệu ba hình trạng mạng phổ biến:
\begin{inparaenum}[(1)]
\item Mạng vòng tròn (Ring), trong đó các thiết bị được kết nối với nhau
  theo hình tròn;
\item Mạng hình tuyến (Bus), trong đó tất cả các thiết bị được kết nối với nhau thông qua
  một đường truyền tải gọi là trục chính (Bus); và
\item Mạng hình sao, trong đó một thiết bị phục vụ như là một điểm trung tâm nơi mà các
  thiết bị khác được kết nối tới nó.
\end{inparaenum}

Trong những mô hình trên, mạng hình sao được sử dụng phổ biến nhất. Mạng này đã
phát triển từ mô hình trung tâm máy tính lớn phục vụ cho nhiều người sử dụng. Thể hiện rõ nhất là trường hợp những người dùng sử dụng các thiết bị đầu cuối kết nối vào các máy tính nhỏ. Ngày nay, mô hình mạng hình tuyến (bus) cũng
được sử dụng rộng rãi dưới hình thức mạng chuẩn là \textbf{Ethernet}, một
trong những mô hình mạng khá phổ biến.

Cần chú ý rằng hình trạng của mạng có thể không  thể hiện rõ qua mô hình vật
lý của nó. Ví dụ, một mạng hình tuyến (bus) không nhất thiết phải được triển khai
dưới một đường trục dài nơi mà các máy tính được kết nối thông qua các liên kết ngắn như
đã mô tả trong Hình~\ref{fig:fig4.1}. Thay vào đó, thông thường người ta dựng một mạng
hình tuyến bằng cách chạy các liên kết từ chính mỗi máy tính tới một vùng trung tâm, nơi
mà chúng được kết nối với nhau thông qua một thiết bị gọi là \textbf{hub}. Thiết bị hub
này nhỏ hơn bất kỳ một trục ngắn nào. Nó thực hiện tiếp sóng bất kỳ tín hiệu nào nó nhận
được (thông qua một vài bộ khuếch đại tín hiệu) và truyền tới tất cả các máy tính kết nối
với nó thông qua các cổng. Kết quả là một mạng trông như mạng hình sao lại được hoạt động
dưới hình thức là một mạng hình tuyến. Sự khác nhau ở đây là thiết bị trung tâm trong mô
hình mạng hình sao là một máy tính (thường là một thiết bị với nhiều khả năng hơn tại các
điểm nút của hình sao); nó  nhận và xử lý các thông điệp từ các máy tính khác. Ngược lại, thiết
bị trung tâm trong mô hình mạng hình tuyến là một hub; nó  chỉ đơn thuần cung cấp một kênh truyền
thông tin cho các máy tính.

Một quan điểm khác cũng cần phải chú ý là các kết nối giữa các thiết bị trong hệ thống
mạng không nhất thiết phải là các thiết bị vật lý. Hệ thống mạng không dây, sử dụng công
nghệ truyền phát qua sóng radio, cũng dần được sử dụng phổ biến. Đặc biệt, thiết bị hub
trong rất nhiều mô hình mạng hình tuyến ngày nay thực chất là một trạm tiếp và phát sóng
radio.

\subsection*{Các giao thức}
Để cho một hệ thống mạng hoạt động tin cậy, việc thiết lập những quy tắc nhờ
đó các hoạt động của mạng được kiểm soát là rất quan trọng. Những quy tắc này được gọi là
các \textbf{giao thức}. Thông qua việc phát triển và kế thừa các chuẩn giao thức, các nhà
cung cấp có thể xây dựng những sản phẩm cho các ứng dụng mạng tương thích với những sản
phẩm của nhà cung cấp khác. Việc phát triển các chuẩn giao thức là quá trình không
thể thiếu trong sự phát triển các công nghệ mạng.


Khi giới thiệu về khái niệm giao thức, ta hãy xem xét vấn đề  phối hợp truyền thông điệp giữa các máy tính trong một mạng. Nếu không có các quy tắc  quản lý quá
trình truyền  này, các máy tính có thể yêu cầu truyền thông điệp vào cùng một
thời điểm hoặc cũng có thể bị lỗi khi tiếp nhận các thông điệp khi hỗ trợ đó được yêu cầu.


\begin{figure}[tbh] 
\centering
    \scalebox{0.45}{\includegraphics{ch5/fig42.pdf}}
    \caption{Truyền thông trong mạng vòng tròn}
  \label{fig:fig4.2}
\end{figure}

Một cách tiếp cận để có thể giải quyết được vấn đề này là giao thức \textbf{thẻ bài} trong
mạng vòng, được phát triển bởi IBM từ  những năm 1970 và tiếp tục trở nên phổ biến
trong các hệ thống mạng dựa trên nền tảng mô hình mạng vòng. Với giao thức này, tất
cả các thiết bị trong mạng truyền tải thông điệp theo một hướng chung duy nhất
(Hình~\ref{fig:fig4.2}), nghĩa là tất cả các thông điệp đó được gửi qua mạng di chuyển
vòng tròn theo cùng một hướng bằng cách chuyển tiếp từ máy tính này tới máy tính khác. Khi
một thông điệp được truyền tới đích của nó, máy tính đích giữ lại một bản sao và chuyển
tiếp một bản sao của thông điệp đó sang máy tính tiếp theo theo hình tròn. Khi bản sao
được chuyển tiếp về tới máy tính ban đầu, máy tính đó nhận thấy rằng thông điệp này đã
truyền được tới đích cần thiết, nó sẽ loại bỏ thông điệp ra khỏi vòng tròn. Tất nhiên, hệ
thống này còn phụ thuộc vào sự hợp tác của các liên máy tính. Nếu một máy tính yêu cầu
truyền phát liên tục các thông điệp từ chính nó thay vì chuyển tiếp các thông điệp sang
máy tính khác thì không có quá trình truyền tải nào được hoàn thành.


Để giải quyết vấn đề này, một xâu bít duy nhất, gọi là thẻ bài, được sử dụng trong quá
trình truyền tải theo vòng tròn. Quyền sở hữu của thẻ bài này được trao cho máy tính
nguồn, nơi bắt đầu truyền phát thông điệp; nếu không có thẻ bài, một máy tính chỉ được
phép chuyển tiếp thông điệp mà nó nhận được. Thông thường, mỗi máy tính đơn thuần chỉ tiếp
nhận và chuyển tiếp thẻ bài theo đúng cách mà nó chuyển tiếp thông điệp. Tuy nhiên, nếu
máy tính nhận được thẻ bài có thông điệp của chính nó đưa lên mạng, nó sẽ truyền một thông
điệp trong khi giữ lại thẻ bài. Khi thông điệp này hoàn tất quá trình truyền tải qua một
vòng tròn, máy tính chuyển tiếp thẻ bài cho máy tính tiếp theo trong vòng tròn. Như vậy,
khi máy tính tiếp theo nhận được thẻ bài, nó có thể chuyển tiếp thẻ bài ngay lập tức hoặc
truyền phát một thông điệp mới của mình trước khi chuyển tiếp thẻ bài cho máy tính tiếp
theo. Theo cách thức này, mỗi máy tính trong mạng đều có cơ hội như nhau để có thể gửi
thông điệp của nó cũng như thẻ bài đi xung quanh vòng tròn.


\begin{figure}[tbh]
\centering
    \scalebox{0.4}{\includegraphics{ch5/fig43.pdf}}
    \caption{Truyền thông trong mạng hình tuyến}
  \label{fig:fig4.3}
\end{figure}

Một giao thức khác cũng được sử dụng trong công nghệ mạng hình tuyến cho việc phối hợp
truyền tải các thông điệp, đó là giao thức dựa trên tập các giao thức Ethernet. Trong hệ
thống mạng Ethernet, quyền truyền tải thông điệp được điều khiển bởi giao thức \textbf{Đa
  truy nhập có thăm dò và tách đụng độ} (Carrier Sense, Multiple Access with Collision
Detection--CSMA/CD). Giao thức này yêu cầu mỗi thông điệp phải được quảng bá cho tất cả
các máy tính trong trục chính (Hình~\ref{fig:fig4.3}). Mỗi máy tính sẽ theo dõi tất cả các
thông điệp những chỉ giữ lại những thông điệp nào được gửi tới chính nó thông qua địa
chỉ. Để truyền phát một thông điệp, một máy tính đợi cho đến khi đường trục chính rỗi, và
tại thời điểm này nó sẽ bắt đầu truyền phát tín hiệu trong khi tiếp tục theo dõi đường
trục chính. Nếu một máy tính khác cũng bắt đầu truyền tín hiệu, cả hai máy tính sẽ phát
hiện ra sự xung đột và sẽ tạm dừng trong một khoảng thời gian ngẫu nhiên trước khi truyền
tín hiệu lại. Một nhóm nhỏ người có nhu cầu đàm luận với nhau có thể sử dụng một hệ thống
tương đương như vậy. Nếu hai người cùng bắt đầu nói tại một thời điểm, cả hai sẽ dừng
lại. Một trong hai người có thể sẽ tiếp tục với một chuỗi câu như ``Xin lỗi, anh định nói
gì vậy?'', ``Không, không. Anh nói trước đi'', trong khi với giao thức CSMA/CD, mỗi máy
tính đơn thuần chỉ là thử truyền phát tín hiệu lại.


\subsection*{Kết hợp các hệ thống mạng}

Đôi khi cần phải kết nối các hệ thống mạng đã tồn tại thành một hệ thống truyền thông mở
rộng. Điều này có thể được thực hiện bằng việc kết nối các mạng thành một phiên bản lớn
hơn nhưng vẫn cùng ``kiểu'' như hệ thống cũ. Ví dụ, trong trường hợp những hệ thống mạng
hình tuyến dựa trên các giao thức mạng Ethernet, thông thường có thể kết nối các trục
chính thành một trục đơn lớn hơn. Việc này được thực hiện với điều kiện sử dụng các thiết
bị khác nhau được biết đến như\textbf{ bộ lặp tín hiệu} (repeater), \textbf{cầu nối}
(bridge), và \textbf{bộ chuyển mạch} (switch), đó là những sự khác biệt không dễ phát hiện
ra nhằm phục vụ cho việc mở rộng hệ thống mạng. Đơn giản nhất trong số đó là bộ lặp tín
hiệu, thiết bị dùng để kết nối trục chính của hai mạng hình tuyến lại với nhau thành một
trục dài hơn (Hình~\ref{fig:fig4.4}). Bộ khuếch đại (bộ lặp tín hiệu) đơn giản chỉ truyền
các tín hiệu về phía sau hay trước giữa hai tuyến gốc ban đầu (thông thường sử dụng với
một vài bộ khuếch đại tín hiệu) mà không cần biết ý nghĩa của những tín hiệu đó là gì.

\begin{figure}[tb] 
  \centering \scalebox{0.45}{\includegraphics{ch5/fig44.pdf}}
  \caption{Xây dựng một mạng hình tuyến lớn từ những mạng nhỏ hơn}
  \label{fig:fig4.4}
\end{figure}

\textbf{Cầu nối} là một thiết bị tương đương, nhưng phức tạp hơn so với bộ lặp tín
hiệu. Cũng như bộ lặp tín hiệu, cầu nối kết nối hai mạng hình tuyến với nhau, nhưng nó
không nhất thiết phải chuyển tiếp tất cả các thông điệp từ tuyến này sang tuyến kia. Thay
vào đó, nó sẽ xem địa chỉ đích đi kèm với thông điệp và chuyển tiếp thông điệp qua kết nối
chỉ khi thông điệp đó đã được chỉ định trước là sẽ được gửi tới một máy tính ở phía bên
kia của kết nối. Do đó, hai thiết bị nằm ở trên cùng một phía của cầu nối có thể trao đổi
thông điệp mà không gây phiền phức cho sự truyền thông ở phía bên kia của cầu. Cầu nối
thường làm việc có hiệu quả hơn so với bộ lặp tín hiệu.

\textbf{Bộ chuyển mạch} về bản chất là một cầu nối có nhiều cổng kết nối, cho phép nó kết
nối tới nhiều hơn hai tuyến. Do đó, bộ chuyển mạch tạo ra một mạng bao gồm một vài tuyến
được mở rộng từ bộ chuyển mạch như những chiếc nan hoa của bánh xe
(Hình~\ref{fig:fig4.4}b). Cũng giống như cầu nối, bộ chuyển mạch sẽ xem xét địa chỉ đích
của tất cả các thông điệp và chỉ chuyển tiếp những thông điệp nào đã được chỉ định trước
sang các tuyến khác. Hơn nữa, mỗi thông điệp được chuyển tiếp chỉ được chuyển tới các
tuyến tương ứng của địa chỉ đích, chính vì vậy mà nó có thể giảm thiểu được giao thông
trên mỗi tuyến.

Cần phải chú ý rằng khi các mạng được kết nối với những thiết bị như bộ lặp tín hiệu, cầu
nối hay thiết bị chuyển mạch, kết quả là một mạng đơn lớn hơn được tạo ra. Mỗi máy tính
tiếp tục truyền tải thông qua hệ thống theo cách thức cũ (sử dụng cùng một giao thức mạng)
nếu hệ thống được xây dựng ban đầu như là một mạng đơn lớn. Điều đó có nghĩa là, các máy
tính cá nhân trong hệ thống không cần biết đến sự tồn tại của các bộ lặp tín hiệu, các cầu
nối, hay các thiết bị chuyển mạch.

Tuy nhiên, các hệ thống mạng được kết nối đôi khi cũng có những đặc tính không tương thích
nhau. Ví dụ, những đặc tính của mạng vòng tròn sử dụng giao thức thẻ bài vòng tròn
không tương thích với mạng Ethernet hình tuyến sử dụng CSMA/CD. Trong những trường hợp
này, các hệ thống mạng cần phải được kết nối theo một cách thức dùng để tạo ra hệ thống
mạng của các mạng, được biết đến như là hệ thống \textbf{liên mạng} (internet), trong đó
những mạng gốc ban đầu duy trì những tính chất riêng của chúng và tiếp tục vận hành như là
những hệ thống độc lập. (Chú ý rằng \textit{liên mạng (internet)} khác với mạng
\textit{Internet}. Mạng Internet, với chữ cái I được viết hoa, được nói đến như một hệ
thống liên mạng đặc biệt, có phạm vi rộng lớn mà ta sẽ nghiên cứu trong một phần khác của
chương này. Có rất nhiều ví dụ về những hệ thống liên mạng. Quả thực, hệ thống truyền
thông qua điện thoại cổ điển đã được sử dụng khá tốt trong các hệ thống liên mạng phạm vi
rộng trước khi mà mạng Internet được phổ biến.)


\begin{figure}[tb] 
  \centering \scalebox{0.45}{\includegraphics{ch5/fig45.pdf}}
  \caption{Bộ dẫn đường kết nối một mạng hình tuyến với một mạng hình sao tạo thành một hệ
    thống liên mạng}
  \label{fig:fig4.5}
\end{figure}

Sự kết nối giữa hai hệ thống mạng tạo thành một hệ thống liên mạng được thực hiện bởi một
thiết bị gọi là \textbf{bộ dẫn đường} (Router). Một bộ dẫn đường là một máy tính thuộc về
cả hai hệ thống mạng ở hai phía của nó với nhiệm vụ là chuyển tiếp các thông điệp từ mạng
này sang mạng kia (Hình~\ref{fig:fig4.5}). Chú ý rằng nhiệm vụ của bộ dẫn đường đặc biệt
to tát hơn nhiều so với các thiết bị như bộ lặp tín hiệu, cầu nối hay thiết bị chuyển mạch
bởi vì một bộ dẫn đường cần phải thực hiện chuyển đổi giữa các đặc tính riêng biệt của hai
mạng gốc ban đầu. Ví dụ, khi truyền phát một thông điệp từ một mạng sử dụng giao thức thẻ
bài vòng tròn tới một mạng sử dụng giao thức CSMA/CD, bộ dẫn đường phải nhận thông điệp sử
dụng một giao thức sau đó lại sử dụng một giao thức khác để truyền thông điệp đó tới mạng
kia.


Ta sẽ xem xét một ví dụ khác sẽ mô tả sự phức tạp khi thực hiện việc dẫn đường của
Router. Đó là vấn đề khi hai mạng kết nối với nhau lại sử dụng hai hệ thống địa chỉ khác
nhau để xác định các máy tính trong mạng. Khi một máy tính trong một mạng muốn gửi một
thông điệp tới một máy tính ở mạng bên kia, nó không thể xác định được máy tính đích theo
cách thức thông thường mà nó vẫn thường thực hiện.

\begin{figure}[t]
  \begin{quotation}
    \noindent
    \textbf{Mạng Enthernet} \vspace{0.3cm}
    \\
    Mạng Ethernet là một tập các chuẩn được triển khai trong một mạng LAN với mô hình mạng
    hình tuyến. Tên gọi của nó được bắt nguồn từ thiết kế mạng Ethernet ban đầu trong đó
    các thiết bị được kết nối với nhau qua cáp đồng trục. Khởi đầu mạng Ethernet được phát
    triển vào những năm 1970 và ngày nay được chuẩn hóa bởi IEEE là một phần của họ chuẩn
    IEEE 802, mạng Ethernet có hầu hết các cách thức chung của một mạng các máy tính. Việc
    cài đặt các card điều khiển mạng của các máy tính cá nhân trong mạng Ethernet có thể
    thực hiện được và khá dễ dàng.

    Ngày nay trên thực tế có một vài phiên bản của mạng Ethernet với những công nghệ tiên
    tiến hơn và tốc độ truyền tải thông tin cũng cao hơn. Tuy nhiên, tất cả những phiên
    bản mới đó vẫn có đủ các đặc tính chung của họ mạng Ethernet. Mỗi một phiên bản trong
    số đó là một khuôn thức mà trong đó dữ liệu được đóng gói trước khi truyền đi, sử dụng
    mã hóa Manchester (một phương pháp mà trong đó đại diện là các bít 0 và 1, với một bít
    0 sẽ đại diện cho một tín hiệu giảm dần và bít 1 đại diện cho một tín hiệu tăng dần)
    để truyền tải thực sự các bít dữ liệu, và sử dụng giao thức CSMA/CD để điều khiển
    quyền truyền phát.
  \end{quotation}
\end{figure}

Trong những trường hợp như vậy, một hệ thống địa chỉ với phạm vi liên mạng được thiết
lập. Kết quả là mỗi thiết bị trong một hệ thống liên mạng có hai địa chỉ: một địa chỉ của
chính mạng gốc ban đầu của nó và một địa chỉ liên mạng mới. Để gửi một thông điệp từ một
máy tính thuộc một trong những mạng gốc tới một máy tính trong mạng khác--máy tính có địa chỉ
liên mạng của gói tin gốc ban đầu, nó sẽ sử dụng hệ thống địa chỉ gốc của mạng cục bộ để
gửi gói tin tới bộ dẫn đường. Bộ dẫn đường sau đó sẽ xem xét bên trong của gói tin nhận
được, thực hiện tìm địa chỉ liên mạng đích sau cùng của thông điệp, dịch địa chỉ đó thành
địa chỉ có định dạng thích hợp với mạng kia, sau đó chuyển tiếp thông điệp tới đích của
nó. Nói một cách ngắn gọn, các thông điệp trong mỗi một mạng gốc tiếp tục được truyền theo
cách thức của hệ thống địa chỉ gốc của mỗi mạng, và bộ dẫn đường được phân công nhiệm vụ
là chuyển đổi giữa các hệ thống.

\subsection*{Truyền thông liên tiến trình}
Các tiến trình thực thi trên các máy tính khác nhau trong một mạng
máy tính (hay thậm chí trên cùng một máy  theo cách thức chia sẻ thời gian)
thường phải liên lạc với các tiến trình khác  để thực hiện những nhiệm vụ đã được xác định. Việc liên lạc này được gọi là sự \textbf{truyền thông liên tiến trình}.

Một trong những phương thức phổ biến trong truyền thông liên tiến trình là mô hình
\textbf{khách/chủ} (client/server). Mô hình này xác định  vai trò của các tiến
trình trên máy trạm, nơi mà sẽ phát sinh các yêu cầu tới các tiến trình khác trên máy chủ, và các tiến trính trên máy chủ  
nơi sẽ thực hiện các yêu cầu của máy trạm.

Một ứng dụng sơ khai trong mô hình khách/chủ đã xuất hiện trên những mạng liên kết tất cả
máy tính trong một nhóm các văn phòng. Trong tình huống như vậy, một máy in đơn lẻ, chất
lượng cao được kết nối vào mạng nơi mà tất cả các máy tính trong đó có thể sử dụng được
máy in đó. Với trường hợp này máy in đã đóng vai trò của một máy chủ (thường được gọi là
\textbf{máy chủ in} (printer server), và các máy tính khác được lập trình để đóng vai của
các máy trạm sẽ gửi các yêu cầu in ấn tới máy chủ in.

Một ứng dụng khác của mô hình khách/chủ cũng sớm được đưa vào sử dụng nhằm giảm chi phí
lưu trữ bằng cách loại bỏ những nhu cầu về các bản sao trùng lặp của những mẫu tin. Ở đây,
một máy tính trong mạng được trang bị một hệ thống lưu trữ thứ cấp có khả năng cao (thường
sử dụng một đĩa từ) mà trên đó chứa toàn bộ các thông tin dữ liệu của một đơn vị. Các máy
tính khác trên mạng có thể yêu cầu truy cập tới các thông tin dữ liệu mà chúng cần. Khi
đó, máy tính chứa thông tin dữ liệu đóng vai trò là một máy chủ (gọi là \textbf{máy chủ
  file}--file server), và các máy tính khác đóng vai trò là các máy trạm sẽ gửi các yêu
cầu truy cập với những file dữ liệu được lưu trữ trên máy chủ file.

\begin{figure}[tb] 
  \centering \scalebox{0.4}{\includegraphics{ch5/fig46.pdf}}
  \caption{Mô hình client/server so sánh với mô hình peer-to-peer}
  \label{fig:fig4.6}
\end{figure}

Ngày nay, mô hình khách/chủ được sử dụng rộng rãi trong các ứng dụng mạng, ta sẽ xem xét ở
phần sau trong chương này. Tuy nhiên, không chỉ mô hình khách/chủ mới hoạt động theo cách
thức như một sự truyền thông liên tiến trình. Một mô hình khác với tên gọi
\textbf{peer-to-peer} (thường được viết tắt là P2P), có những tính chất trái ngược hoàn
toàn với mô hình khách/chủ. Trong khi mô hình khách/chủ bao gồm một tiến trình (trên máy
chủ) thực hiện liên lạc với nhiều tiến trình khác (tại các máy trạm) thì mô hình
peer-to-peer lại bao gồm hai tiến trình trao đổi ngang hàng với nhau
(Hình~\ref{fig:fig4.6}). Ngoài ra, một máy chủ phải chạy liên tục nhằm phục vụ cho các máy
trạm của nó tại bất kỳ thời điểm nào, ngược lại mô hình peer-to-peer với hai tiến trình có
thể thực hiện theo cách thức tạm thời. Ví dụ, các ứng dụng trong mô hình peer-to-peer bao
gồm việc gửi tức thì các thông điệp mà hai người thực hiện đối thoại với nhau qua Internet
cũng như những tình huống như hai người chơi những trò chơi như cờ vua hay cờ đam (một
loại trò chơi gồm 24 quân cờ cho 2 người chơi).


Mô hình peer-to-peer cũng hoạt động theo cách thức chia sẻ file như các bản nhạc hay những
bộ phim qua Internet (đôi khi đi kèm là vấn đề về bản quyền). Trong trường hợp này, những
cá nhân cần tìm kiếm những khoản mục phổ biến có thể quảng bá mong muốn của họ lên
Internet và liên hệ được với những ai sở hữu những khoản mục đó. Sau đó, những khoản mục
này được truyền tải giữa hai phía sử dụng mô hình peer-to-peer. Điều này trái ngược hoàn
toàn với cách tiếp cận của mô hình khách/chủ qua việc thiết lập một ``trung tâm phân
phối'' (máy chủ file) cho các máy trạm tải các bản nhạc (hay ít nhất là tìm thấy các nguồn
của những khoản mục đó). Tuy nhiên, máy chủ trung tâm, đã chứng thực được là một điểm
trung tâm mà tại đó ngành công nghiệp âm nhạc có thể được tuân thủ theo đúng luật bản
quyền, dẫn tới kết quả cuối cùng là sự dỡ bỏ các trung tâm phân phối âm nhạc. Ngược lại,
sự thiếu vắng các trung tâm điều hành như vậy trong mô hình peer-to-peer sẽ khiến nỗ lực
làm cho luật bản quyền có hiệu lực trở nên khó khăn.

Thông thường bạn có thể đọc và nghe về \textit{mạng peer-to-peer}, ví dụ như việc sử dụng
sai những thuật ngữ có thể mắc phải khi những ngôn từ kỹ thuật được thông qua bởi một cộng
đồng phi kỹ thuật. Mạng \textit{peer-to-peer} được biết đến như một hệ thống mà trong đó,
hai tiến trình trao đổi với nhau qua mạng (hoặc liên mạng). Nó không phải là một thuộc
tính của mạng (hay liên mạng). Một tiến trình có thể sử dụng mô hình peer-to-peer để trao
đổi với tiến trình khác thông qua cùng một hệ thống mạng. Vì vậy cần phải nói một cách
chính xác là truyền thông theo cách thức của mô hình peer-to-peer chứ không phải là truyền
thông qua một mạng peer-to-peer.

\subsection*{Các hệ thống phân tán}
Với sự thành công của công nghệ mạng, sự tương tác giữa những máy tính qua các hệ thống
mạng trở nên phổ biến và được thể hiện ở nhiều khía cạnh. Nhiều hệ thống phần mềm hiện
đại, như các hệ thống tìm kiếm hay phục hồi thông tin toàn cầu, các hệ thống kiểm toán có
phạm vi toàn công ty, các trò chơi máy tính, và thậm chí các phần mềm điều khiển chính hệ
thống cơ sở hạ tầng của mạng được thiết kế như là những \textbf{hệ thống phân tán}. Điều
đó có nghĩa là chúng gồm có những đơn vị phần mềm được thực thi dưới dạng các tiến trình trên
các máy tính khác nhau. Ta có thể hình dùng những tiến trình này như là những vị khách cư
trú tại các máy tính khác nhau mà qua đó các máy tính trong một mạng sẽ được gọi là các
\textbf{máy chủ} (host). Điều đó có nghĩa rằng host là một máy tính mà các tiến trình
trú ngụ trên đó theo một hay nhiều ngữ cảnh.

Các hệ thống phân tán đầu tiên được phát triển độc lập từ những hệ thống hỗn tạp. Nhưng
ngày nay, việc nghiên cứu một cách cẩn thận đã cho thấy một cơ sở hạ tầng phổ biến vận
hành trong suốt toàn bộ những hệ thống này, bao gồm cả những thứ như các hệ thống truyền
thông và bảo mật. Đổi lại, kết quả của sự cố gắng đã tạo ra các hệ thống mà có thể cung
cấp cơ sở hạ tầng cơ bản và chính vì vậy, nó cho phép các ứng dụng phân tán được xây dựng
bởi sự phát triển phần hệ thống duy nhất đối với ứng dụng đó.

Một kết quả của nhận định trên là hệ thống đặc tả về giao diện lập trình JavaBeans (được
phát triển bởi Sun Microsystems). Hệ thống này là một môi trường phát triển trợ giúp việc
xây dựng các hệ thống phần mềm phân tán mới. Sử dụng JavaBeans, một hệ thống phân tán được
xây dựng từ những đơn vị được gọi là các bean được tự động thừa kế những đặc tính cơ sở hạ
tầng của hệ thống cha. Do đó, chỉ có những thành phần ứng dụng phụ thuộc duy nhất của một
hệ thống mới mới được phát triển. Một cách tiếp cận khác là môi trường phát triển phần mềm
với tên gọi .NET Framework (được phát triển bởi Microsoft). Với thuật ngữ .NET, các thành
phần của hệ thống phân tán được gọi là các assembly. Mặt khác, bằng việc phát triển các
đơn vị này trong môi trường .NET, chỉ những nét đặc trưng duy nhất đối với một ứng dụng
phổ biến là cần phải được xây dựng dựa trên nền tảng cơ sở hạ tầng có sẵn. Cả hai môi
trường JavaBeans và .NET Framework đều rất đơn giản trong việc phát triển các hệ thống
phần mềm phân tán mới.

\subsection*{Câu hỏi \& Bài tập}

\begin{enumerate}
\item Thế nào là một hệ thống mạng mở?

\item Tóm tắt những đặc tính khác biệt giữa thiết bị lặp tín hiệu và cầu nối

\item Thiết bị dẫn đường là gì?

\item Mô tả một vài mối quan hệ trong xã hội tuân theo mô hình khách/chủ.

\item Mô tả một vài giao thức được sử dụng trong xã hội.

\end{enumerate}



%%% Local Variables: 
%%% mode: latex
%%% TeX-master: "../tindaicuong"
%%% End: 

\input{ch5/internet.tex}

\section{World Wide Web}

Trong phần này ta sẽ tập trung vào thảo luận về một ứng dụng trên mạng Internet mà
nhờ nó thông tin đa phương tiện có thể được phổ biến thông qua mạng Internet. Nó được dựa
trên khái niệm về \textbf{siêu văn bản} (hypertext), một cụm từ mô tả những tài liệu dạng
văn bản mà trong đó có chứa các liên kết, với tên gọi là \textbf{các siêu liên kết}
(hyperlinks), hay chứa những văn bản khác. Ngày nay, siêu văn bản đã mở rộng ra và có thể
bao gồm cả hình ảnh, âm thanh và video, và cũng chính sự mở rộng phạm vi này mà đôi khi nó
còn được gọi là \textbf{siêu phương tiện} (hypermedia).

Khi sử dụng một giao diện đồ họa (GUI), người đọc một tài liệu siêu văn bản có thể lần
theo những siêu liên kết trong tài liệu bằng cách kích chuột vào nó. Ví dụ, giả sử câu
``Sự trình diễn nhạc qua điệu nhảy ‘Bolero’ bởi Maurice Ravel rất ấn tượng'' xuất hiện
trong một tài liệu siêu văn bản và tên \textit{Maurice Ravel} được liên kết tới một tài
liệu khác--có thể cho ta thông tin về nhà soạn nhạc đó. Một người đọc có thể chọn xem
thông tin liên quan đó bằng cách di chuyển trỏ chuột vào tên \textit{Maurice Ravel} và
kích vào nút chuột. Ngoài ra, nếu những siêu liên kết thích ứng được cài đặt, người đọc có
thể nghe được một bản ghi âm của buổi hòa nhạc bằng cách kích chuột vào tên
\textit{Bolero}.

Theo cách đó, một người đọc những tài liệu siêu văn bản có thể khai thác được những tài
liệu liên quan hay lần theo chuỗi từ tài liệu này đến tài liệu khác. Khi nhiều phần khác
nhau của các tài liệu được liên kết tới những tài liệu khác, một mạng lưới thông tin liên
quan với nhau được hình thành. Khi triển khai trên một mạng máy tính, các tài liệu trong
đó như một mạng lưới có thể thường trú trên nhiều máy tính khác nhau, dạng như một lưới
mạng diện rộng. Mạng lưới mà đã phát triển trên mạng Internet mở rộng ra phạm vi toàn cầu
và được biết đến với tên gọi là World Wide Web (thường được viết tắt là \textbf{WWW},
\textbf{W3} hay \textbf{Web}). Một tài liệu siêu văn bản trên World Wide Web thường được
gọi là một \textbf{trang Web} (Web page). Một tập hợp những trang Web có mối quan hệ gần
nhau được gọi là một \textbf{website}.

World Wide Web có nguồn gốc khởi đầu là từ công việc của Tim Berners-Lee, người đã nhận ra
được tiềm năng của việc kết hợp khái niệm liên kết tài liệu với công nghệ liên mạng và đã
đưa ra được phần mềm đầu tiên là việc triển khai WWW vào tháng 12 năm 1990.

\subsection*{Triển khai Web}

Những gói phần mềm cho phép người sử dụng truy cập tới những siêu văn bản trên mạng
Internet thuộc về một trong hai loại: những gói đóng vai trò là những ứng dụng khách, và
những gói đóng vai trò là những ứng dụng phục vụ. Một gói ứng dụng khách thường được đặt
trên máy tính của người sử dụng và được giao nhiệm vụ thu nạp các tài liệu mà được yêu cầu
từ phía người dùng, sau đó trình diễn những tài liệu này cho người dùng xem theo một cách
thức có tổ chức. Ứng dụng khách mà cung cấp giao diện người dùng cho phép một người sử
dụng có thể duyệt qua lại trên Web. Do đó những ứng dụng khách dạng như vậy thường được
gọi là \textbf{trình duyệt} (browser), hay đôi khi còn được gọi là trình duyệt Web. Gói
ứng dụng trên máy chủ (thường gọi là \textbf{ứng dụng phục vụ Web} (Web server)) thường
trú trên một máy tính chứa các tài liệu siêu văn bản sẽ được truy cập tới. Nhiệm vụ của nó
là cung cấp quyền truy cập tới các tài liệu trên nó khi có yêu cầu từ phía các ứng dụng
khách. Nói tóm lại, một người dùng giành được quyền truy cập tới các tài liệu siêu văn bản
thông qua một trình duyệt thường trú trên máy tính của người dùng đó. Trình duyệt này,
đóng vai trò là một ứng dụng khách, thu nạp các tài liệu bằng cách gửi các yêu cầu về dịch
vụ tới các máy chủ Web nằm rải rác trên mạng Internet. Các tài liệu siêu văn bản thông
thường được truyền qua lại giữa các trình duyệt và máy chủ Web sử dụng một giao thức được
biết đến là \textbf{Giao thức Truyền Siêu văn bản} (HTTP: Hypertext Transfer Protocol).

Để có thể định vị và truy nạp các tài liệu trên World Wide Web, mỗi tài liệu được gắn vào
nó là một địa chỉ duy nhất với tên gọi là Địa chỉ \textbf{Uniform Resource
  Locator(URL)}. Mỗi URL chứa thông tin cho phép một trình duyệt liên lạc được tới một máy
chủ tương ứng và yêu cầu nạp tài liệu mong muốn. Do đó, khi xem một trang Web, một người
nào đó trước tiên cần phải cung cấp cho trình duyệt URL của tài liệu mà anh ta hay cô ta
cần nạp và sau đó ra lệnh cho trình duyệt nạp và hiển thị tài liệu đó lên.

\begin{figure}[bt] 
  \centering \scalebox{0.45}{\includegraphics{ch5/fig48.pdf}}
  \caption{Một URL đặc trưng}
  \label{fig:fig4.8}
\end{figure}

Một URL đặc trưng được biểu diễn qua Hình~\ref{fig:fig4.8}. Nó bao gồm bốn phần: giao thức
được sử dụng để giao tiếp với ứng dụng trên máy chủ điều khiển truy cập tới tài liệu, địa
chỉ dễ nhớ của máy tính chứa ứng dụng chủ, đường dẫn cần cho ứng dụng chủ tìm tới thư mục
chứa tài liệu, và cuối cùng là tên của tài liệu. Nói một cách ngắn gọn, URL trong
Hình~\ref{fig:fig4.8} chỉ ra rằng một trình duyệt liên lạc tới một ứng dụng phục vụ Web
đặt trên một máy tính được biết đến là \url{ssenterprise.aw.com} sử dụng giao thức HTTP để
thu nạp tài liệu tên là \texttt{Julius\_Caesar.html} được đặt trong thư mục con
\texttt{Shakespeare} trong thư mục \texttt{authors}.

Đôi khi một URL có thể không nhất thiết phải chứa đủ cả các thành phần được chỉ ra trong
Hình~\ref{fig:fig4.8}. Ví dụ, nếu ứng dụng chủ không cần phải lần theo một đường dẫn thư
mục tới tài liệu, khi đó sẽ không có đường dẫn xuất hiện trong URL nữa. Ngoài ra, đôi khi
một URL sẽ bao gồm chỉ một giao thức và địa chỉ dễ nhớ của một máy tính. Trong trường hợp
này, ứng dụng phục vụ Web tại máy tính đó sẽ trả về một tài liệu được chỉ định trước,
thông thường được gọi là trang chủ (home page), mà thường mô tả thông tin sẵn có trên
website đó. Những URL được làm ngắn gọn như vậy cung cấp một cách thức đơn giản để liên
lạc được với các tổ chức. Ví dụ, URL \url{http://www.aw.com} sẽ dẫn tới trang chủ của Nhà
xuất bản~Addison-Wesley, mà trên đó có chứa các siêu liên kết tới rất nhiều các tài liệu
khác liên quan đến nhà xuất bản và các sản phẩm của họ.

Nhằm đơn giản hóa việc định vị các website, rất nhiều trình duyệt mặc định hiểu rằng giao
thức HTTP sẽ được sử dụng nếu không có giao thức nào được chỉ định. Những trình duyệt này
có thể nạp chính xác trang chủ của Addison-Wesley khi nhận được yêu cầu nạp từ ``URL'' bao
gồm chỉ đơn thuần là \url{www.aw.com}.

Tất nhiên, một người sử dụng Web có thể cần phải tìm kiếm về một chủ đề nào đó hơn là truy
nạp một tài liệu cụ thể. Với mục đích này, rất nhiều website (bao gồm cả các trang chủ của
hầu hết các ISP) cung cấp các dịch vụ của một phương tiện tìm kiếm. Một \textbf{phương
  tiện tìm kiếm} là một gói phần mềm được thiết kế nhằm trợ giúp cho người sử dụng Web xác
định các tài liệu liên quan tới các chủ đề khác nhau. Để sử dụng một phương tiện tìm kiếm,
người sử dụng cần gõ một tập các từ hay cụm từ mà tài liệu mong muốn tìm được có thể chứa
chúng, sau đó phương tiện tìm kiếm sẽ quét toàn bộ các bản ghi của nó, đưa ra báo cáo về
các tài liệu mà nội dung có chứa văn bản cần xác định. Việc cải tiến công nghệ cho các
phương tiện tìm kiếm, bao gồm những phương pháp tốt hơn cho việc xác định các tài liệu
liên quan và cải tiến những hệ thống xây dựng và lưu trữ những bản ghi nằm trong hệ thống
tìm kiếm, đang là một quá trình tiếp diễn.


\begin{figure}[t]
  \begin{quotation}
    \noindent
    \textbf{World Wide Web Consortium} \vspace{0.3cm}
    \\
    The World Wide Web Consortium (W3C) được hình thành vào năm 1994 nhằm đẩy mạnh World
    Wide Web bằng cách phát triển các quy ước chuẩn (được biết đến là các chuẩn W3C). Trụ
    sở W3C được đặt tại CERN, phòng thí nghiệm vật lý hạt nhân năng lượng cao tại Geneva,
    Thụy Sĩ. CERN là nơi mà ngôn ngữ đánh dấu HTML gốc được phát triển theo giao thức HTTP
    nhằm truyền tải các tài liệu HTML qua mạng Internet. Ngày nay W3C là nguồn gốc của rất
    nhiều chuẩn (bao gồm cả những chuẩn cho XML và rất nhiều các ứng dụng đa phương tiện)
    mà có thể tương thích trên diện rộng của những sản phẩm Internet. Bạn có thể nghiên
    cứu thêm về W3C thông qua địa chỉ website của nó tại \url{http://www.w3c.org}
  \end{quotation}
\end{figure}

\subsection*{HTML}
Một tài liệu siêu văn bản truyền thống cũng tương tự như một tệp văn bản vì văn bản của nó
được mã hóa ký tự qua các ký tự sử dụng một hệ thống bảng mã như ASCII hay Unicode. Sự
khác biệt là một tài liệu siêu văn bản có thể chứa các ký tự đặc biệt, gọi là các
\textbf{thẻ} (tag), mà mô tả tài liệu đó xuất hiện trên màn hình hiển thị như thế nào, các
tài nguyên đa phương tiện (như các hình ảnh) đi kèm với tài liệu là gì, và các mục nằm bên
trong tài liệu đó được liên kết tới những tài liệu khác ra sao. Hệ thống các thẻ này được
biết đến là \textbf{Ngôn ngữ Đánh dấu Siêu văn bản (HTML)} (HTML: Hypertext Markup
Language).

Theo cách thức đó, nó chính là thể hiện dưới dạng HTML mà tác giả của một trang Web mô tả
thông tin một trình duyệt cần có để trình diễn trang đó lên màn hình của người sử dụng
và tìm ra bất kỳ tài liệu nào có liên quan đến trang hiện tại. Quá trình này tương tự như
việc thêm các quy định xếp chữ vào một văn bản được gõ trơn (plain text) (có thể sử dụng
một bút màu đỏ) mà một máy xếp chữ sẽ biết được làm thế nào mà các tài liệu có thể xuất
hiện theo một định dạng cuối cùng. Trong trường hợp của siêu văn bản, việc đánh dấu đỏ
được thay thế bằng các thẻ HTML, và một trình duyệt đơn giản là đóng vai trò của máy xếp
chữ, đọc các thẻ HTML nhằm nhận biết được làm thế nào để cho văn bản được trình diễn lên
màn hình máy tính.

\begin{figure} 
  \centering \scalebox{0.5}{\includegraphics{ch5/fig49.pdf}}
  \caption{Một trang Web đơn giản}
  \label{fig:fig4.9}
\end{figure}

Bản HTML được mã hóa (gọi là bản nguồn) của một trang Web cực kỳ đơn giản được chỉ ra
trong Hình~\ref{fig:fig4.9}a. Cần chú ý rằng các thẻ được phác họa qua các ký tự
\texttt{<} và \texttt{>}.  Tài liệu HTML nguồn bao gồm hai phần--phần đầu (được bao quanh
bởi cặp thẻ \texttt{<head>} và \texttt{</head>}) và phần thân (được bao quanh bởi cặp thẻ
\texttt{<body>} và \texttt{</body>}). Sự khác biệt giữa phần đầu và phần thân của một
trang Web tương tự như phần đầu và phần thân của một cuốn sổ ghi nhớ trong nội bộ một tổ
chức. Trong cả hai trường hợp, phần đầu thường chứa thông tin sơ bộ về tài liệu (ngày,
tiêu đề,… trong trường hợp của sổ ghi nhớ). Phần thân chứa nội dung cốt lõi của tài liệu,
mà trong trường hợp của trang web thì đó là các tài liệu được trình chiếu trên màn hình
máy tính khi trang đó được hiển thị lên.

Phần đầu của trang Web hiển thị trong Hình~\ref{fig:fig4.9}a chứa chỉ duy nhất tiêu đề của
tài liệu (được bao quanh bởi cặp thẻ ``title''). Tiêu đề này chỉ có mang tính chất là dẫn
chứng cho tài liệu; nó không phải là phần được hiển thị lên trên màn hình máy tính. Nội
dung mà sẽ hiển thị lên trên màn hình máy tính được chứa trong phần thân của tài liệu.


Mục đầu tiên trong phần thân của tài liệu trong Hình~\ref{fig:fig4.9}a là tiêu đề cấp độ 1
(được bao quanh bởi cặp thẻ \texttt{<h1>} và \texttt{</h1>}) chứa dòng văn bản ``My Web
Page.''. Là tiêu đề cấp độ 1 có nghĩa là trình duyệt sẽ hiển thị văn bản này nổi bật trên
màn hình. Mục tiếp theo trong phần thân là một đoạn văn bản (được bao quanh bởi cặp thẻ
\texttt{<p>} và \texttt{</p>}) chứa đoạn văn bản ``Click here for another
page.''. Hình~\ref{fig:fig4.9}b chỉ ra trang web sẽ được hiển thị như thế nào trên màn
hình máy tính thông qua một trình duyệt.

Trong hình dạng hiện tại của nó, trang Web trong Hình~\ref{fig:fig4.9} không có đầy đủ các
chức năng nên khi đó sẽ không có gì xảy ra khi người xem kích chuột vào từ \textit{here},
 mặc dù theo yêu cầu trình duyệt sẽ phải hiển thị một trang khác. Để đạt được yêu cầu, ta cần phải liên kết từ \textit{here} tới một tài liệu khác.

\begin{figure}
  \centering \scalebox{0.5}{\includegraphics{ch5/fig410.pdf}}
  \caption{Một trang Web mở rộng}
  \label{fig:fig4.10}
\end{figure}

Ta giả sử rằng khi từ here được kích chuột vào, ta muốn trình duyệt truy lục và hiển thị
trang web tại URL \url{http://crafty.com/demo.html}. Để làm được điều đó, trước tiên ta
phải bao quanh từ \texttt{here} trong bản nguồn của trang bằng cặp thẻ \texttt{<a>} và
\texttt{</a>}, cặp thẻ này được gọi là thẻ mấu neo. Bên trong thẻ mở mấu neo, ta chèn thêm
tham số \texttt{href= “http://crafty.com/demo.html”} (như trong Hình~\ref{fig:fig4.10}a)
với mục đích chỉ ra siêu văn bản tham chiếu (href: hypertext reference) kết hợp với thẻ
trong URL ngay sau dấu bằng (\url{http://crafty.com/demo.html})

Khi có thêm các thẻ mấu neo, trang Web bây giờ sẽ hiển thị trên màn hình máy tính như
trong Hình~\ref{fig:fig4.10}b. Chú ý rằng sự hiển thị này tương tự như trong
Hình~\ref{fig:fig4.9}b ngoại trừ từ \textit{here} được làm cho nổi bật bằng mầu sắc, điều
đó chỉ ra rằng nó là một liên kết tới một tài liệu Web khác. Việc kích chuột vào những cụm
từ nổi bật như vậy sẽ khiến cho trình duyệt truy lục và hiển thị tài liệu Web kết hợp
trong liên kết đó. Chính vì vậy thông qua các thẻ mấu neo mà các tài liệu Web được liên
kết tới những tài liệu khác.

Cuối cùng, ta cũng cần phải được nói sơ qua làm thế nào để một hình ảnh có thể được
hiển thị trên một trang Web đơn giản. Với ý định này, giả sử rằng một ảnh mã hóa theo định
dạng JPEG mà ta muốn chèn vào được lưu trữ trong cùng thư mục với bản nguồn HTML của
trang Web trên site chủ HTTP. Ngoài ra, ta cũng giả sử rằng tên của tệp ảnh đó là
\texttt{OurImage.jpg}. Với những điều kiện như vậy, ta có thể ra chỉ thị cho một
trình duyệt hiển thị ảnh đó trên đầu của trang Web bằng cách chèn vào thêm thẻ hình ảnh
(img) \texttt{<img src = "OurImage.jpg">} ngay sau thẻ \texttt{<body>} trong tài liệu
nguồn HTML. Điều này diễn đạt cho trình duyệt hiểu là một ảnh có tên là
\texttt{OurImage.jpg} có thể được hiển thị tại vị trí đầu tiên của tài liệu. (Ký hiệu
\texttt{src} là dạng viết ngắn gọn của từ ``source'', có nghĩa là thông tin đằng sau dấu
bằng chỉ ra đường dẫn tới tệp hình ảnh mà sẽ được hiển thị.) Khi trình duyệt tìm thấy thẻ
này, nó sẽ gửi một thông điệp ngược trở về ứng dụng chủ HTTP mà từ đó, ứng dụng chủ sẽ
nhận được yêu cầu của tài liệu gốc tới tệp hình ảnh \texttt{OurImage.jpg} và sau đó trình
duyệt sẽ hiển thị hình ảnh một cách thích hợp.

Nếu ta di chuyển thẻ hình ảnh tới cuối cùng của tài liệu, ngay trước thẻ
\texttt{</body>}, khi đó trình duyệt sẽ hiển thị hình ảnh tại vị trí cuối cùng của trang
Web. Tất nhiên, có nhiều kỹ thuật phức tạp hơn cho việc đặt vị trí của một hình ảnh trên
một trang Web, nhưng chúng không nhất thiết phải được đề cập tới ở đây.

\subsection*{XML}

HTML về cơ bản là một hệ thống ký hiệu mà qua đó một tài liệu văn bản với việc hiển thị
của tài liệu đó có thể được mã hóa như là một tệp văn bản đơn giản. Theo cách thức tương
tự thì ta cũng có thể mã hóa một tài liệu không còn là nguyên bản như những tệp văn
bản--một ví dụ là về các bản nhạc. Khi lưu trữ thông tin về một mẫu khuông nhạc, các đường
gạch nhịp và nốt trong đó âm nhạc được miêu tả theo cách truyền thống không được thể hiện
theo định dạng từng ký tự một của những tệp văn bản. Tuy nhiên, ta có thể khắc phục
được vấn đề này bằng cách phát triển một hệ thống ký hiệu thay thế. Nói một cách chính xác
hơn, ta có thể thỏa thuận trình bày bắt đầu của khuông nhạc bằng thẻ \texttt{<staff
  clef = ``treble''>}, kết thúc một khuông nhạc bằng thẻ \texttt{</staff>}, trình bày ký
hiệu nhịp theo dạng \texttt{<time> 2/4 </time>}, thẻ bắt đầu và kết thúc của một nhịp là
\texttt{<measure>} và \texttt{</measure>}, và theo một thứ tự định sẵn thì một nốt như nốt
thứ tám trên điệu thứ C được ký hiệu là \texttt{<notes>eight C</notes>},… Khi đó, đoạn văn
bản sau:
\begin{verbatim}
      <staff clef = "treble">
           <key>C minor</key>
           <time> 2/4 </time>
           <measure> 
               <rest> egth </rest> 
               <notes> egth G, egth G, egth G </notes>
           </measure>
           <measure> 
               <notes> hlf E </notes>
           </measure>
      </staff>
\end{verbatim}
có thể được sử dụng để mã hóa bản nhạc chỉ ra trong Hình~\ref{fig:fig4.11}.

\begin{figure}[bt] 
  \centering \scalebox{0.4}{\includegraphics{ch5/fig411.pdf}}
  \caption{Hai khuông đầu tiên trong bản Symphony thứ năm của Beethoven}
  \label{fig:fig4.11}
\end{figure}


Bằng việc sử dụng những ký hiệu như vậy, một bản nhạc có thể được mã hóa, sửa đổi, lưu trữ
và truyền qua mạng Internet như những tệp văn bản. Hơn nữa, phần mềm có thể được viết ra
nhằm trình diễn nội dung của những tệp văn bản như vậy theo một khuôn dạng âm nhạc truyền
thống hay thậm chí có thể chơi được bản nhạc đó trên một nhạc cụ điện tử.

Chú ý rằng hệ thống mã hóa bản nhạc của ta bao gồm khuôn dạng tương tự được sử dụng bởi
ngôn ngữ HTML. Ta đã lựa chọn cách thức ký hiệu những thẻ nhận dạng các thành phần bởi cặp
ký tự \texttt{<} và \texttt{>}. Ta cũng quyết định chỉ ra điểm bắt đầu và kết thúc của
những cấu trúc đó (ví dụ như một khuông nhạc, một chuỗi các nốt nhạc, hay một nhịp nhạc)
bởi những thẻ có tên trùng nhau-–thẻ kết thúc được ký hiệu thêm một ký tự \texttt{/} (thẻ
\texttt{<measure>} được kết thúc bằng thẻ \texttt{</measure>)}. Và ta cũng đã quyết định
chỉ ra những thuộc tính đặc biệt bên trong các thẻ bằng các biểu thức như \texttt{clef =
  “treble”}. Khuôn dạng tương tự này có thể cũng được sử dụng để phát triển các hệ thống
mô tả những định dạng khác như các biểu thức toán học hay đồ họa.

Ngôn ngữ \textbf{đánh dấu mở rộng} (XML: Extensible Markup Language) là một khuôn dạng
được chuẩn hóa (tương tự như ví dụ về bản nhạc của ta) cho việc thiết kế những hệ
thống ký hiệu cho việc hiển thị những dữ liệu dạng như các tệp văn bản. (Trên thực tế, XML
là một ngôn ngữ dẫn xuất được làm đơn giản hóa từ một tập các chuẩn cũ hơn gọi là Standard
Generalized Markup Language, với ký hiệu viết tắt là SGML.)  Theo như chuẩn XML, những hệ
thống ký hiệu với tên gọi là \textbf{ngôn ngữ đánh dấu} được phát triển cho việc mô tả
toán học, trình diễn đa phương tiện, và âm nhạc. Nói tóm lại, HTML là ngôn ngữ đánh dấu
dựa trên chuẩn XML mà được phát triển nhằm hiển thị các trang Web. (Thực tế là phiên bản
gốc của HTML được phát triển trước chuẩn XML đã được làm cho vững chắc, và do đó một vài
tính năng của HTML không thích ứng hoàn toàn với XML. Điều này cũng giải thích vì sao bạn
có thể đã thấy những tham khảo về XHTML, đó là một phiên bản của HTML với một sự kết hợp
khá chặt chẽ với XML.)

XML cung cấp một tiền lệ tốt cho việc làm thế nào mà các chuẩn được thiết kế được ứng dụng
trong phạm vi rộng. Đúng hơn là với những ngôn ngữ đánh dấu không liên quan cho việc mã
hóa rất nhiều dạng tài liệu, việc thiết kế một cách độc đáo phương pháp biểu diễn bằng XML
là nhằm phát triển một chuẩn chung cho các ngôn ngữ đánh dấu khác. Với chuẩn này, những
ngôn ngữ đánh dấu có thể được khai thác trong rất nhiều các ứng dụng. Những ngôn ngữ đánh
dấu được phát triển theo cách thức này đều có một tính chất giống nhau mà cho phép chúng
kết hợp với nhau nhằm thu được những ngôn ngữ đánh dấu dùng cho những ứng dụng phức tạp
như những tài liệu văn bản chứa các phân đoạn của một bản nhạc và các biểu thức toán học.

Cuối cùng ta cũng cần phải chú ý rằng XML công nhận sự phát triển của những ngôn ngữ
đánh dấu mới khác so với HTML mà trong đó chúng có mục đích là làm nổi bật ngữ nghĩa hơn
là việc hiển thị. Ví dụ, với HTML các thành phần (ingredient) trong một công thức làm món
ăn có thể được đánh dấu sao cho chúng xuất hiện như là một danh sách mà trong đó mỗi thành
phần được đặt trên một dòng riêng biệt. Nhưng nếu ta sử dụng các thẻ định hướng có
ngữ nghĩa (semantic-oriented), các thành phần trong công thức đó có thể được đánh dấu dưới
dạng như những thành phần hợp thành tổng quát (có thể chỉ sử dụng các thẻ như
\texttt{<ingredient>} và \texttt{</ingredient>}) hơn là các mục chọn đơn thuần trong một
danh sách. Sự khác biệt ở đây tuy là rất nhỏ nhưng lại khá quan trọng. Cách tiếp cận theo
hướng có ngữ nghĩa sẽ cho phép các công cụ tìm kiếm có thể xác định những công thức món ăn
chứa hay không chứa những thành phần nào đó, đây sẽ là một cải tiến đáng kể vượt qua tình
trạng hiện tại của đề tài nghiên cứu mà trong đó chỉ những công thức chứa hay không chứa
những từ cụ thể mới được xác định. Nói một cách chính xác hơn, nếu các thẻ có ngữ nghĩa
được sử dụng, một công cụ tìm kiếm có thể xác định công thức cho món lasagna không chứa
rau bina (spinach), trong khi đó một cách thức tìm kiếm dựa vào chỉ đơn thuần là nội dung
của từ sẽ bỏ qua một công thức mà bắt đầu bằng câu ``This lasagna does not contain
spinach.'' Tóm lại, bằng việc sử dụng một chuẩn có phạm vi trên mạng Internet cho việc
đánh dấu các tài liệu theo hướng có ngữ nghĩa hơn là việc hiển thị, một Web \textit{ngữ
  nghĩa} phạm vi toàn cầu (World Wide Semantic Web) có thể sẽ được tạo ra thay thế cho Web
cú pháp phạm vi toàn cầu (World Wide Syntactic Web) mà ta đang có hiện nay.

\subsection*{Các hoạt động phía client và phía server}

Bây giờ ta xem xét các bước được yêu cầu đối với một trình duyệt để có thể truy nạp
được trang Web đơn giản đã chỉ ra trong Hình~\ref{fig:fig4.10} và hiển thị nó lên màn hình
máy tính qua trình duyệt. Trước tiên, đóng vai trò của một ứng dụng khách, trình duyệt sẽ
sử dụng thông tin có được trong URL (có thể nhận được từ người sử dụng trình duyệt) để
liên lạc tới ứng dụng phục vụ Web, điều khiển truy cập tới trang Web và yêu cầu một bản
sao của trang Web cần phải được truyền tới ứng dụng khách. Ứng dụng chủ sẽ đáp lại yêu cầu
bằng cách gửi tài liệu văn bản được hiển thị trong Hình~\ref{fig:fig4.10}a đến trình
duyệt. Trình duyệt sau đó sẽ dịch các thẻ HTML có trong tài liệu nhằm xác định trang Web
đó cần được hiển thị như thế nào và trình diễn tài liệu đó trên màn hình máy tính của
trình duyệt một cách phù hợp. Người sử dụng trình duyệt sẽ nhìn thấy một hình ảnh như được
mô tả trong Hình~\ref{fig:fig4.10}b. Nếu người dùng sau đó kích chuột vào từ
\textit{here}, trình duyệt sẽ sử dụng URL trong thẻ mấu neo kết hợp trong đó để liên lạc
tới ứng dụng chủ tương ứng nhằm thu nạp và hiển thị một trang Web khác. Nói tóm lại, quá
trình này bao gồm chỉ đơn thuần là thao tác lấy và hiển thị các trang Web một cách trực
tiếp bởi người dùng.

Nhưng nếu ta muốn một trang Web trong đó bao gồm cả hoạt ảnh (animation) hay một
trang Web cho phép khách hàng điền đầy đủ những thông tin vào một đơn đặt hàng và gửi đơn
đặt hàng đó đi? Những thứ đó cần phải yêu cầu hoạt động bổ trợ từ trình duyệt hay ứng dụng
phục vụ Web. Những hoạt động như vậy được gọi là những hoạt động \textbf{phía client}
(client-side) nếu chúng được thực hiện bởi một ứng dụng khách (ví dụ như một trình duyệt)
hay \textbf{phía server} (server-side) nếu chúng được thực hiện bởi một ứng dụng phục vụ
(ví dụ như ứng dụng phục vụ Web).

Hoạt ảnh trên một trang Web thông thường là một hoạt động phía client. Thông tin cần có
cho hoạt ảnh được truyền tới trình duyệt cùng với văn bản có trong trang Web. Sau đó hoạt
ảnh mới được thực thi dưới sự điều khiển của trình duyệt.

Ngược lại, giả sử một đại lý du lịch yêu cầu các khách hàng cần phải xác định đích đến
mong muốn và ngày du lịch, khi đó đại lý sẽ trình chiếu cho khách hàng một trang Web đã
được tùy biến chỉ chứa thông tin thích hợp với nhu cầu của khách hàng. Trong trường hợp
này, website của đại lý du lịch trước tiên sẽ cung cấp một trang Web và được hiển thị cho
một khách hàng với những đích đến có thể. Dựa trên những thông tin cơ bản này, khách hàng
sẽ chỉ ra những đích đến yêu thích và những ngày du lịch mà họ mong muốn (một hoạt động
phía client). Thông tin này sau đó sẽ được truyền về ứng dụng chủ của đại lý nơi mà nó sẽ
được dùng để tạo dựng một trang Web tương ứng đã được tùy biến (một hoạt động phía server)
và sau đó sẽ được gửi tới trình duyệt của khách hàng.

Một ví dụ tương tự xảy ra trong trường hợp một người sử dụng những dịch vụ của một công cụ
tìm kiếm. Ở đây, người này chỉ ra một chủ đề ưa thích (một hoạt động phía client) mà sau
đó được truyền tới công cụ tìm kiếm, nơi mà một trang Web được tùy biến xác định những tài
liệu nào có thể phù hợp được tạo dựng (một hoạt động phía server) và gửi trả về cho ứng
dụng khách.

Có rất nhiều hệ thống thực hiện các hoạt động phía client và phía server, mỗi hệ thống đều
cạnh tranh với hệ thống khác bằng những tính năng nổi bật của mình. Có một cách thức điều
khiển các hoạt động phía client tuy đã ra đời từ sớm những vẫn khá phổ biến, đó là việc
triển khai viết các chương trình dưới mã của ngôn ngữ Javascript (được phát triển bởi Công
ty truyền thông Netscape) bên trong tài liệu nguồn HTML của một trang Web. Khi đó, một
trình duyệt có thể bóc tách những chương trình này và thực hiện chúng theo đúng yêu
cầu. Một cách tiếp cận khác (được phát triển bởi Sun Microsystems) là trước tiên truyền
một trang Web tới trình duyệt và sau đó truyền những đơn vị chương trình bổ trợ gọi là
applet (được viết bằng ngôn ngữ Java) tới trình duyệt theo yêu cầu bên trong tài liệu
nguồn HTML. Vẫn còn một cách tiếp cận khác nữa đó là hệ thống Flash (được phát triển bởi
Macromedia), thông qua hệ thống Flash, những trình diễn đa phương tiện trên diện rộng có
thể được triển khai một cách đơn giản.


Một cách thức điều khiển các hoạt động phía server là sử dụng một tập các chuẩn gọi là CGI
(Common Gateway Interface) mà qua đó các ứng dụng khách có thể yêu cầu thực thi các chương
trình được lưu trữ trên một máy chủ. Một biến tố của cách tiếp cận này (được phát triển
bởi Sun Microsystems) là cho phép các ứng dụng khách khiến  những đơn vị chương trình
như servlet được thực thi trên phía ứng dụng chủ. Một phiên bản được đơn giản hóa của cách
tiếp cận servlet có thể được sử dụng khi yêu cầu được gửi tới phía server là việc tạo dựng
một trang Web được tùy biến, như trong ví dụ về đại lý du lịch. Trong trường hợp này, các
mẫu trang Web được gọi là JavaServer Pages (JSP) được lưu trữ tại máy chủ phục vụ Web và
được bổ sung từ những thông tin nhận từ phía một ứng dụng khách nào đó. Một cách tiếp cận
tương tự được sử dụng bởi Microsoft, nơi mà các mẫu từ những trang Web được tùy biến được
tạo dựng ra với tên gọi là Active Server Page (ASP). Ngược lại với những hệ thống đòi hỏi
cần phải có bản quyền, PHP (tiền thân xuất phát từ Personal Home Page nhưng ngày nay được
hiểu với nghĩa mới là PHP Hypertext Processor) là một hệ thống mã nguồn mở cũng rất thích hợp cho việc triển
khai các tính năng phía server.

Cuối cùng, sẽ thật là thiếu sót nếu ta không nhận ra những vấn đề về an ninh và 
quy tắc phát sinh từ việc cho phép những ứng dụng khách và ứng dụng phục vụ thực hiện những
chương trình trên máy tính của người khác. Trên thực thế những ứng dụng phục vụ Web đó
thường truyền tải các chương trình tới phía các ứng dụng khách nơi mà chúng được thực thi
dẫn tới những sự nghi ngờ liên quan đến vấn đề về nội quy trên phía ứng dụng chủ và những
vấn đề về bảo mật trên phía ứng dụng khách. Nếu ứng dụng khách thực thi bất kỳ chương
trình nào một cách mù quáng được gửi tới từ một ứng dụng phục vụ Web, nó có thể mở cửa
chính mình cho các hoạt động nguy hiểm được thực hiện bởi ứng dụng chủ. Tương tự như vậy,
trên thực tế những ứng dụng khách này có thể là nguyên nhân khiến cho các chương trình có
thể được thực thi trên ứng dụng chủ dẫn tới những vấn đề về nội quy trên phía ứng dụng
khách và những vấn đề liên quan đến bảo mật trên phía ứng dụng chủ. Nếu một ứng dụng chủ
nào đó thực hiện một cách mù quáng bất kỳ chương trình nào được gửi tới mình từ một ứng
dụng khách bất kỳ, những lỗ hổng an ninh và sự thiệt hại tiềm năng trên máy chủ có thể sẽ
xảy ra.

\subsection*{Câu hỏi \& Bài tập}

\begin{enumerate}
\item Một URL là gì? Một trình duyệt là gì?
\item Một ngôn ngữ đánh dấu là gì?
\item Sự khác nhau giữa HTML và XML là gì?
\item Mục đích của những thẻ HTML dưới đây là gì?

  \begin{inparaenum}[a.]
  \item \texttt{<html>} \qquad
  \item \texttt{<head>} \qquad
  \item \texttt{</body>} \qquad
  \item \texttt{<fa>}
  \end{inparaenum}

\item Những cụm từ phía client (client-side) và phía server (server-side) đề cập tới vấn
  đề gì?
\end{enumerate}






























%%% Local Variables: 
%%% mode: latex
%%% TeX-master: "../tindaicuong"
%%% End: 


\section{Phần đọc thêm: Các giao thức Internet}
\label{sec:4.4}
Trong phần này ta sẽ khám phá những thông điệp được truyền qua mạng Internet như thế
nào. Quá trình truyền này yêu cầu sự cộng tác của tất cả các máy tính trên hệ thống, và do
đó phần mềm điều khiển quá trình này cần phải đặt trên mọi máy tính trên mạng
Internet. Ta bắt đầu nghiên cứu cấu trúc toàn diện của phần mềm này.

\subsection*{Cách tiếp cận theo lớp tới các phần mềm trên mạng Internet}

Một nhiệm vụ chính của phần mềm mạng là cung cấp cơ sở hạ tầng cho việc truyền tải các
thông điệp từ một máy tính này đến một máy tính khác. Trên mạng Internet, hoạt động truyền
thông điệp được hoàn thành dựa trên phân cấp của các đơn vị phần mềm. Việc này tương tự
như bạn gửi một gói quà từ vùng West Coast ở Mỹ tới vùng East Coast ở Mỹ
(Hình~\ref{fig:fig4.12}). Trước hết, bạn sẽ thực hiện bọc quà lại thành một gói và viết
lên gói đó địa chỉ thích hợp. Sau đó, bạn sẽ gửi gói quà này tới một công ty vận chuyển
như U.S. Postal Service. Công ty vận chuyển có thể đặt gói quà đó cùng với những thứ khác
trong một công-ten-nơ và gửi nó tới một hãng hàng không mà các dịch vụ của nó đã được đăng
ký từ trước. Hãng hàng không này đặt công ten nơ trên một máy bay và chuyển tới thành phố
đích, hãng hàng không sẽ gỡ bỏ công ten nơ đó xuống từ máy bay chuyên chở của mình và giao
nó cho một công ty vận chuyển nhằm chuyển tới đích. Tiếp đó, công ty vận chuyển sẽ gỡ gói
hàng của bạn ra khỏi công ten nơ và giao nó tới địa chỉ.

\begin{figure} [tbh]
  \centering \scalebox{0.35}{\includegraphics{ch5/fig412.pdf}}
  \caption{Ví dụ việc chuyển gói hàng hoá}
  \label{fig:fig4.12}
\end{figure}

Nói tóm lại, quá trình chuyên chở của gói hàng sẽ được thực hiện theo dạng cây ba cấp
(tầng):
\begin{inparaenum}[(a)]
\item cấp người sử dụng (bao gồm bạn và bạn bè của bạn),
\item công ty vận chuyển, và
\item hãng hàng không.
\end{inparaenum}
Mỗi cấp sử dụng cấp thấp hơn như là một công cụ trừu tượng. (Bạn không cần quan tâm tới
những chi tiết của công ty vận chuyển, và công ty vận chuyển cũng không cần quan tấm tới
những điều hành cục bộ của hãng hàng không.) Mỗi cấp độ trong cây ba cấp đều có những đại
diện ở cả nơi gửi và nơi nhận hàng,.

\begin{figure}[tbh]
  \centering \scalebox{0.35}{\includegraphics{ch5/fig413.pdf}}
  \caption{Các tầng phần mềm Internet}
  \label{fig:fig4.13}
\end{figure}

Ví dụ như trường hợp với sản phẩm điều khiển truyền thông qua mạng Internet, trừ phần mềm
Internet có bốn tầng chứ không phải ba tầng, thì mỗi phần mềm bao gồm một tập các chương
trình con đúng hơn là con người và công việc kinh doanh. Bốn tầng (cấp) này được biết đến
là \textbf{tầng ứng dụng}, \textbf{tầng vận chuyển}, \textbf{tầng mạng} và \textbf{tầng
  liên kết} (Hình~\ref{fig:fig4.13}). Tất cả các tầng đều hiện diện trên mỗi máy tính trên
mạng Internet. Một thông điệp thông thường đều bắt nguồn từ tầng ứng dụng. Từ đó nó được
chuyển qua xuống tầng vận chuyển và tầng mạng để chuẩn bị cho việc truyền tải, và cuối
cùng nó được truyền đi bởi tầng liên kết. Thông điệp đó được nhận bởi tầng liên kết phía
bên đích và được chuyển lên tầng trên cho đến khi nó được giao cho tầng ứng dụng tại đích
nhận của thông điệp.

Ta hãy xem xét quá trình này một cách kỹ lưỡng hơn bằng cách lần theo vết của một
thông điệp khi nó tìm đường qua hệ thống (Hình~\ref{fig:fig4.14}). Ta bắt đầu quá
trình lần vết tại tầng ứng dụng.


Tầng ứng dụng bao gồm các đơn vị phần mềm như phần mềm khách và phần mềm chủ mà sử dụng
truyền thông Internet để thực hiện những nhiệm vụ của chúng. Mặc dù tên của chúng là giống
nhau, tầng này không bị hạn chế các phần mềm theo sự phân loại ứng dụng được giới thiệu
trong Mục~\ref{sec:3.2}, nhưng cũng bao gồm nhiều gói tiện ích. Ví dụ, phần mềm truyền tải tệp sử
dụng giao thức truyền tệp (FTP) hay phần mềm cung cấp khả năng truy cập từ xa sử dụng
telnet trở nên phổ biến đến nỗi mà chúng thông thường được xem như là phần mềm tiện ích.


\begin{figure} 
  \centering \scalebox{0.5}{\includegraphics{ch5/fig414.pdf}}
  \caption{Truyền một thông điệp qua Internet}
  \label{fig:fig4.14}
\end{figure}


Tầng ứng dụng sử dụng tầng vận chuyển để gửi và nhận thông điệp qua Internet theo cùng một
cách thức mà ta sử dụng một công ty vận chuyển để gửi và nhận các gói hàng. Tương tự
như trách nhiệm của bạn là phải cung cấp một địa chỉ chi tiết hợp lệ với công ty vận
chuyển, trách nhiệm của tầng ứng dụng cũng là cung cấp một địa chỉ hợp lệ cho tầng vận
chuyển. (Để thực hiện được yêu cầu này, tầng ứng dụng có thể sử dụng các dịch vụ tên miền
trên Internet để chuyển đổi từ địa chỉ dễ nhớ được sử dụng bởi người dùng sang địa chỉ IP
tương thích với mạng Internet.)

Nhiệm vụ chính của tầng vận chuyển là chấp nhận các thông điệp đến từ tầng ứng dụng và đảm
bảo rằng những thông điệp này là đúng định dạng cho việc truyền tải qua mạng Internet. Để
phục vụ cho mục đích sau đó, tầng vận chuyển chia các thông điệp dài thành những đoạn nhỏ
(segment), mà sẽ được truyền qua mạng Internet như những đơn vị độc lập nhau. Việc chia
nhỏ này là cần thiết vì một thông điệp đơn và dài có thể làm tắc nghẽn luồng đi của những
thông điệp khác tại những điểm nút trên mạng Internet nơi mà rất nhiều thông điệp phải
được truyền qua những đoạn đường giao nhau.

Thật vậy, những đoạn thông điệp nhỏ có thể truyền xen lẫn với nhau tại những điểm nút thay vì một thông điệp dài bắt ép những thông điệp khác phải chờ trong khi nó truyền qua (giống
như những chiếc ô tô con phải chờ một đoàn tầu dài đi qua tại một ngã tư đường sắt).

Tầng vận chuyển thêm những số theo một trình tự vào những đoạn thông điệp nhỏ để các đoạn
này có thể được ghép nối lại tại đích của thông điệp. Sau đó nó gắn thêm địa chỉ đích vào
mỗi đoạn thông điệp và chuyển giao những đoạn thông điệp được đánh địa chỉ này, được biết
đến như là các gói tin (packet), cho tầng mạng. Từ đó, các gói tin được xử lý như những
thông điệp riêng rẽ và không liên quan đến nhau cho đến khi chúng được truyền tới tầng vận
chuyển tại đích cuối cùng của chúng.

Có thể nói là các gói tin gắn liền với một thông điệp chung có khả năng đi theo những
đường khác nhau qua mạng Internet.

Tầng mạng có nhiệm vụ là chuyển tiếp những gói tin nó nhận được từ một mạng trên Internet
tới một mạng khác trong quá trình chúng được chuyển tới đích cuối cùng của chúng. Theo
cách đó thì tầng mạng phải làm việc với mô hình của mạng Internet. Nói một cách cụ thể,
nếu đường đi của một gói tin qua mạng Internet phải được truyền qua rất nhiều mạng riêng
lẻ, nó chính là tầng mạng tại mỗi điểm dừng trung gian mà xác định hướng đi  gói tin sẽ
được gửi đi sau đó. Việt quyết định đưa ra ở đây là như sau: Nếu đích cuối cùng của gói
tin là thuộc bên trong của mạng hiện thời, tầng mạng sẽ gửi gói tin đến ngay đó; ngược
lại, tầng mạng sẽ gửi gói tin tới một thiết bị dẫn đường (router) trong mạng hiện tại,
thiết bị dẫn đường này sẽ có nhiệm vụ chuyển tiếp gói tin đến mạng gần kề. Theo cách thức
này thì một gói tin được gửi tới đích là một máy tính trong mạng hiện tại sẽ được gửi ngay
tới máy tính đó, ngược lại thì một gói tin được gửi tới một máy tính nằm ngoài mạng hiện
tại sẽ tiếp tục hành trình của nó từ mạng này sang mạng tiếp theo gần kề cho đến mạng đích
cuối cùng.

Để quyết định đích tiếp theo trong hành trình của gói tin, tầng mạng cập nhật thêm địa chỉ
này vào gói tin như là một địa chỉ trung gian và chuyển giao gói tin này xuống tầng liên
kết.

Tầng liên kết có trách nhiệm truyền tải gói tin tới địa chỉ trung gian mà đã được xác định
bởi tầng mạng ở trên. Do đó, tầng liên kết phải thỏa thuận với sự truyền thông tới mạng
riêng biệt của máy tính. Nếu mạng đó là mạng vòng tròn có sử dụng thẻ bài, tầng liên kết
phải đợi có quyền chiếm hữu thẻ bài trước khi truyền gói tin đi. Nếu mạng sử dụng CSMA/CD,
tầng liên kết phải lắng nghe khi đường trục truyền rỗi mới được thực hiện truyền tải.

Khi một gói tin được truyền đi, nó được nhận bởi tầng liên kết tại máy tính được chỉ định
rõ bởi địa chỉ cục bộ đã được gắn thêm vào thông điệp. Tại đó, tầng liên kết sẽ chuyển
giao gói tin lên tầng mạng, nơi mà đích cuối cùng của gói tin được so sánh và kiểm tra với
địa chỉ hiện tại. Nếu những địa chỉ này không trùng khớp, tầng mạng xác định một địa chỉ
trung gian mới cho gói tin, đính kèm địa chỉ đó vào gói tin và đưa gói tin quay trở về
tầng liên kết để tiếp tục thực hiện truyền gói tin đi. Trong cách thức này, mỗi gói tin
được chuyển qua từng máy tính trên hành trình tới đích của nó. Cần chú ý rằng chỉ có tầng
liên kết và tầng mạng mới liên quan tại những điểm dừng trung gian trong suốt hành trình
này (xem lại Hình~\ref{fig:fig4.14}).

Nếu tầng mạng xác định rằng một gói tin đến đã tiến tới được đích cuối cùng, nó sẽ chuyển
giao gói tin đó cho tầng vận chuyển. Khi tầng vận chuyển nhận được các gói tin được gửi
tới từ tầng mạng, nó sẽ bóc tách những thông tin cần thiết bên trong của những đoạn tin và
xây dựng lại thông điệp gốc ban đầu dựa trên những số trình tự mà đã được cung cấp bởi
tầng vận chuyển tại nơi gửi của thông điệp. Khi thông điệp được ghép nối lại, tầng vận
chuyển thực hiện chuyển giao nó cho đơn vị phần mềm thích hợp trên tầng ứng dụng--kết thúc
quá trình truyền thông của thông điệp.


Việc xác định xem đơn vị phần mềm nào trong tầng ứng dụng được phép nhận một thông điệp
được gửi tới là nhiệm vụ quan trọng của tầng vận chuyển. Nhiệm vụ này được thực hiện thông
qua việc gán những \textbf{số cổng} duy nhất (không liên quan đến những cổng I/O đã được
thảo luận trong Chương~\ref{chap:2}) cho các đơn vị phần mềm khác nhau và yêu cầu số cổng thích
hợp được chèn thêm vào địa chỉ của thông điệp trước khi bắt đầu gửi thông điệp. Sau đó,
khi thông điệp được nhận bởi tầng vận chuyển tại đích nhận, tầng vận chuyển chỉ đơn thuần
chuyển giao thông điệp đó cho phần mềm trên tầng ứng dụng có cổng được chỉ định trước.

Người sử dụng của mạng Internet rất hiếm khi cần phải quan tâm tới những con số cổng này
bởi vì những ứng dụng thông thường thường có những cổng đã được công nhận một cách phổ
biến. Ví dụ, nếu một trình duyệt Web được yêu cầu tải một tài liệu mà URL của nó là
\url{http://www.zoo.org/animals/frog.html}, trình duyệt sẽ thừa nhận rằng nó cần phải liên
lạc với một phần mềm chủ HTTP tại địa chỉ \url{www.zoo.org} thông qua cổng 80. Tương tự
như vậy, khi truyền tải một tệp, một phần mềm khách FTP cũng thừa nhận rằng nó cần phải
giao tiếp với phần mềm chủ FTP thông qua cổng~20 và 21.

Nói tóm lại, việc truyền thông qua mạng Internet bao gồm sự tương tác giữa bốn tầng của
phần mềm. Tầng ứng dụng xử lý các thông điệp đứng trên góc độ tầm nhìn của ứng dụng. Tầng
vận chuyển chuyển đổi những thông điệp này thành những gói tin mà tương thích với mạng
Internet và tập hợp những thông điệp mà nhận được trước khi giao chúng cho ứng dụng thích
hợp. Tầng mạng liên quan đến việc gửi trực tiếp những gói tin qua mạng Internet. Tầng liên
kết đảm nhận việc truyền thông thực sự của những gói tin từ máy tính này tới máy tính
khác. Với tất cả các hoạt động này, có thể nói là hết sức kinh ngạc với thời gian trả lời
của mạng Internet được đo lường theo đơn vị phần nghìn giây (millisecond) thì rất nhiều
giao dịch xuất hiện ngay tức thời.

\subsection*{Bộ giao thức TCP/IP}

Yêu cầu đặt ra cho các hệ thống mạng mở đã phát sinh một nhu cầu cho những chuẩn được
công bố bởi các hãng sản xuất có thể cung cấp thiết bị và phần mềm mà hoạt động một cách
đúng đắn với những sản phẩm của các hãng khác. Một chuẩn mà kết quả là mô hình tham chiếu
Hệ thống mở kết nối liên mạng (OSI: Open System Interconnection), đã được đưa ra bởi Tổ
chức quốc tế về tiêu chuẩn hóa (International Organization Standardization). Chuẩn này
được dựa trên một hệ thống phân cấp bảy tầng trái với hệ thống phân cấp bốn tầng mà chúng
ta đã xem xét ở trên. Nó là một mô hình often-quoted vì nó mang theo uy quyền của một tổ
chức quốc tế, nhưng nó cũng làm trì hoãn việc thay thế quan điểm về mô hình bốn tầng, chủ
yếu là bởi vì nó được thiết lập sau hệ thống phân cấp bốn tầng vốn đã trở thành một chuẩn
trên thực tế cho mạng Internet.


Bộ giao thức TCP/IP là một tập các giao thức được sử dụng bởi mạng Internet nhằm triển
khai việc liên lạc theo hệ thống phân cấp trên mạng Internet. Trên thực tế, \textbf{Giao
  thức Điều khiển Việc truyền tải(TCP)} (TCP: Transmission Control Protocol) và
\textbf{Giao thức Mạng Internet (IP)} (IP: Internet Protocol) là tên của chỉ hai trong số
những giao thức trong tập giao thức rộng lớn này--thực tế là toàn bộ tập hợp này được hiểu
như là bộ giao thức TCP/IP. Một cách chính xác hơn, TCP định nghĩa một phiên bản của tầng
vận chuyển. Ta phát biểu là một phiên bản bởi vì bộ giao thức TCP/IP cung cấp hơn
một cách thức thực thi tại tầng vận chuyển; một phiên bản khác được định nghĩa bởi giao
thức UDP (User Datagram Protocol). Việc tách đôi như vậy cũng tương tự như trên thực tế
khi việc vận chuyển một gói hàng, bạn có một lựa chọn giữa nhiều công ty vận chuyển, mỗi
công ty lại đưa ra dịch vụ cơ bản là giống nhau nhưng cũng vẫn có những nét đặc trưng duy
nhất cho từng công ty. Do phụ thuộc vào chất lượng đặc biệt của dịch vụ đã yêu cầu, một
đơn vị trên tầng ứng dụng có thể lựa chọn gửi dữ liệu đi thông qua một trong hai phiên bản
của tầng vận chuyển, TCP hoặc UDP (Hình~\ref{fig:fig4.15})

\begin{figure} [tbh]
  \centering \scalebox{0.5}{\includegraphics{ch5/fig415.pdf}}
  \caption{Lựa chọn giữa TCP và UDP}
  \label{fig:fig4.15}
\end{figure}


Có hai điểm khác biệt cơ bản giữa TCP và UDP. Thứ nhất là trước khi gửi một thông điệp
được yêu cầu bởi tầng ứng dụng, tầng vận chuyển sử dụng giao thức TCP để gửi thông điệp
của chính nó tới tầng vận chuyển của bên đích nói rằng một thông điệp sắp được gửi sang
đó. Sau đó nó sẽ chờ thông điệp này được công nhận trước khi bắt đầu gửi thông điệp của
tầng ứng dụng. Theo cách đó, tầng vận chuyển sử dụng giao thức TCP sẽ thiết lập một kết
nối trước khi gửi một thông điệp. Còn tầng vận chuyển sử dụng giao thức UDP không thiết
lập một kết nối như vậy trước khi gửi một thông điệp. Tầng vận chuyển chỉ đơn thuần gửi
thông điệp nào đó tới địa chỉ mà nó nhận được và không quan tâm đến địa chỉ đó. Với tất cả
những gì mà nó biết được, máy tính đích có thể không sẵn sàng hoạt động. Với lý do này,
UDP được gọi là giao thức không có kết nối.

Sự khác biệt cơ bản thứ hai giữa TCP và UDP là tầng vận chuyển sử dụng giao thức TCP tại
nguồn gửi và đích nhận là cùng làm việc với nhau theo cách thức trả lời xác nhận và truyền
lại gói tin nhằm chắc chắn rằng tất cả các đoạn tin của thông điệp thực sự đã được truyền
tới đích thành công. TCP được gọi là giao thức tin cậy, trong khi đó thì UDP do không đưa
ra dịch vụ truyền lại nên được gọi là giao thức không tin cậy. Điều này không có nghĩa là
UDP là một lựa chọn tồi. Sau tất cả những điều trên, ta có thể kết luận tầng vận
chuyển sử dụng giao thức UDP được tổ chức hợp lý hơn so với tầng vận chuyển sử dụng giao
thức TCP, và do đó nếu một ứng dụng được chuẩn bị để có thể đạt được những kết quả tiềm
tàng của UDP, ý kiến đó có thể sẽ là một sự lựa chọn tốt hơn. Ví dụ, thư điện tử (email)
thông thường được gửi thông qua giao thức TCP nhưng sự truyền tải thực hiện thông qua hệ
thống máy chủ tên miền khi chuyển đổi những địa chỉ từ dạng dễ nhớ sang dạng IP lại sử
dụng giao thức UDP.

IP là một chuẩn của mạng Internet đối với tầng mạng. Một trong số những đặc tính của nó là
mỗi khoảng thời gian tầng mạng sử dụng giao thức IP lại chuẩn bị một gói tin để gửi xuống
tầng liên kết, nó sẽ thêm một giá trị được gọi là bộ đếm bước truyền, hay thời gian sống
(time to live), vào gói tin đó. Giá trị này là một giới hạn cho số lần gói tin được chuyển
tiếp khi nó cố thử tìm một đường đi qua mạng Internet. Mỗi khoảng thời gian tầng mạng sử
dụng giao thức IP chuyển tiếp một gói tin, nó sẽ giảm bộ đếm bước truyền xuống một giá
trị. Với thông tin này, tầng mạng có thể bảo vệ cho mạng Internet bởi sự lặp vòng vô hạn
của các gói tin trong hệ thống. Cho dù mạng Internet vẫn tiếp tục phát triển từ những thời
sơ khai của nó, một bộ đếm bước truyền ban đầu với giá trị là 64 là chưa đủ để cho phép
một gói tin có thể tìm được đường đi của nó khi được truyền xuyên suốt qua mê cung của
những mạng LAN, MAN, WAN, và các bộ dẫn đường.

Cho đến nay một phiên bản của giao thức IP được biết đến là IPv4 (IP phiên bản 4) đã được
sử dụng trong việc thực thi tầng mạng trong mạng Internet. Tuy nhiên, mạng Internet đã
nhanh chóng phát triển và vượt ra ngoài phạm vi hệ thống 32 bít địa chỉ của IPv4. Do đó,
một phiên bản mới của giao thức IP được biết đến là Ipv6, sử dụng địa chỉ liên mạng bao
gồm 128 bít, đã được thiết lập. Quá trình chuyển đổi từ IPv4 sang IPv6 hiện nay vẫn đang
thực hiện. (Đây là sự chuyển đổi mà ta đã được xem xét qua phần giới thiệu về địa
chỉ Internet trong Mục~\ref{sec:4.2}) Trong một vài phạm vi, IPv6 đang được sử dụng trên thực tế;
một vài phạm vi khác, sự chuyển đổi vẫn đang tiếp diễn trong một vài năm nữa. Ví dụ, theo
những kế hoạch hiện tại của chính phủ Hoa Kỳ là chuyển đổi sang IPv6 trước năm 2008. Trong
bất kỳ trường hợp nào, địa chỉ 32 bít trên mạng Internet vẫn được mong đợi là sẽ không còn
được sử dụng trước năm 2025.

\subsection*{Câu hỏi \& Bài tập}

\begin{enumerate}
\item Những tầng nào trong hệ thống phân cấp của phần mềm Internet được sử dụng để chuyển
  tiếp một thông điệp đến sang một máy tính khác?

\item Chỉ ra một vài sự khác biệt giữa tầng vận chuyển sử dụng giao thức TCP và tầng vận
  chuyển sử dụng giao thức UDP?

\item Làm thế nào để phần mềm Internet có thể đảm bảo rằng các thông điệp không bị chuyển
  tiếp trên mạng Internet mãi mãi?

\item Điều gì khiến cho một máy tính trên mạng Internet tránh được thao tác ghi lại các
  bản sao của những thông điệp truyền qua nó?

\end{enumerate}


%%% Local Variables: 
%%% mode: latex
%%% TeX-master: "../tindaicuong"
%%% End: 


\section{An ninh của máy tính}

Bởi vì các hệ điều hành giám sát các hoạt động của máy tính, nên nó cũng đóng vai trò
chính trong việc đảm bảo an ninh. Theo nghĩa rộng, nó có thể ở nhiều dạng, một trong số
chúng là độ tin cậy. Nếu sai sót trong trình quản lý file gây ra mất dữ liệu của file, vậy
file không an toàn. Nếu chương trình điều phối gây ra đổ vỡ hệ thống, làm mất dữ liệu ta
mất cả giờ để đánh, vậy công việc của ta không an toàn. Bởi vậy, an ninh của một hệ
thống tính toán đòi hỏi hệ điều hành phải được thiết kế tốt và đáng tin cậy.

Việc phát triển các phần mềm đáng tin cậy không phải là vấn đề nghiên cứu của hệ điều
hành. Nó thuộc phạm vi của Công nghệ phần mềm, ta sẽ xem xét sau trong
Chương~\ref{}. Trong phần này, ta chỉ quan tâm đế các vấn đề an ninh liên quan riêng đến
hệ điều hành.

\subsection*{Tấn công từ bên ngoài}
Một trong những nhiệm vụ quan trọng của hệ điều hành là bảo vệ tài nguyên của máy tính
tránh khỏi các truy cập bất hợp lệ. Trong trường hợp hệ thống có nhiều người sử dụng, việc
bảo vệ này dựa trên ``tài khoản'' (account)--một tài khoản được quản lý trong trong hệ
điều hành như một mục gồm tên người dùng, mật khẩu và quyền truy cập gắn với người
dùng. Hệ điều hành dùng các thông tin này trong mỗi lần đăng nhập (login) để điều khiển
việc truy cập vào hệ thống.

Các tài khoản được tạo bởi một người gọi là \textbf{super user} hay \textbf{người quản
  trị} (administrator). Người này có quyền cao nhất trong hệ thống, và cũng phải đăng nhập
vào hệ thống để xác thực anh/chị ta là người quản trị (thường bởi tên và mật khẩu). Khi đã
đăng nhập, người quản trị có thể làm nhiều thay đổi bên trong hệ thống như: thay đổi gói
phần mềm, gán quyền cho người dùng, thực hiện các hoạt động bảo trì hệ thống,...

Dùng ``quyền rất cao'' này, người quản trị phải điều khiển hoạt động trong hệ thống để
kiểm tra các hành vi phá hoại hệ thống, do vô tình hay cố ý. Cũng có nhiều phần mềm công
cụ trợ giúp người quản trị, được gọi là \textbf{phần mềm kiểm tra} (auditing software). Nó
ghi lại và phân tích các hoạt động xảy ra bên trong hệ thống. Ví dụ, phần mềm kiểm tra có
thể cho biết những lần đăng nhập sai mật khẩu. Phần mềm kiểm tra cũng phát hiện các hoạt
động của một tài khoản người dùng không phù hợp với các hành vi của anh ta trong quá khứ,
để từ đó chỉ ra những người dùng không có thẩm quyền đã giành được quyền truy cập vào tài
khoản này. (ví dụ, với một người dùng bình thường chỉ dùng gói phần mềm xử lý văn bản và
bảng tính, bây giờ lại dùng các ứng dụng phần mềm kỹ thuật cao hoặc thực hiện các gói công
cụ không hợp lệ với quyền của anh ta.)

Phần mềm kiểm tra cũng được thiết kế để phát hiện các \textbf{phần mềm sniffing}, là phần
mềm khi được phép chạy trên hệ thống sẽ tìm cách ghi lại các hoạt động và sau đó thông báo
lại cho kẻ thâm nhập (intruder). Một ví dụ tuy cũ nhưng được biết rộng rãi là một chương
trình một phỏng thủ tục đăng nhập của hệ điều hành. Các chương trình như thế này có thể
làm cho người dùng khác nhầm tưởng họ đang giao tiếp với hệ điều hành, và cung cấp tên và
mật khẩu cho kẻ mạo danh.

Với mọi sự phức tạp về mặt kỹ thuật được gắn với máy tính, thật đáng ngạc nhiên là rào cản
chính của an ninh của máy tính là do sự thiếu thận trọng của người dùng. Họ chọn các mật
khẩu rất dễ đoán (như tên và ngày sinh), họ chia sẻ mật khẩu của họ với bạn bè, họ không
thay đổi mật khẩu thường xuyên, họ đưa các thiết bị lưu trữ khối off-line của mình đến chỗ
hỏng hóc khi họ chuyển các thiết bị này giữa các máy, họ cài đặt các phần mềm có thể gây
mất an toàn vào hệ thống. Để giải quyết những vấn đề này, hầu hết các hệ thống máy tính
lớn đều bắt ép người dùng tuân theo một số yêu cầu về an toàn để nâng cao ý thức trách
nhiệm của họ.

\subsection*{Tấn công từ bên trong}
Khi một kẻ thâm nhập (có thể là người dùng hợp lệ nhưng có ý đồ xấu) tấn công vào hệ
thống, chúng thường tìm cách thăm dò, tìm các thông tin quan tâm, hoặc cài đặt vào hệ
thống các phần mềm phá hoại. Điều này rất đơn giản nếu kẻ rình mò có thể truy cập hệ thống
bằng tài khoản của người quản trị. Đây chính là lý do tại sao mà mật khẩu của người quản
trị phải được bảo vệ một cách nghiêm ngặt. Tuy nhiên, nếu truy cập được vào tài khoản
người dùng thông thường, kẻ thâm nhập phải tìm cách làm đánh lừa hệ điều hành để truy cập
vào các vùng bị cấm. Ví dụ, kẻ truy cập có thể đánh lừa trình quản lý bộ nhớ cho phép một
tiến trình truy cập ra ngoài vùng nhớ dành cho nó, hoặc kẻ truy cập có thể cố gắng đánh
lừa trình quản lý file để lấy các file mà nó không có quyền truy cập.


Các CPU hiện đại được thiết kế có thêm các đặc tính nhằm ngăn chặn những vấn đề này. Ví
dụ, có thể xét nhu cầu hạn chế một tiến trình chỉ được truy cập vào vùng bộ nhớ mà trình
quản lý bộ nhớ gán cho nó; nếu không hạn chế, một tiến trình có thể xoá hệ điều hành trong
bộ nhớ chính và chiếm quyền điều khiển máy tính. Để ngăn chặn vấn đề này, các CPU được
thiết kế cho hệ điều hành đa nhiệm có thể chứa các thanh ghi đặc biệt cho phép hệ điều
hành lưu giữ các giới hạn trên và dưới của vùng nhớ được gán cho tiến trình. Và trong khi
thực hiện xử lý, CPU so sánh mỗi vùng nhớ được tham chiếu đến với các thanh ghi này để đảm
bảo nó nằm trong giới hạn cho phép. Nếu vùng nhớ tham chiếu đến vượt ra ngoài giới hạn
này, CPU tự động chuyển quyền điều khiển tới hệ điều hành (bằng cách thực hiện một dãy các
ngắt) để hệ điều hành có các xử lý phù hợp.

Dù đặc điểm ta mô tả ở trên có vẻ rất tinh tế, nhưng trên thực tế nó vẫn có vấn đề. Nếu
CPU không có thêm một vài đặc tính an toàn nữa, một tiến trình vẫn có thể truy cập vào các
ô nhớ bất hợp lệ bằng cách thay đổi thanh ghi đặc biệt (chứa giới hạn bộ nhớ). Có nghĩa
rằng, một tiến trình có thể truy cập một bộ nhớ bên ngoài đơn thuần bằng cách thay đổi các
giá trị trong thanh ghi chứa giới hạn trên và dưới của bộ nhớ, và do đó nó có thể sử dụng
không gian bộ nhớ thêm mà không cần hệ điều hành cho phép.

Để tránh các hoạt động kiểu này, CPU được thiết kế để có thể thực hiện trong một hoặc hai
\textbf{mức đặc quyền} (privilege level); ta sẽ gọi là ``mode đặc quyền'' và ``mode
không đặc quyền.'' Khi ở trong mode đặc quyền, CPU có thể thực hiện mọi lệnh có trong ngôn
ngữ máy của nó. Tuy nhiên, khi ở trong mode không đặc quyền, các lệnh mà nó có thể thực
hiện sẽ bị giới hạn. Các lệnh chỉ được phép chạy ở mode đặc quyền gọi là \textbf{lệnh đặc
  quyền}. (ví dụ lệnh đặc quyền điển hình là lệnh làm thay đổi nội dung các thanh ghi giới
hạn bộ nhớ và các lệnh làm thay đổi mode đặc quyền của CPU.) Mọi nỗ lực thực hiện một lệnh
đặc quyền khi CPU ở mode không đặc quyền đều gây ra một ngắt. Ngắt này chuyển CPU tới mode
đặc quyền và chuyển quyền điều khiển tới trình xử lý ngắt của hệ điều hành.

Khi máy được bật, CPU ở mode đặc quyền. Bởi vậy, khi kết thúc quá trình khởi động và hệ
điều hành chiếm quyền điều khiển, lúc này mọi lệnh máy đều có thể được hiện. Tuy nhiên,
mỗi khi hệ điều hành cho phép một tiến trình chạy một time slide, nó chuyển CPU tới mode
không đặc quyền bằng cách thực hiện một lệnh ``chuyển mode đặc quyền''. Và từ lúc này, hệ
điều hành sẽ được thông báo nếu tiến trình cố gắng thực hiện lệnh ở mode đặc quyền.

Các lệnh đặc quyền và điều khiển các mức đặc quyền là các công cụ chính sẵn có để các hệ
điều hành quản lý an ninh. Tuy nhiên, việc sử dụng các công cụ này là một công việc hết
sức phức tạp trong thiết kế hệ điều hành. Một lỗi nhỏ trong điều khiển mức đặc quyền có
thể gây ra thảm hoạ do những người lập trình có ý đồ xấu hoặc do các lỗi vô ý gây ra khi
lập trình. Nếu một tiến trình được phép thay đổi thay đổi bộ định thời gian điều khiển
việc chia sẻ thời gian thực của hệ thống có thể cho phép một tiến trình mở rộng time slide
và chiếm quyền điều khiển máy. Nếu một tiến trình được phép truy cập trực tiếp vào thiết
bị ngoại vi, vậy nó có thể đọc các file mà không bị giám sát bởi trình quản lý file. Nếu
một tiến trình được phép truy cập vào các ô nhớ bên ngoài vùng cho phép, nó có thể đọc và
thậm chí thay đổi dữ liệu đang được sử dụng bởi tiến trình khác.
  
\subsection*{Câu hỏi \& Bài tập}
\begin{enumerate}
\item Hãy cho vài ví dụ về việc chọn mật khẩu kém an toàn và giải thích tại sao chúng lại
  kém?

\item Các bộ xử lý của Intel sử dụng bốn mức đặc quyền. Tại sao người thiết kế lại quyết
  định dùng bốn mà không phải là ba hay năm mức?

\item Nếu một tiến trình trong hệ thống chia sẻ thời gian thực có thể truy cập vào vùng
  nhớ không được phép, làm thế nào nó có thể chiếm quyền điều khiển máy?
\end{enumerate}
%%% Local Variables: 
%%% mode: latex
%%% TeX-master: "../tindaicuong"
%%% End: 

\newpage
\section{Bài tập cuối chương}
\begin{multicols}{2}
  \begin{enumerate}

  \item Giao thức là gì? Chỉ ra 3 giao thức đã giới thiệu trong chương và mô tả mục đích
    của mỗi giao thức.

  \item Xác định và mô tả một giao thức máy trạm/máy chủ được sử dụng trong cuộc sống hàng ngày.

  \item Mô tả mô hình máy trạm/máy chủ.

  \item Chỉ ra 2 cách thức phân loại mạng máy tính.

  \item Sự khác nhau giữa hệ thống mạng mở và hệ thống mạng đóng?

  \item Có những giao thức dựa trên thẻ bài có thể được sử dụng để điều khiển quyền truyền
    phát tín hiệu mà không theo mô hình mạng vòng tròn. Thiết kế một giao thức dựa trên
    thẻ bài nhằm điều khiển quyền truyền phát tín hiệu trong một mạng LAN với mô hình mạng
    hình tuyến.

  \item Mô tả các bước thực hiện bởi một máy tính muốn truyền một thông điệp trong hệ
    thống mạng sử dụng giao thức CSMA/CD.

  \item Hub khác biệt so với Repeater như thế nào?

  \item Chỉ ra sự khác biệt khi so sánh Router với các thiết bị như Repeater, Bridge, Switch.

  \item Phân biệt một mạng nói chung với một mạng Internet nói riêng.

  \item Chỉ ra 2 giao thức điều khiển việc truyền tải một thông điệp trong một hệ thống mạng.

  \item 1.Mã hóa mỗi chuỗi bit sau đây bằng cách sử dụng ký hiệu dấu chấm thập phân:
    \begin{enumerate}
    \item $000000010000001000000011$

    \item $1000000000000000$

    \item $0001100000001100$
    \end{enumerate}

  \item Chuỗi bit tương ứng với mỗi mẫu ký hiệu dấu thập phân sau:
    \begin{enumerate}
    \item $0.0$ 
    \item $25.18.1$
    \item $5.12.13.10$
    \end{enumerate}

  \item Giả sử địa chỉ của một máy chủ trên mạng Internet là $138.48.4.123$. Địa chỉ tương
    ứng ở dạng Hexa (hệ cơ số mười sáu) là gì?

  \item Nếu phần xác định địa chỉ mạng của một vùng là $192.207.77$, có bao nhiêu địa chỉ
    IP có thể sử dụng được để cấu hình cho các máy tính trong vùng? (Sau khi bạn tìm ra
    được câu trả lời, bạn có thể phỏng đoán rằng có thể có ít hơn địa chỉ IP so với số
    lượng máy tính có trong vùng, và đây cũng là trường hợp thường xảy ra. Một giải pháp
    để có thể đặt địa chỉ IP cho các máy tính chỉ khi máy cần dùng đến, đó là hệ thống đặt
    địa chỉ IP động).


  \item Nếu một địa chỉ theo dạng tên dễ nhớ của một máy tính trên mạng Internet có dạng:\\
    \url{batman.batcave.metropolis.gov}\\ Bạn có thể phỏng đoán vùng chứa máy tính đó là
    gì?


  \item Giải thích các thành phần xuất hiện trong địa chỉ\\ \url{kermit@animals.com}

  \item Trong trường hợp truyền tải tệp sử dụng giao thức FTP, sự khác biệt rõ nét giữa
    ``tệp văn bản'' và ``tệp nhị phân'' là gì?

  \item Vai trò của máy chủ thư trong một vùng là gì?

  \item Định nghĩa lại mỗi khái niệm sau:
    \begin{enumerate}
    \item Name server
    \item Domain
    \item Router
    \item Host
    \end{enumerate}

  \item Vai trò của Network Virtual Terminal trong giao thức telnet?

  \item Định nghĩa lại mỗi khái niệm sau:
    \begin{enumerate}
    \item Hypertext
    \item HTML
    \item Browser
    \end{enumerate}


  \item Có nhiều cách nhìn nhận về sự hoán đổi giữa hai thuật ngữ \textit{Internet} và
    \textit{World Wide Web} trong mạng Internet. Mỗi thuật ngữ trên đề cập tới vấn đề gì?

  \item Khi thực hiện xem một trang web đơn giản, yêu cầu trình duyệt hiển thị nguồn của
    trang web đó. Sau đó xác định cấu trúc cơ bản của tài liệu nguồn hiện ra, xác định
    phần tiêu đề và phần thân của tài liệu đồng thời chỉ ra một vài câu lệnh tìm thấy
    trong mỗi phần.


  \item Sửa tài liệu HTML duới đây với yêu cầu là từ ``Rover'' được liên kết tới tài liệu
    khác theo đường dẫn sau: \url{http://animals.org/pets/dogs.html}
\begin{verbatim}
<html>
<head>
<title>Example</title>
</head>
<body>
<h1>My Pet Dog</h1>
<p>My dog’s name is Rover./p>
</body
<html>
\end{verbatim}

  \item Vẽ ra một bản phác họa mô tả xem một tài liệu HTML sau đây sẽ xuất hiện như thế
    nào khi nó hiển lên màn hình máy tính.
\begin{verbatim}
<html>
<head>
<title>Example</title>
</head>
<body>
<h1>My Pet Dog</h1>
<p>My dog’s name is Rover./p>
</body
<html>
\end{verbatim}

  \item Xác định các phần tử cấu thành trong địa chỉ sau và nêu ý nghĩa của chúng:\\
    \url{http://lifeforms.com/animals/moviestars/kermit.html}

  \item Xác định các phần tử cấu thành trong các địa chỉ vắn tắt sau:

    \begin{enumerate}
    \item \url{http://www.farmtools.org/windmills.html} 

    \item \url{http://castles.org/} 

    \item \url{www.coolstuff.com}
    \end{enumerate}

  \item Trình duyệt sẽ đáp ứng lại khác nhau như thế nào khi bạn yêu cầu nó tìm một tài
    liệu qua địa chỉ:\\
    \url{telnet://stargazer.universe.org} \\
    so với \\
    \url{http://stargazer.universe.org }

  \item Đưa ra $2$ ví dụ về các hoạt động ở phía máy trạm và $2$ ví dụ về các hoạt động ở
    phía máy chủ.

  \item Giả sử mỗi máy tính trong một mạng vòng tròn được lập trình để truyền tức thì về
    cả hai phía các thông điệp mà xuất phát từ một máy trạm nào đó và được đánh địa chỉ
    gửi tới tất cả các trạm làm việc khác trong mạng. Hơn nữa, giả sử điều này được thực
    hiện thông qua việc giành được quyền truy cập đầu tiên tới đường truyền truy cập tới
    các máy phía bên trái, duy trì truy cập này cho tới khi đường truyền truy cập phía bên
    phải được yêu cầu và sau đó truyền tải thông điệp đi. Xác định khi nào
    \textup{deadlock} (xem thêm Mục~\ref{sec:3.4}) xảy ra nếu tất cả các máy tính trong
    mạng đều cố gắng gửi một thông điệp cùng tại một thời điểm.


  \item Mô hình tham chiếu OSI là gì?

  \item Trong một hệ thống mạng được thiết kế theo dạng hình tuyến, trục bus là một dạng
    tài nguyên không thể chia sẻ được mà trong đó các máy trạm đều cần phải cạnh tranh với
    nhau để có thể gửi được các thông điệp một cách có thứ tự. Trong trường hợp này, tắc
    nghẽn (deadlock) (xem thêm Mục~\ref{sec:3.4}) được điều khiển như thế nào?

  \item Liệt kê 4 tầng trong mô hình Internet và xác định nhiệm vụ được thực hiện bởi mỗi
    tầng.

  \item Tại sao tầng vận chuyển lại thực hiện chia nhỏ các thông điệp (messages) lớn thành
    những gói tin nhỏ (packets).

  \item Khi một ứng dụng yêu cầu tầng vận chuyển sử dụng giao thức TCP để truyền tải một
    thông điệp, những thông điệp nào khác sẽ được truyền tải bởi tầng này nhằm đáp ứng đủ
    các yêu cầu của tầng ứng dụng?

  \item Theo cách thức nào mà có thể nhận xét giao thức TCP được xem là tốt hơn so với
    giao thức UDP trong việc thực thi tại tầng vận chuyển? Trong trường hợp nào thì giao
    thức UDP được coi là tốt hơn so với giao thức TCP?

  \item Điều gì khẳng định nhận xét UDP là một giao thức không có kết nối
    (connectionless)?

  \item Tại tầng nào trong mô hình phân cấp giao thức TCP/IP ta có thể đặt một bức tường
    tại đó nhằm lọc các luồng giao thông đi tới theo:
    \begin{enumerate}
    \item Nội dung của thông điệp

    \item Địa chỉ nguồn của thông điệp

    \item Kiểu của ứng dụng
    \end{enumerate}

  \item Giả sử bạn muốn thiết lập một bức tường lửa nhằm lọc các thông điệp thư điện tử
    chứa các câu hay cụm từ chỉ định. Bức tường lửa này nên được đặt tại cổng vào ra của
    vùng hay đặt tại máy chủ thư của vùng? Giải thích câu trả lời của bạn.

  \item Máy chủ proxy là gì và lợi ích của nó đem lại?

  \item Tóm tắt các nguyên lý của mật mã hóa khóa công khai.

  \item Mạng toàn cầu Internet thường gây nguy hại cho một máy tính không được bảo vệ theo
    cách thức nào?

  \end{enumerate}
\end{multicols}

%%% Local Variables: 
%%% mode: latex
%%% TeX-master: "../tindaicuong"
%%% End: 


%%% Local Variables: 
%%% mode: latex
%%% TeX-master: "../tindaicuong"
%%% End: 
