\newpage
\section{Bài tập cuối chương}
\begin{multicols}{2}
  \begin{enumerate}

  \item Giao thức là gì? Chỉ ra 3 giao thức đã giới thiệu trong chương và mô tả mục đích
    của mỗi giao thức.

  \item Xác định và mô tả một giao thức máy trạm/máy chủ được sử dụng trong cuộc sống hàng ngày.

  \item Mô tả mô hình máy trạm/máy chủ.

  \item Chỉ ra 2 cách thức phân loại mạng máy tính.

  \item Sự khác nhau giữa hệ thống mạng mở và hệ thống mạng đóng?

  \item Có những giao thức dựa trên thẻ bài có thể được sử dụng để điều khiển quyền truyền
    phát tín hiệu mà không theo mô hình mạng vòng tròn. Thiết kế một giao thức dựa trên
    thẻ bài nhằm điều khiển quyền truyền phát tín hiệu trong một mạng LAN với mô hình mạng
    hình tuyến.

  \item Mô tả các bước thực hiện bởi một máy tính muốn truyền một thông điệp trong hệ
    thống mạng sử dụng giao thức CSMA/CD.

  \item Hub khác biệt so với Repeater như thế nào?

  \item Chỉ ra sự khác biệt khi so sánh Router với các thiết bị như Repeater, Bridge, Switch.

  \item Phân biệt một mạng nói chung với một mạng Internet nói riêng.

  \item Chỉ ra 2 giao thức điều khiển việc truyền tải một thông điệp trong một hệ thống mạng.

  \item 1.Mã hóa mỗi chuỗi bit sau đây bằng cách sử dụng ký hiệu dấu chấm thập phân:
    \begin{enumerate}
    \item $000000010000001000000011$

    \item $1000000000000000$

    \item $0001100000001100$
    \end{enumerate}

  \item Chuỗi bit tương ứng với mỗi mẫu ký hiệu dấu thập phân sau:
    \begin{enumerate}
    \item $0.0$ 
    \item $25.18.1$
    \item $5.12.13.10$
    \end{enumerate}

  \item Giả sử địa chỉ của một máy chủ trên mạng Internet là $138.48.4.123$. Địa chỉ tương
    ứng ở dạng Hexa (hệ cơ số mười sáu) là gì?

  \item Nếu phần xác định địa chỉ mạng của một vùng là $192.207.77$, có bao nhiêu địa chỉ
    IP có thể sử dụng được để cấu hình cho các máy tính trong vùng? (Sau khi bạn tìm ra
    được câu trả lời, bạn có thể phỏng đoán rằng có thể có ít hơn địa chỉ IP so với số
    lượng máy tính có trong vùng, và đây cũng là trường hợp thường xảy ra. Một giải pháp
    để có thể đặt địa chỉ IP cho các máy tính chỉ khi máy cần dùng đến, đó là hệ thống đặt
    địa chỉ IP động).


  \item Nếu một địa chỉ theo dạng tên dễ nhớ của một máy tính trên mạng Internet có dạng:\\
    \url{batman.batcave.metropolis.gov}\\ Bạn có thể phỏng đoán vùng chứa máy tính đó là
    gì?


  \item Giải thích các thành phần xuất hiện trong địa chỉ\\ \url{kermit@animals.com}

  \item Trong trường hợp truyền tải tệp sử dụng giao thức FTP, sự khác biệt rõ nét giữa
    ``tệp văn bản'' và ``tệp nhị phân'' là gì?

  \item Vai trò của máy chủ thư trong một vùng là gì?

  \item Định nghĩa lại mỗi khái niệm sau:
    \begin{enumerate}
    \item Name server
    \item Domain
    \item Router
    \item Host
    \end{enumerate}

  \item Vai trò của Network Virtual Terminal trong giao thức telnet?

  \item Định nghĩa lại mỗi khái niệm sau:
    \begin{enumerate}
    \item Hypertext
    \item HTML
    \item Browser
    \end{enumerate}


  \item Có nhiều cách nhìn nhận về sự hoán đổi giữa hai thuật ngữ \textit{Internet} và
    \textit{World Wide Web} trong mạng Internet. Mỗi thuật ngữ trên đề cập tới vấn đề gì?

  \item Khi thực hiện xem một trang web đơn giản, yêu cầu trình duyệt hiển thị nguồn của
    trang web đó. Sau đó xác định cấu trúc cơ bản của tài liệu nguồn hiện ra, xác định
    phần tiêu đề và phần thân của tài liệu đồng thời chỉ ra một vài câu lệnh tìm thấy
    trong mỗi phần.


  \item Sửa tài liệu HTML duới đây với yêu cầu là từ ``Rover'' được liên kết tới tài liệu
    khác theo đường dẫn sau: \url{http://animals.org/pets/dogs.html}
\begin{verbatim}
<html>
<head>
<title>Example</title>
</head>
<body>
<h1>My Pet Dog</h1>
<p>My dog’s name is Rover./p>
</body
<html>
\end{verbatim}

  \item Vẽ ra một bản phác họa mô tả xem một tài liệu HTML sau đây sẽ xuất hiện như thế
    nào khi nó hiển lên màn hình máy tính.
\begin{verbatim}
<html>
<head>
<title>Example</title>
</head>
<body>
<h1>My Pet Dog</h1>
<p>My dog’s name is Rover./p>
</body
<html>
\end{verbatim}

  \item Xác định các phần tử cấu thành trong địa chỉ sau và nêu ý nghĩa của chúng:\\
    \url{http://lifeforms.com/animals/moviestars/kermit.html}

  \item Xác định các phần tử cấu thành trong các địa chỉ vắn tắt sau:

    \begin{enumerate}
    \item \url{http://www.farmtools.org/windmills.html} 

    \item \url{http://castles.org/} 

    \item \url{www.coolstuff.com}
    \end{enumerate}

  \item Trình duyệt sẽ đáp ứng lại khác nhau như thế nào khi bạn yêu cầu nó tìm một tài
    liệu qua địa chỉ:\\
    \url{telnet://stargazer.universe.org} \\
    so với \\
    \url{http://stargazer.universe.org }

  \item Đưa ra $2$ ví dụ về các hoạt động ở phía máy trạm và $2$ ví dụ về các hoạt động ở
    phía máy chủ.

  \item Giả sử mỗi máy tính trong một mạng vòng tròn được lập trình để truyền tức thì về
    cả hai phía các thông điệp mà xuất phát từ một máy trạm nào đó và được đánh địa chỉ
    gửi tới tất cả các trạm làm việc khác trong mạng. Hơn nữa, giả sử điều này được thực
    hiện thông qua việc giành được quyền truy cập đầu tiên tới đường truyền truy cập tới
    các máy phía bên trái, duy trì truy cập này cho tới khi đường truyền truy cập phía bên
    phải được yêu cầu và sau đó truyền tải thông điệp đi. Xác định khi nào
    \textup{deadlock} (xem thêm Mục~\ref{sec:3.4}) xảy ra nếu tất cả các máy tính trong
    mạng đều cố gắng gửi một thông điệp cùng tại một thời điểm.


  \item Mô hình tham chiếu OSI là gì?

  \item Trong một hệ thống mạng được thiết kế theo dạng hình tuyến, trục bus là một dạng
    tài nguyên không thể chia sẻ được mà trong đó các máy trạm đều cần phải cạnh tranh với
    nhau để có thể gửi được các thông điệp một cách có thứ tự. Trong trường hợp này, tắc
    nghẽn (deadlock) (xem thêm Mục~\ref{sec:3.4}) được điều khiển như thế nào?

  \item Liệt kê 4 tầng trong mô hình Internet và xác định nhiệm vụ được thực hiện bởi mỗi
    tầng.

  \item Tại sao tầng vận chuyển lại thực hiện chia nhỏ các thông điệp (messages) lớn thành
    những gói tin nhỏ (packets).

  \item Khi một ứng dụng yêu cầu tầng vận chuyển sử dụng giao thức TCP để truyền tải một
    thông điệp, những thông điệp nào khác sẽ được truyền tải bởi tầng này nhằm đáp ứng đủ
    các yêu cầu của tầng ứng dụng?

  \item Theo cách thức nào mà có thể nhận xét giao thức TCP được xem là tốt hơn so với
    giao thức UDP trong việc thực thi tại tầng vận chuyển? Trong trường hợp nào thì giao
    thức UDP được coi là tốt hơn so với giao thức TCP?

  \item Điều gì khẳng định nhận xét UDP là một giao thức không có kết nối
    (connectionless)?

  \item Tại tầng nào trong mô hình phân cấp giao thức TCP/IP ta có thể đặt một bức tường
    tại đó nhằm lọc các luồng giao thông đi tới theo:
    \begin{enumerate}
    \item Nội dung của thông điệp

    \item Địa chỉ nguồn của thông điệp

    \item Kiểu của ứng dụng
    \end{enumerate}

  \item Giả sử bạn muốn thiết lập một bức tường lửa nhằm lọc các thông điệp thư điện tử
    chứa các câu hay cụm từ chỉ định. Bức tường lửa này nên được đặt tại cổng vào ra của
    vùng hay đặt tại máy chủ thư của vùng? Giải thích câu trả lời của bạn.

  \item Máy chủ proxy là gì và lợi ích của nó đem lại?

  \item Tóm tắt các nguyên lý của mật mã hóa khóa công khai.

  \item Mạng toàn cầu Internet thường gây nguy hại cho một máy tính không được bảo vệ theo
    cách thức nào?

  \end{enumerate}
\end{multicols}

%%% Local Variables: 
%%% mode: latex
%%% TeX-master: "../tindaicuong"
%%% End: 
