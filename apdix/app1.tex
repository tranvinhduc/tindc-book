
\chapter {Một ngôn ngữ máy đơn giản}
\label{phuluc1}

Trong Phụ lục này ta giới thiệu một ngôn ngữ máy đơn giản, mục
đích để giải thích các khái niệm của kiến trúc máy một cách trực
quan. Ta bắt đầu với kiến trúc của máy này.

\subsection*{Kiến trúc máy}
Máy có $16$ thanh ghi đa năng đánh số từ $0$ đến $F$ (ở dạng
hexa). Mỗi thanh ghi có độ dài là một byte (tám bít). Để xác định
thanh ghi bên trong lệnh, mỗi thanh ghi được gán với một xâu bốn
bít. Ví dụ, thanh ghi $0$ xác định bởi $0000$ (ở dạng hexa là $0$),
thanh ghi $4$ xác định bởi $0100$ ($4$ ở dạng hexa).

Bộ nhớ chính gồm $256$ ô nhớ. Mỗi ô nhớ được gán một địa chỉ duy nhất
là một số nguyên trong khoảng từ $0$ tới $255$. Từ đó, một địa chỉ có
thể biểu diễn bởi tám bít, phạm vi từ $00000000$ đến $11111111$ (ở
dạng hexa là từ $00$ đến $FF$).

Các giá trị dấu chấm động được lưu trữ ở dạng tám bít như được thảo
luận ở Mục~\ref{} và được tóm tắt trong Hình~\ref{}.

\subsection*{Ngôn ngữ máy}
Mỗi lệnh máy có độ dài hai byte. Bốn bít đầu tiên là trường op-code; $12$ bít sau dành cho
phần toán hạng. Bảng dưới đây liệt kê các lệnh (ở dạng hexa) với mô tả ngắn gọn các
lệnh. Các chữ cái R, S và T ở dạng hexa để biểu diễn các thanh ghi. Các chữ cái X và Y sử
dụng cho các số hexa trong các trường không phải để biểu diễn thanh ghi.  
\begin{longtable}{ccm{10cm}} 
  \textbf{Op-code} & \textbf{Toán hạng} & \textbf{Mô tả} \\ \\
 
  $1$ & RXY & LOAD (nạp) thanh ghi R với giá trị nằm trong ô nhớ tại
  địa chỉ XY.                                      \\
  & & \textit{Ví dụ:} $14A3$ đặt nội dung của thanh ghi $4$ bằng với
  nội dung của ô nhớ tại
  địa chỉ $A3$. \\\\
   
  $2$ & RXY & LOAD (nạp) thanh ghi R bằng với xâu bít XY. \\
  & & \textit{Ví dụ:} $20A3$ đặt giá trị $A3$ vào thanh ghi $0$. \\ \\

  $3$ & RXY & STORE (ghi) nội dung thanh ghi $R$ vào trong bộ nhớ tại địa chỉ XY. \\
  & & \textit{Ví dụ:} $35B1$ ghi nội dung thanh ghi $5$ vào ô nhớ tại địa chỉ $B1$ \\ \\

  $4$ & 0RS & MOVE (sao chép) xâu bít trong thanh ghi $R$ vào thanh ghi~$S$.\\
  & & \textit{Ví dụ:} $40A4$ sao chép nội dung thanh ghi  $A$ vào thanh ghi $4$ \\ \\

  $5$ & RST & ADD (cộng) nội dung trong thanh ghi S và T ở dạng bù hai
  và đặt kết quả vào
  thanh ghi R. \\
  & & \textit{Ví dụ:} $5726$ cộng giá trị trong thanh ghi $2$ với giá
  trị trong thanh ghi
  $6$ và kết quả đặt vào thanh ghi $7$. \\ \\


  $6$ & RST & ADD (cộng) nội dung trong thanh ghi S và T ở dạng dấu
  chấm động và đặt kết quả vào
  thanh ghi R. \\
  & & \textit{Ví dụ:} $634E$ cộng giá trị trong thanh ghi $4$ với giá
  trị trong thanh ghi
  $E$ dạng dấu chấm động và kết quả đặt vào thanh ghi $3$. \\ \\

  $7$ & RST & OR (tuyển) các xâu bít trong thanh ghi S và T và đặt kết
  quả vào
  thanh ghi R. \\
  & & \textit{Ví dụ:} $7CB4$ OR giá trị trong thanh ghi $B$ với giá
  trị trong thanh ghi
  $4$ dạng dấu chấm động và kết quả đặt vào thanh ghi $C$. \\ \\

  $8$ & RST & AND (hội) các xâu bít trong thanh ghi S và T và đặt kết
  quả vào
  thanh ghi R. \\
  & & \textit{Ví dụ:} $8045$ AND giá trị trong thanh ghi $4$ với giá
  trị trong thanh ghi
  $5$ dạng dấu chấm động và kết quả đặt vào thanh ghi $0$. \\ \\

  $9$ & RST & XOR (tuyển loại) các xâu bít trong thanh ghi S và T và
  đặt kết quả vào
  thanh ghi R. \\
  & & \textit{Ví dụ:} $95F3$ XOR giá trị trong thanh ghi $F$ với giá
  trị trong thanh ghi
  $3$ dạng dấu chấm động và kết quả đặt vào thanh ghi $5$. \\ \\

  $A$ & R0X & ROTATE (quay phải) một bít xâu bít trong thanh ghi R đi
  X lần. Mỗi lần đặt
  bít thấp nhất lại vào vị trí cao nhất.\\
  & & \textit{Ví dụ:} $A403$ quay nội dung thanh ghi $4$ đi sang phải
  $3$ bít theo cách
  vòng. \\ \\

  $B$ & RXY & JUMP (nhảy) tới lệnh tại vị trí XY trong bộ nhớ nếu nội
  dung của thanh ghi R bằng với nội dung thanh ghi $0$. Ngược lại,
  tiếp tục thực hiện lệnh tiếp theo như bình thường. (Lệnh nhảy được
  cài đặt bằng cách sao chép XY vào bộ đếm chương trình trong
  bước thực hiện.) \\
  & & \textit{Ví dụ:} $B43C$ đầu tiên nó so sánh nội dung của thanh
  ghi $4$ với nội dung của thanh ghi $0$. Nếu bằng nhau, xâu bít $3C$
  được đặt trong bộ đếm chương trình sao cho lệnh tại địa chỉ này sẽ
  là lệnh tiếp theo được thực hiện. Ngược lại, nó không làm gì
  và chương trình tiếp tục thực hiện theo thứ tự bình thường. \\ \\

  $C$ & $000$ & HALT (dừng) việc thực hiện chương trình.\\
  & & \textit{Ví dụ:} $C000$  làm dừng chương trình đang thực hiện. \\ \\
\end{longtable}
    

%%% Local Variables: 
%%% mode: latex
%%% TeX-master: "../tindaicuong"
%%% End: 
