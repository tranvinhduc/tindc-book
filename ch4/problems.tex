\section{Bài tập cuối chương}
  

\begin{multicols}{2}
  \begin{enumerate}
  \item Liệt kê bốn hoạt động của một hệ điều hành điển hình.

  \item Tóm tắt sự khác nhau giữa xử lý theo lô và xử lý tương tác.

  \item Ta đặt (theo thứ tự) ba phần tử $R, S$ và $T$ vào trong một
    hàng đợi. Đầu tiên, ta lấy hai phần tử ra khỏi hàng đợi và đặt
    thêm một phần tử $X$ vào hàng đợi. Sau đó, ta lại lấy tiếp hai
    phần tử ra khỏi hàng đợi, và lại đặt thêm hai phần tử $Y, Z$ (theo
    thứ tự) vào hàng đợi, và sau đó lấy hết các phần tử cho đến khi
    hàng đợi rỗng. Hãy liệt kê các phần tử trong hàng đợi theo thứ tự
    mà chúng bị lấy ra.

  \item Chỉ ra sự khác biệt giữa xử lý tương tác và xử lý thời gian
    thực.

  \item Hệ điều hành đa nhiệm là gì?

  \item Giả sử bạn có một máy PC, hãy chỉ ra một vài tình huống bạn
    thấy rõ lợi thế của khả năng đa nhiệm của nó.

  \item Hãy chỉ ra hai phần mềm ứng dụng và hai phần mềm công cụ mà
    bạn quen thuộc.

  \item Cấu trúc thư mục được mô tả bởi đường dẫn X/Y/Z là gì?

  \item Bảng mô tả tiến trình chứa những thông tin gì?

  \item Chỉ ra sự khác nhau giữa tiến trình sẵn sàng và tiến trình
    đang đợi.

  \item Nêu sự khác nhau giữa bộ nhớ ảo và bộ nhớ chính.

  \item Giả sử máy tính của bạn có $512$MB (MiB) bộ nhớ chính, và một
    hệ điều hành cần tạo một bộ nhớ ảo gấp hai lần kích thước bộ nhớ
    chính với kích thước các trang được dùng là $2$KB (KiB). Có bao
    nhiêu trang có thể được yêu cầu.

  \item Vấn đề phức tạp gì xảy ra trong hệ thống chia sẻ thời gian
    thực nếu hai tiến trình yêu cầu truy cập vào cùng một file tại
    cùng một thời điểm? Có trường hợp nào mà trình quản lý file nên
    cho phép các yêu cầu kiểu này? Có trường hợp nào mà trình quản lý
    file nên cấm các yêu cầu kiểu này?

  \item Định nghĩa cân bằng tải và tỷ xích (scaling) trong ngữ cảnh của kiến
    trúc đa bộ xử lý.

  \item Tóm tắt quá trình khởi động máy.

  \item Giả sử bạn có một máy PC, hãy ghi lại dãy các hoạt động mà bạn
    quan sát được khi bật máy. Sau đó hãy xác định các thông điệp được
    hiện lên màn hình máy tính trước khi quá tình khởi động thực sự
    bắt đầu. Phần mềm gì viết các thông điệp này?

  \item Giả sử rằng hệ điều hành chia sẻ thời gian thực cấp phát các time slide~$20$mili
    giây và máy thực hiện trung bình $5$ lệnh trong một micro giây. Vậy máy có thể thực
    hiện bao nhiêu lệnh trong một time slide?

  \item Nếu một người đánh được $60$ từ trong một phút (một từ được xem là gồm $5$ ký tự),
    vậy mỗi ký tự người đó đánh mất bao lâu?  nếu người đó dùng một hệ điều hành chia sẻ
    thời gian thực cấp phát time slide theo đơn vị~$20$ mili giây và chúng ta bỏ qua việc
    chuyển đổi giữa các tiến trình, vậy có bao nhiêu time-slide có thể được cấp phát giữa
    lúc hai ký tự được đánh?


  \item Giả sử một hệ điều hành chia sẻ thời gian thực chia time slide là~$50$ milli
    giây. Nếu bình thường đầu đọc/ghi đĩa mất $8$ milli giây để chuyển tới track mong muốn
    và mất thêm $17$ mili giây để tới dữ liệu mong muốn, vậy một chương trình mất bao
    nhiêu time slide để đợi thao tác đọc đĩa hoàn thành? Nếu máy có khả năng thực hiện
    mười lệnh mỗi micro giây, bao nhiêu lệnh có thể thực hiện trong khi đợi chu kỳ này?
    (Đây là lý do tại sao khi một tiến trình thực hiện một thao tác với thiết bị ngoại vi,
    hệ thống chia sẻ thời gian thực kết thúc time slide của tiến trình này và cho phép
    tiến trình khác chạy trong khi tiến trình ban đầu đợi phục vụ của thiết bị ngoại vi.)


  \item Liệt kê năm nguồn tài nguyên mà hệ điều hành đa nhiệm phải
    điều phối việc truy cập.


  \item Một tiến trình được gọi là I/O-bound nếu nó yêu cầu nhiều phép
    toán vào/ra, còn một tiến trình được gọi là compute-bound nếu hầu
    hết thời gian thực hiện nó dành cho việc tính toán. Giả sử có hai
    tiến trình, một là I/O-bound và một là compute-bound, đang cùng
    đợi một time-slide, vậy ta nên ưu tiên tiến trình nào? Tại sao?

  \item Trong một hệ thống chia sẻ thời gian thì hiệu suất chạy hai
    tiến trình I/O bound tốt hơn hay chạy một tiến trình I/O bound và
    một compute-bound tốt hơn? Tại sao?

  \item Viết các chỉ thị mà bộ điều phối của hệ điều hành phải làm khi
    một tiến trình hết time slide.

  \item Trạng thái của tiến trình gồm những thành phần gì?

  \item Chỉ ra một tình huống mà một tiến trình trong hệ thống chia sẻ
    thời gian thực không dùng hết time slide được cấp cho nó.

  \item Liệt kê theo thứ tự thời gian các sự kiện xuất hiện khi một
    tiến trình bị ngắt.

  \item Trả lời các câu hỏi sau đây theo hệ điều hành bạn đang dùng:
    \begin{enumerate}[a.]
    \item Làm thế nào để yêu cầu hệ điều hành copy một file từ vị
      trí này tới vị trí khác?

    \item Làm thế nào để xem các thư mục trên đĩa?

    \item Làm thế nào để yêu cầu hệ điều hành thực hiện một chương trình?
    \end{enumerate}

  \item Trả lời các câu hỏi sau đây theo hệ điều hành bạn đang dùng:
    \begin{enumerate}[a.]
    \item Làm thế nào hệ điều hành hạn chế truy cập chỉ cho những
      người được phép?

    \item Làm thế nào để yêu cầu hệ điều hành chỉ ra các tiến trình
      hiện đang có trong bảng tiến trình?

    \item Làm thế nào để bảo hệ điều hành rằng bạn không muốn người
      dùng khác truy cập vào file của bạn?
    \end{enumerate}

  \item Làm thế nào một hệ điều hành giữ không cho một tiến trình truy
    cập vào không gian bộ nhớ của tiến trình khác?

  \item Giả sử một mật khẩu bao gồm một xâu chín ký tự trong bảng chữ
    cái Tiếng Anh ($26$ ký tự). Nếu mỗi mật khẩu có thể được kiểm tra
    trong một milli giây, vậy mất bao lâu có thể kiểm tra mọi mật khẩu
    có thể?

  \item Tại sao các CPU thiết kế cho hệ điều hành đa nhiệm lại cần
    phân chia các thao tác theo các mức đặc quyền khác nhau?

  \item Hãy chỉ ra hai hoạt động yêu cầu các lệnh đặc quyền?

  \item Hãy chỉ ra ba cách mà một tiến trình có thể gây mất an toàn
    cho hệ thống máy tính nếu hệ điều hành không ngăn chặn.
  \end{enumerate}
\end{multicols}


%%% Local Variables: 
%%% mode: latex
%%% TeX-master: "../tindaicuong"
%%% End: 
