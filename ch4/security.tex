
\section{An ninh của máy tính}

Bởi vì các hệ điều hành giám sát các hoạt động của máy tính, nên nó cũng đóng vai trò
chính trong việc đảm bảo an ninh. Theo nghĩa rộng, nó có thể ở nhiều dạng, một trong số
chúng là độ tin cậy. Nếu sai sót trong trình quản lý file gây ra mất dữ liệu của file, vậy
file không an toàn. Nếu chương trình điều phối gây ra đổ vỡ hệ thống, làm mất dữ liệu ta
mất cả giờ để đánh, vậy công việc của ta không an toàn. Bởi vậy, an ninh của một hệ
thống tính toán đòi hỏi hệ điều hành phải được thiết kế tốt và đáng tin cậy.

Việc phát triển các phần mềm đáng tin cậy không phải là vấn đề nghiên cứu của hệ điều
hành. Nó thuộc phạm vi của Công nghệ phần mềm, ta sẽ xem xét sau trong
Chương~\ref{}. Trong phần này, ta chỉ quan tâm đế các vấn đề an ninh liên quan riêng đến
hệ điều hành.

\subsection*{Tấn công từ bên ngoài}
Một trong những nhiệm vụ quan trọng của hệ điều hành là bảo vệ tài nguyên của máy tính
tránh khỏi các truy cập bất hợp lệ. Trong trường hợp hệ thống có nhiều người sử dụng, việc
bảo vệ này dựa trên ``tài khoản'' (account)--một tài khoản được quản lý trong trong hệ
điều hành như một mục gồm tên người dùng, mật khẩu và quyền truy cập gắn với người
dùng. Hệ điều hành dùng các thông tin này trong mỗi lần đăng nhập (login) để điều khiển
việc truy cập vào hệ thống.

Các tài khoản được tạo bởi một người gọi là \textbf{super user} hay \textbf{người quản
  trị} (administrator). Người này có quyền cao nhất trong hệ thống, và cũng phải đăng nhập
vào hệ thống để xác thực anh/chị ta là người quản trị (thường bởi tên và mật khẩu). Khi đã
đăng nhập, người quản trị có thể làm nhiều thay đổi bên trong hệ thống như: thay đổi gói
phần mềm, gán quyền cho người dùng, thực hiện các hoạt động bảo trì hệ thống,...

Dùng ``quyền rất cao'' này, người quản trị phải điều khiển hoạt động trong hệ thống để
kiểm tra các hành vi phá hoại hệ thống, do vô tình hay cố ý. Cũng có nhiều phần mềm công
cụ trợ giúp người quản trị, được gọi là \textbf{phần mềm kiểm tra} (auditing software). Nó
ghi lại và phân tích các hoạt động xảy ra bên trong hệ thống. Ví dụ, phần mềm kiểm tra có
thể cho biết những lần đăng nhập sai mật khẩu. Phần mềm kiểm tra cũng phát hiện các hoạt
động của một tài khoản người dùng không phù hợp với các hành vi của anh ta trong quá khứ,
để từ đó chỉ ra những người dùng không có thẩm quyền đã giành được quyền truy cập vào tài
khoản này. (ví dụ, với một người dùng bình thường chỉ dùng gói phần mềm xử lý văn bản và
bảng tính, bây giờ lại dùng các ứng dụng phần mềm kỹ thuật cao hoặc thực hiện các gói công
cụ không hợp lệ với quyền của anh ta.)

Phần mềm kiểm tra cũng được thiết kế để phát hiện các \textbf{phần mềm sniffing}, là phần
mềm khi được phép chạy trên hệ thống sẽ tìm cách ghi lại các hoạt động và sau đó thông báo
lại cho kẻ thâm nhập (intruder). Một ví dụ tuy cũ nhưng được biết rộng rãi là một chương
trình một phỏng thủ tục đăng nhập của hệ điều hành. Các chương trình như thế này có thể
làm cho người dùng khác nhầm tưởng họ đang giao tiếp với hệ điều hành, và cung cấp tên và
mật khẩu cho kẻ mạo danh.

Với mọi sự phức tạp về mặt kỹ thuật được gắn với máy tính, thật đáng ngạc nhiên là rào cản
chính của an ninh của máy tính là do sự thiếu thận trọng của người dùng. Họ chọn các mật
khẩu rất dễ đoán (như tên và ngày sinh), họ chia sẻ mật khẩu của họ với bạn bè, họ không
thay đổi mật khẩu thường xuyên, họ đưa các thiết bị lưu trữ khối off-line của mình đến chỗ
hỏng hóc khi họ chuyển các thiết bị này giữa các máy, họ cài đặt các phần mềm có thể gây
mất an toàn vào hệ thống. Để giải quyết những vấn đề này, hầu hết các hệ thống máy tính
lớn đều bắt ép người dùng tuân theo một số yêu cầu về an toàn để nâng cao ý thức trách
nhiệm của họ.

\subsection*{Tấn công từ bên trong}
Khi một kẻ thâm nhập (có thể là người dùng hợp lệ nhưng có ý đồ xấu) tấn công vào hệ
thống, chúng thường tìm cách thăm dò, tìm các thông tin quan tâm, hoặc cài đặt vào hệ
thống các phần mềm phá hoại. Điều này rất đơn giản nếu kẻ rình mò có thể truy cập hệ thống
bằng tài khoản của người quản trị. Đây chính là lý do tại sao mà mật khẩu của người quản
trị phải được bảo vệ một cách nghiêm ngặt. Tuy nhiên, nếu truy cập được vào tài khoản
người dùng thông thường, kẻ thâm nhập phải tìm cách làm đánh lừa hệ điều hành để truy cập
vào các vùng bị cấm. Ví dụ, kẻ truy cập có thể đánh lừa trình quản lý bộ nhớ cho phép một
tiến trình truy cập ra ngoài vùng nhớ dành cho nó, hoặc kẻ truy cập có thể cố gắng đánh
lừa trình quản lý file để lấy các file mà nó không có quyền truy cập.


Các CPU hiện đại được thiết kế có thêm các đặc tính nhằm ngăn chặn những vấn đề này. Ví
dụ, có thể xét nhu cầu hạn chế một tiến trình chỉ được truy cập vào vùng bộ nhớ mà trình
quản lý bộ nhớ gán cho nó; nếu không hạn chế, một tiến trình có thể xoá hệ điều hành trong
bộ nhớ chính và chiếm quyền điều khiển máy tính. Để ngăn chặn vấn đề này, các CPU được
thiết kế cho hệ điều hành đa nhiệm có thể chứa các thanh ghi đặc biệt cho phép hệ điều
hành lưu giữ các giới hạn trên và dưới của vùng nhớ được gán cho tiến trình. Và trong khi
thực hiện xử lý, CPU so sánh mỗi vùng nhớ được tham chiếu đến với các thanh ghi này để đảm
bảo nó nằm trong giới hạn cho phép. Nếu vùng nhớ tham chiếu đến vượt ra ngoài giới hạn
này, CPU tự động chuyển quyền điều khiển tới hệ điều hành (bằng cách thực hiện một dãy các
ngắt) để hệ điều hành có các xử lý phù hợp.

Dù đặc điểm ta mô tả ở trên có vẻ rất tinh tế, nhưng trên thực tế nó vẫn có vấn đề. Nếu
CPU không có thêm một vài đặc tính an toàn nữa, một tiến trình vẫn có thể truy cập vào các
ô nhớ bất hợp lệ bằng cách thay đổi thanh ghi đặc biệt (chứa giới hạn bộ nhớ). Có nghĩa
rằng, một tiến trình có thể truy cập một bộ nhớ bên ngoài đơn thuần bằng cách thay đổi các
giá trị trong thanh ghi chứa giới hạn trên và dưới của bộ nhớ, và do đó nó có thể sử dụng
không gian bộ nhớ thêm mà không cần hệ điều hành cho phép.

Để tránh các hoạt động kiểu này, CPU được thiết kế để có thể thực hiện trong một hoặc hai
\textbf{mức đặc quyền} (privilege level); ta sẽ gọi là ``mode đặc quyền'' và ``mode
không đặc quyền.'' Khi ở trong mode đặc quyền, CPU có thể thực hiện mọi lệnh có trong ngôn
ngữ máy của nó. Tuy nhiên, khi ở trong mode không đặc quyền, các lệnh mà nó có thể thực
hiện sẽ bị giới hạn. Các lệnh chỉ được phép chạy ở mode đặc quyền gọi là \textbf{lệnh đặc
  quyền}. (ví dụ lệnh đặc quyền điển hình là lệnh làm thay đổi nội dung các thanh ghi giới
hạn bộ nhớ và các lệnh làm thay đổi mode đặc quyền của CPU.) Mọi nỗ lực thực hiện một lệnh
đặc quyền khi CPU ở mode không đặc quyền đều gây ra một ngắt. Ngắt này chuyển CPU tới mode
đặc quyền và chuyển quyền điều khiển tới trình xử lý ngắt của hệ điều hành.

Khi máy được bật, CPU ở mode đặc quyền. Bởi vậy, khi kết thúc quá trình khởi động và hệ
điều hành chiếm quyền điều khiển, lúc này mọi lệnh máy đều có thể được hiện. Tuy nhiên,
mỗi khi hệ điều hành cho phép một tiến trình chạy một time slide, nó chuyển CPU tới mode
không đặc quyền bằng cách thực hiện một lệnh ``chuyển mode đặc quyền''. Và từ lúc này, hệ
điều hành sẽ được thông báo nếu tiến trình cố gắng thực hiện lệnh ở mode đặc quyền.

Các lệnh đặc quyền và điều khiển các mức đặc quyền là các công cụ chính sẵn có để các hệ
điều hành quản lý an ninh. Tuy nhiên, việc sử dụng các công cụ này là một công việc hết
sức phức tạp trong thiết kế hệ điều hành. Một lỗi nhỏ trong điều khiển mức đặc quyền có
thể gây ra thảm hoạ do những người lập trình có ý đồ xấu hoặc do các lỗi vô ý gây ra khi
lập trình. Nếu một tiến trình được phép thay đổi thay đổi bộ định thời gian điều khiển
việc chia sẻ thời gian thực của hệ thống có thể cho phép một tiến trình mở rộng time slide
và chiếm quyền điều khiển máy. Nếu một tiến trình được phép truy cập trực tiếp vào thiết
bị ngoại vi, vậy nó có thể đọc các file mà không bị giám sát bởi trình quản lý file. Nếu
một tiến trình được phép truy cập vào các ô nhớ bên ngoài vùng cho phép, nó có thể đọc và
thậm chí thay đổi dữ liệu đang được sử dụng bởi tiến trình khác.
  
\subsection*{Câu hỏi \& Bài tập}
\begin{enumerate}
\item Hãy cho vài ví dụ về việc chọn mật khẩu kém an toàn và giải thích tại sao chúng lại
  kém?

\item Các bộ xử lý của Intel sử dụng bốn mức đặc quyền. Tại sao người thiết kế lại quyết
  định dùng bốn mà không phải là ba hay năm mức?

\item Nếu một tiến trình trong hệ thống chia sẻ thời gian thực có thể truy cập vào vùng
  nhớ không được phép, làm thế nào nó có thể chiếm quyền điều khiển máy?
\end{enumerate}
%%% Local Variables: 
%%% mode: latex
%%% TeX-master: "../tindaicuong"
%%% End: 
